\section{Featherweight Java (note 11)} 
\subsection*{Esercizio 7.1}\
\\ 
Si noti che una class table pu\`o contenere definizioni di classi mutuamente ricorsive.
\linebreak Scrivere un esempio:


\vspace{0,5cm}
\noindent
\subparagraph{Svolgimento}:
Definiamo le due seguenti Classi

\begin{lstlisting}
Class A extends B {A(){super();} }

Class B extends A {B(){super();} }

\end{lstlisting}

\noindent
A e B sono mutamente ricorsive. La relativa Class Table non \`e per\`o ben fatta, in quanto contiene una relazione di sottotipo indotta con cicli.

\vspace{1cm}
\subsection*{Esercizio 7.2}\
\\ 
Dato il seguente codice:
\begin{lstlisting}
class A extends Object { A(){ super(); } }

class B extends Object { B(){ super(); } }

class Pair extends Object { 
	Object fst; Object snd;
	
	Pair(Object fst, Object snd){
		super(); this.fst=fst; this.snd=snd; 
	} 
	
	Pair setfst(Object newfst){ return new Pair(newfst, this.snd); }
}

\end{lstlisting}

\vspace{0,5cm}
Descrivere la semantica operazionale dei seguenti termini:

\begin{itemize}
\item new Pair(new A(), new B()).snd
\item (Pair) new Pair(new A(), new B())
\item new Pair(new A(),new B()).setfst(new B())
\item (Pair) (new Pair(new Pair(new A(), new B()), new A()).fst)).snd
\item (B) ((A)new C())
\end{itemize}

\vspace{0,5cm}
\subparagraph{Svolgimento}:

Per motivi di spazio i seguenti termini saranno abbreviati con delle lettere
\begin{itemize}
\item[$\ast$] K $\equiv$ new Pair(new A(),new B())
\item[$\ast$] H $\equiv$ Pair setfst(Object newfst) \{ return new Pair(newfst, this.snd) 
\item[$\ast$] J $\equiv$ CT(Pair)= class Pair extends Object {Object fst, Object snd; Pair(Object fst, Object snd),Pair setfst(Object newfst)}
\end{itemize}

\vspace{0,5cm}

\begin{itemize}
\item \fbox{new Pair(new A(), new B()).snd}

\begin{prooftree}
	\AxiomC{J}
	\AxiomC{fields(Object) = $\emptyset$}	
	\BinaryInfC{fields(Pair) = (fst : Object, snd : Object)}
	\LeftLabel{(PROJNEW)}	
	\AxiomC{snd $\in$ \{fst , snd\} }
	\BinaryInfC{new Pair(new A(), new B()).snd $\rightarrow$ new B()}
	\end{prooftree}
	
\vspace{1,5cm}

\item \fbox{(Pair) new Pair(new A(), new B())}


\begin{prooftree}
	\AxiomC{}	
	\RightLabel{(REFLEX)}	
	\UnaryInfC{Pair <: Pair}
	\LeftLabel{(CASTNEW)}
	\UnaryInfC{(Pair) new Pair(new A(), new B()) $\rightarrow$ new Pair(new A(), new B())}
	\end{prooftree}
\vspace{1,5cm}

\item \fbox{new Pair(new A(),new B()).setfst(new B())}



\begin{prooftree}
	\AxiomC{J}
	\AxiomC{H}	
	\BinaryInfC{mbody(setfst, Pair)= (newfst, new Pair(newfst, this.snd))}
	\LeftLabel{(INVKNEW)}	
	\AxiomC{|fst|=|newB()|}
	\BinaryInfC{K.setfst(new B()) $\rightarrow$ new Pair(newfst,this.snd)\{fst=new B(),this=K\}}
\end{prooftree}

\vspace{1,5cm}

	
\item \fbox{((Pair) (new Pair(new Pair(new A(), new B()), new A()).fst)).snd}

\vspace{0,5cm}
Passo 1

\begin{prooftree}
	\AxiomC{J}
	\AxiomC{field(Object) = $\emptyset$}
	\BinaryInfC{fields (Pair) = \{Object fst,Object snd\} }
	\AxiomC{$fst \in \tilde{f}$}
	\LeftLabel{(PROJNEW)}	
	\BinaryInfC{(new Pair(K, new A()).fst) $\rightarrow$ K }
	\LeftLabel{(CAST)}	
	\UnaryInfC{((Pair) (new Pair(K, new A()).fst)) $\rightarrow$ (Pair) K  }
	\LeftLabel{(FIELD)}
	\UnaryInfC{((Pair) (new Pair(K, new A()).fst)).snd $\rightarrow$ ((Pair) K).snd}
\end{prooftree}

\vspace{0,5cm}
Passo 2

\begin{prooftree}
	\AxiomC{}
	\LeftLabel{(REFLEX)}	
	\UnaryInfC{Pair <: Pair}
	\RightLabel{(CASTNEW)}	
	\UnaryInfC{((Pair) K $\rightarrow$ K }
	\LeftLabel{(FIELD)}
	\UnaryInfC{((Pair) K).snd $\rightarrow$ K.snd}
\end{prooftree}

\vspace{0,5cm}
Passo 3

\begin{prooftree}
	\AxiomC{J}
	\AxiomC{field(Object) = $\emptyset$}
	\BinaryInfC{fields (Pair) = \{Object fst,Object snd\} }
	\AxiomC{snd $\in$ $\widetilde{f}$}	
	\LeftLabel{((PROJNEW))}
	\BinaryInfC{(new Pair(new A(), new B()).snd $\rightarrow$ new B()}
\end{prooftree}

\vspace{0,5cm}

\item \fbox{(B) ((A)new C())}
\vspace{0,5cm}

Possiamo fare due ipotesi di subtyping: 
\vspace{0,5cm}	
	\begin{itemize}
	\item[-] C <: A <: B
			
Passo 1
\vspace{0,3cm}
	\begin{prooftree}
		\AxiomC{CT(C)=class C extends A\{...\}}	
		\UnaryInfC{C <: A}
		\RightLabel{(CASTNEW)}	
		\UnaryInfC{(A)new C() $\rightarrow$ new C() }
		\LeftLabel{(CAST)}
		\UnaryInfC{(B) ((A)new C()) $\rightarrow$ (B) new C() }
	\end{prooftree}				
\vspace{1cm}

Passo 2
\vspace{0,5cm}
	\begin{prooftree}
		
		\AxiomC{CT(A)=class A extends B\{...\}}
		\UnaryInfC{A <: B}
		\AxiomC{CT(C)=class C extends A\{...\}}
		\UnaryInfC{C <: A}	
		\BinaryInfC{C <: B}
		\LeftLabel{(CASTNEW)}
		\UnaryInfC{(B) new C() $\rightarrow$ new C() }
	\end{prooftree}	
\vspace{1,5cm}

	\item[-] C <: B <: A
	
	Passo 1
\vspace{0,5cm}
	\begin{prooftree}
		\AxiomC{CT(B)=class B extends A\{...\}}
		\UnaryInfC{B <: A}
		\AxiomC{CT(C)=class C extends B\{...\}}
		\UnaryInfC{C <: B}	
		\BinaryInfC{C <: A}
		\RightLabel{(CASTNEW)}	
		\UnaryInfC{(A)new C() $\rightarrow$ new C() }
		\LeftLabel{(CAST)}
		\UnaryInfC{(B) ((A)new C()) $\rightarrow$ (B) new C() }
	\end{prooftree}				
\vspace{1cm}

Passo 2
\vspace{0,3cm}
	\begin{prooftree}
		\AxiomC{CT(C)=class C extends B\{...\}}	
		\UnaryInfC{C <: B}
		\LeftLabel{(CASTNEW)}
		\UnaryInfC{(B) new C() $\rightarrow$ new C() }
	\end{prooftree}	
		\end{itemize}
\end{itemize}

\vspace{1cm}
\subsection*{Esercizio 7.3}\
\\
\
Scrivere un programma con override di un metodo e descriverne la valutazione, evidenziando il binding dinamico
per la chiamata del metodo riscritto.

\vspace{0,5cm}
\subparagraph{Svolgimento}:

\begin{lstlisting}
class A extends Object{ 

	Nat fst; Nat snd;
 
	A (Nat fst, Nat snd){ 
		super(); this.fst=fst; this.snd=snd; 
	}

	Nat print() { return fst; } 
}

class B extends A{  
	B (Nat fst, Nat snd){ 

	super(); this.fst=fst; this.snd=snd; 
	}
} 

Nat print() { return snd; }
 

\end{lstlisting}    

\vspace{0,6cm}

consideriamo ora il seguente termine:

\vspace{0,3cm}
\hspace{0,3cm}\textit{((A) new B(10,4)).print()}

\vspace{0,3cm}
Per motivi di spazio il seguente termine sar\`a abbreviato con una lettera
\begin{itemize}
\item[$\ast$] J $\equiv$ CT(B)= class B extends A {Object fst, Object snd; B(Nat fst, Nat snd),Nat print()}
\end{itemize}

\vspace{1cm}
Passo 1
\vspace{0,3cm}
	\begin{prooftree}
		\AxiomC{J}	
		\UnaryInfC{B <: A}
		\RightLabel{(CASTNEW)}
		\UnaryInfC{(A) new B(10,4) $\rightarrow$ new B(10,4)}
		\LeftLabel{(INVKRECV)}
		\UnaryInfC{((A) new B(10,4)).print() $\rightarrow$ new B(10,4).print()}
	\end{prooftree}	
		


\vspace{0,5cm}
Passo 2
\vspace{0,3cm}
	\begin{prooftree}
		\AxiomC{J}
		\AxiomC{Nat print() \{ return snd \} $\in$ $\widetilde{M}$}	
		\BinaryInfC{mbody($\emptyset$,B)=(snd, return snd)}
		\AxiomC{$\emptyset$ = $\emptyset$}
		\LeftLabel{(INVKNEW)}
		\BinaryInfC{ new B(10,4).print() $\rightarrow$ 4}
	\end{prooftree}	

\vspace{1cm}
\subsection*{Esercizio 7.4}\
\\
\
Perch\`e c'\`e una regola di tipo sia per l'Up-cast che per il Downcast, mentre c'\`e la sola regola di valutazione per
Up-cast, nella semantica operazionale?

\vspace{0,5cm}
\subparagraph{Svolgimento}:

\vspace{0,3cm}
L'operazione di Up-cast, applicata su termini non stuck, non produrr\`a mai termini stuck in quanto effettua una conversione da un tipo pi\`u informativo(sottoclasse) ad un tipo meno informativo(superclasse).
Ci\`o non \`e sempre valido con l'operazione di Down-cast, la quale effettua invece una conversione da un tipo meno informativo(superclasse) ad uno che richiede pi\`u informazione (sottoclasse), con conseguente mancanza di valori da assegnare ai campi dati della sottoclasse stessa.
    
La regola di typing di Down-cast, \`e inserita per permettere la valutazioni di termini, che il compilatore non considerebbe ben tipati anche se a run-time non evolvono in un termine stuck.

\vspace{0,5cm}
Consideriamo ad esempio il seguente termine: 

	- (B)((C) new A())
	
	con la seguente relazione di sottotipo (A<:B<:C)

\vspace{0,5cm}	
Valutazione di tipo:

\vspace{0,3cm}
\begin{prooftree}
		\AxiomC{$|\emptyset| = |\emptyset|$ $fields(A)=\emptyset$}
		\LeftLabel{(NEW)}
		\UnaryInfC{$\emptyset \vdash $ new A(): A }
		\AxiomC{A <: C}
		\LeftLabel{(UP-CAST)}
		\BinaryInfC{$\emptyset \vdash $ (C)new A(): C }		
		\AxiomC{B <: C}	
		\AxiomC{C $\neq$ B }
		\RightLabel{(DOWN-CAST)}
		\TrinaryInfC{$\emptyset \vdash $ (B) ((C)new A()): B }
	\end{prooftree}
	
\vspace{1cm}	
Applicazione regole di semantica operazionale:

\vspace{0,3cm}
Passo 1
\begin{prooftree}
		\AxiomC{CT(A)=class A extends B\{...\}}
		\UnaryInfC{A <: B}
		\AxiomC{CT(B)=class B extends C\{...\}}
		\UnaryInfC{B <: C}	
		\BinaryInfC{A <: C }
		\RightLabel{(CASTNEW)}	
		\UnaryInfC{(C)new A() $\rightarrow$ new A() }
		\LeftLabel{(CAST)}
		\UnaryInfC{(B) ((C)new A()) $\rightarrow$ (B) (new A()) }
	\end{prooftree}	

\vspace{0,3cm}
Passo 2

	\begin{prooftree}
		\AxiomC{CT(A)=class A extends B\{...\}}	
		\UnaryInfC{A <: B}
		\LeftLabel{(CASTNEW)}
		\UnaryInfC{(B) new A() $\rightarrow$ new A() }
	\end{prooftree}	

\vspace{0,5cm}
Risulta essere ben tipato grazie anche all'utilizzo della regola di tipo DOWN-CAST, ed a Run-time effettivamente non evolve in un termine stuck.




\vspace{1cm}			 
\subsection*{Esercizio 7.5}\
\\
\
Ha senso aggiungere a FJ la regola di sub-typing per i tipi freccia A$\rightarrow$B?


\vspace{0,5cm}
\noindent
\subparagraph{Svolgimento}:

\vspace{0,3cm}
In FJ le funzioni non appartengono all'insieme Termini t della sintassi del linguaggio.

I tipi freccia li troviamo solamente all'interno di metodi, i quali hanno gi\`a una regola di typing (INVK) che, essendo il sistema di  regole sintax-direct (o algoritmico), integra direttamente la regola di sub-typing.

%\vspace{1cm}			 
%\subsection*{\fbox{Esercizio 7.12}}\
%\\
%\
%Aggiungere a FJ il termine ClassCastException: come cambia la semantica operazionale? Le regole di tipo? Il teorema di Safety e i teoremi di Preservazione e Progressione? Aggiungere
%in seguito anche la possibilit\`a di gestire le eccezioni.

%\vspace{0,5cm}
%\noindent
%\textbf{Svolgimento}:
 