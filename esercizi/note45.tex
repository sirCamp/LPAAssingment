\section{I Tipi Semplici (note 45)}
 
\subsubsection*{Esercizio 3.2}
Provare che ogni sottotermine di un termine ben tipato \`e ben tipato.
\subparagraph{Svolgimento}
Per la risoluzione di questo esercizio \`e necessario utilizzare il metodo dell'induzione sui passi di derivazione, ovvero sull'altezza dell'albero di derivazione.\\
Se $M$ \`e un termine ben tipato vuol dire che il giudizio di tipo $\Gamma$ $\vdash$ $M$ : $T$ \`e derivabile con un albero di derivazione di altezza $h$ e quindi l'induzione viene fatta su $h+1$.


\begin{description}

 \item[Caso Base $h=1$] A questo livello il termine e il sottotermine coincidono. Questo perch\`e l'albero consiste nell'applicazione di un assioma e quindi il termine $M$ non contiene sotto-termini, quindi la propriet\`a \`e verificata.

 \item[Caso Induttivo $h+1$ ] Questo caso \`e composto dall'applicazione della regola e dai vari sotto-alberi che provano le ipotesi della regola.\\I sotto-alberi in questione, riguardano il giudizio dei sotto-termini di $M$ e sono di altezza inferiore rispetto a questo albero ($h$). Possiamo quindi applicare l'ipotesi induttiva per definire che i sotto-termini dei sotto-termini di $M$ sono ben tipati e quindi di conseguenza lo sono anche i sotto-termini di $M$. A questo punto la propriet\`a risulta essere verificata.
 
 \end{description}
 
 \`E possibile inoltre, utilizzare un altra dimostrazione che sfrutta la \emph{struttura di $M$} e il \emph{Lemma di Inversione}.
 
 
 \begin{itemize}[label=$\star$]

 \item $M$ $\equiv$ $v$:  In questo caso $M$ \`e un valore, come ad esempio $true$, $false$ o $n$ e quindi, poich\`e questi sono termini privi di sotto-termini la propriet\`a risulta essere verificata.

 \item $M$ $\equiv$ $A+B$:  In questo caso, per il Lemma di Inversione se $M$ \`e ben tipato, allora lo sono anche $A$ e$B$ ( tipo $Nat$ )
 
 \item $M$ $\equiv$ $A-B$:  questo caso \`e analogo al precedente

 \item $M$ $\equiv$ $if$ $C$ $then$ $A$ $else$ $B$:  In questo caso, per il Lemma di Inversione se $M$ \`e ben tipato, allora $C$ \`e ben tipato con tipo $Bool$ mentre $A$ e $B$ sono ben tipati con tipo $T$
 
  \item $M$ $\equiv$ $A$ $B$: In questo caso, per il Lemma di Inversione se $M$ \`e ben tipato, allora $A$ \`e ben tipato con tipo $T_1 \rightarrow T$ mentre $B$ \`e ben tipato con tipo $T$
  
    \item $M$ $\equiv$ $fn$ $x.N$: In questo caso, per il Lemma di Inversione $N$ \`e ben tipato con tipo $T$
 \end{itemize}
 
 
\subsubsection*{Esercizio 3.12} 
Dimostrare subject reduction per induzione sulla derivazione di $\Gamma\:\vdash$ M : T.

\subparagraph*{Svolgimento}

Si vuole quindi dimostrare che:

\begin{center}
	Se $\Gamma \vdash{} M : T$ e $M \to{} M'$, allora $\Gamma \vdash{} M' : T$.
\end{center}
%La dimostrazione per induzione va fatta sul numero di passi che $M$ portano a $M'$.
%Dobbiamo ragionare su quella che \`e la lunghezza dei cammini della derivazione:

%Caso Base l = 0:\\
%Otteniamo che $M$ = $M'$ e la tesi \`e trivialmente vera.

%Caso Induttivo l+1:\\
%In questo passo ho che $M \rightarrow^{*}_l M_1 \rightarrow M'$, e per ipotesi induttiva ho che $\Gamma \vdash M_1 : T$.\\
%A questo punto ho che $M_1 \rightarrow M'$ e quindi posso applicare subject-reduction per ottenere che $\Gamma \vdash M' : T$.

%Ovviamente grazie ai teoremi di Sostituzione, Preservazione e Progressione so per certo che qualsiasi sia il contesto %il risultato non cambia

\begin{description}


 \item[True] $\Gamma \vdash{} M : T$ \'e $\Gamma \vdash{} \true{} :
  \Bool$.
  Il teorema \'e vacuamente vero per questa derivazione siccome
  $\not \exists{M'}. M \to{} M'$.
  
 \item[False]
  $\Gamma \vdash{} M : T$ \'e $\Gamma \vdash{} \false{} : \Bool$.
  Analogo a True.

\item[Nat]
  $\Gamma \vdash{} M : T$ \'e $\Gamma \vdash{} n : \Nat$. Analogo a True.

\item[Var] $\Gamma \vdash{} M : T$ \'e $\Gamma \vdash{} x : T$.
  Analogo a True.

\item[Fun] $\Gamma \vdash{} M : T$ \'e
  $\Gamma \vdash{} fn x\:T_1 C : T_1 \to{} T_2$. Analogo a True.
  
  \item[Sum]
  $\Gamma \vdash{} M : T$ \'e $\Gamma \vdash{} A + B : \Nat$.
 
  Per quanto assunto dalla regola di tipo \emph{Sum}, sappiamo per certo che
  anche $A$ e $B$ hanno tipo $\Nat$ sotto contesto $\Gamma$.

  Supponiamo che $A + B \to{} M'$ (sappiamo che $M'$ \'e unico poich\'e l'esecuzione
   \'e  deterministica).Tale riduzione pu\'o essere:

  \begin{itemize}
    \item \emph{Sum}: $M' \equiv{} n$. Per la regola di tipo
      \emph{Nat}, si ha che $T \equiv{} \Nat$ e che
      $\Gamma{} \vdash{} M' : \Nat$;
    \item \emph{Sum-Left}: $M' \equiv{} A' + B$. 
    Per gli assunti della regola $\emph{Sum}$ ho che $\Gamma \vdash A : \Nat$ e per $\emph{Sum-Left}$ $A \to A'$.
    Posso quindi applicare l'ipotesi induttiva\footnote{perch \'e  la derivazione $\Gamma \vdash A :\Nat$ richiede meno passaggi}
    per ottenere il giudizio $ \Gamma \vdash A' : \Nat$.
    Sono quindi soddisfatte le premesse della regola $\emph{Sum}$ per $M'$ e quindi  $\Gamma{} \vdash{} M' : \Nat \equiv T$;

    \item \emph{Sum-Right}: $M' \equiv{} v + B'$. Il ragionamento  \'e  analogo al caso $\emph{Sum-Left}$.
  \end{itemize}

\item[Minus]
  $\Gamma \vdash{} M : T$ \'e $\Gamma \vdash{} A - B : \Nat$. Analogo a
  \emph{Sum}.

\item[IfThenElse] $\Gamma \vdash{} M : T$ \'e 
  $\Gamma \vdash{} \IF{M_1}{M_2}{M_3}$.
  
  Dal giudizio di tipo abbiamo che:

  \begin{itemize}
  	\item $\Gamma \vdash M_1 : \Bool$;
    \item $\Gamma \vdash{} M_2 : T$;
    \item $\Gamma \vdash{} M_3 : T$.
  \end{itemize}

  Supponiamo che $\IF{M_1}{M_2}{M_3} \to{} M'$ (sappiamo che $M'$ \'e unico poich\'e l'esecuzione
  \'e deterministica). Tale riduzione pu\'o essere:

  \begin{itemize}
    \item \myrule{If}: $M' \equiv{} \Gamma \vdash{} \mbox{if } M'_1
      \mbox{ then } M_2 \mbox{ else } M_3$. Dato che il giudizio $\Gamma \vdash M_1 : \Bool$ richiede una derivazione in meno e che $M_1 \to M_1'$, possiamo applicare l'ipotesi induttiva per ottenere il giudizio $\Gamma \vdash M_1' : \Bool$. Si ha quindi che la regola di tipo \myrule{IfThenElse} continua a valere per il termine $M'$ e dato che i termini $M_2$ e $M_3$ non sono cambiati, anche $M'$ ha tipo $T$;
    \item \myrule{If-True}: $M' \equiv{} M_2$. $M_2$ ha tipo $T$ e quindi anche $M'$ ha tipo $T$;
    \item \myrule{If-False}: $M' \equiv{} M_3$.  Analogo a \myrule{If-True}
  \end{itemize}
  
  
\item[App] $\Gamma \vdash{} M : T$ \'e  $\Gamma \vdash{} A \: B : T_2$.
  Dal giudizio di tipo abbiamo che:

  \begin{itemize}
    \item $\Gamma \vdash{} A : T_1 \to{} T_2$;
    \item $\Gamma \vdash{} B : T_1$.
  \end{itemize}

  Supponiamo che $A \: B \to{} M'$ (sappiamo che $M'$ \'e  unico poich\'e  l'esecuzione
  \'e  deterministica). Tale riduzione pu\'o essere:

  \begin{itemize}
    \item \myrule{Beta}: $M' \equiv{} \Gamma \vdash{}
      Z[x \coloneqq{} B]$\footnote{dato $A \equiv{} FN{x}{T_1}Z$}.
      Per la regola di tipo $\myrule{Fun}$ ho che $Z : T_2$ e quindi, per
       il Substitution Lemma, $\Gamma \vdash{} Z[x \coloneqq{} B] : T_2$. Per
      questo motivo si ha che $\Gamma \vdash{} M' : T_2$ con $T_2 \equiv T$;
    \item \myrule{App-1}: $M' \equiv{} A'\: B$. Dato che il giudizio $\Gamma \vdash A : T_1 \to T_2$ richiede una derivazione con meno passaggi e che $A \to A'$ posso applicare l'ipotesi induttiva per ottenere il giudizio $A' : T_1 \to T_2$. Il termine $M'$ soddisfa le premesse della regola di tipo \myrule{App} e quindi $M'$ ha tipo $T_2 \equiv T$.

    \item \myrule{App-2}: $M' \equiv{} v\: B'$. Analogo ad \myrule{App-1}.
  \end{itemize}

 
\end{description}




\subsubsection*{Esercizio 3.13}

Vale l'opposto di subject reduction, i.e. se $\Gamma\:\vdash$ $M'\: :  $T e $M\:\rightarrow$ $M'$, allora $\Gamma\:\vdash$ $M$ : $T$ (detto subject expansion)? Dimostrarlo oppure dare un controesempio.
\subparagraph*{Svolgimento}
La risposta \`e no:

\`E sufficiente pensare a $M \rightarrow M^{\prime}$ con $\Gamma\vdash$ $M^{\prime}$ : $T$ e che 
$\Gamma\nvdash$ $M$: $T$\\
Quanto riportato sopra pu\`o essere semplificato tramite questo esempio:\\

Supponiamo di avere $M^{\prime}$ $\equiv$ $A$ : $T$ e $M$ $\equiv$ $if$ $true$ $then$ $A$ $else$ $B$, ovvero il caso in cui $M$ $\rightarrow$ $M^{\prime}$ per la regola (If-True).\\
In questo caso la derivazione di $\Gamma$ $\vdash$ $M$ : $T$ riesce ad ottenere usando la regola $IF-THEN-ELSE$, tuttavia questo impone che nel contesto ci siano questi giudizi:

\begin{itemize}
\item $\Gamma$ $\vdash$ $true$:$Bool$ che \`e un assioma
\item $\Gamma$ $\vdash$ $A$:$T$ soddisfatto per l'ipotesi 
\item  $\Gamma$ $\vdash$ $B$:$T$ purtroppo questa informazione non si riesce a reperire dall'ipotesi ne dal contesto e pertanto non c'\`e garanzia sull'uguaglianza di tipi di $A$ e $B$\\
Pertanto possiamo creare un termine $N$ come $if$ $true$ $then$ $4$ $else$ $false$, da cui si riesce ad attenere $M'$ $\equiv$ $A$ $\equiv$ $4$. Questo termine nonostante esegua risulta essere non ben-tipato.


\end{itemize}







\subsubsection*{Esercizio 3.14}
Se al posto della regola $(APP)\:$ si definisce la regola seguente:

\begin{prooftree} 
	\AxiomC{$\Gamma\:\vdash$ $M\: :  $T$\:\rightarrow\:$T}
	\AxiomC{$\Gamma\:\vdash$ $N\ :$T}
	
	\LeftLabel{\textsc{($APP^{\prime}$)}}
	\BinaryInfC{$\Gamma\:\vdash$ $M$ $N$ : $T$  }
\end{prooftree} 

sarebbe ancora vero il teorema di safety?
\subparagraph*{Svolgimento}
Si, lo si pu\`o dimostrare considerando che $APP^{\prime}$ non \`e altro che una specificazione di $APP$: se poniamo $X$ uguale all'insieme dei tipi di $APP$, $Y$ \`e un sottoinsieme di $X$ corrispondente ai possibili tipi di $APP^{\prime}$.\\
In sostanza la regola restringere l'insieme dei programmi realizzabili con il precedente type system

Di conseguenza il teorema di safety vale anche per $APP^{\prime}$ poich\`e la dimostrazioni continuano a valere.

\subsubsection*{Esercizio 3.15}
Se al posto delle regole $(APP)$ e $(FUN)$ si definissero le seguenti regole:
\begin{prooftree} 
	\AxiomC{$\Gamma\:\vdash$ $M\: :  $T$\:\rightarrow\:$T}
	\AxiomC{$\Gamma\:\vdash$ $N\ :$T}
	
	\LeftLabel{\textsc{($APP^{\prime}$)}}
	\BinaryInfC{$\Gamma\:\vdash$ $M$ $N$ : $T$  }
\end{prooftree} 

e


\begin{prooftree} 
	\AxiomC{$\Gamma\:$, x : $T1\vdash$ $M\: : $T}
	\AxiomC{}
	\LeftLabel{\textsc{($FUN^{\prime}$)}}
	\BinaryInfC{$\Gamma\:\vdash$ $fn$ x : $T1.M:T\rightarrow\:$T  } 
\end{prooftree} 

sarebbe ancora vero il teorema di safety?

\subparagraph*{Svolgimento}

La risposta e' no:\\
In questo caso la $FUN^{\prime}$ non fa alcun controllo sul tipo dell'argomento e quindi esiste la possibilit\`a di arrivare ad un passo della derivazione che produce un termine STUCK, invalidando il teorema stesso. Questo perch\`e la nuova regola risulta essere pi\`u stringente rispetto a prima visto che ammette solo tipi uguali al tipo di ritorno senza controllare effettivamente che la signatura dei parametri sia conforme a quanto richiesto.\\

Supponiamo di avere il termine:\\
$M$ $=$ $(fn$ $x$:$Bool.if$ $x$ $then$ $1$ $else$ $0)$ $(1)$ 

e creiamo il suo albero di derivazione:



\scalebox{.90}{
\parbox{1cm}{
\begin{prooftree} 
	\AxiomC{$ \checkmark $}
	\UnaryInfC{$x$:$Bool$ $\in$  $\Gamma$ }
	\RightLabel{\textsc{($Var$)}}
	\UnaryInfC{$\Gamma$ $\vdash$ $x$:$Bool$ }
	\AxiomC{$ \checkmark $}
	\LeftLabel{\textsc{($Nat$)}}
	\UnaryInfC{$\Gamma$ $\vdash$ $1$:$Nat$}
	\AxiomC{$ \checkmark $}
	\LeftLabel{\textsc{($Nat$)}}
	\UnaryInfC{$\Gamma$ $\vdash$ $0$:$Nat$}
	\LeftLabel{\textsc{($IF-THEN-ELSE$)}}
	\TrinaryInfC{$x$:$Bool$ $\vdash$ $if$ $x$ $then$ $1$ $else$ $0$ :$Nat$}
	\LeftLabel{\textsc{($FUN^{\prime}$)}}
	\UnaryInfC{$\emptyset \vdash fn$ $x$:$Bool.if$ $x$ $then$ $1$ $else$ $0$ :$Nat \rightarrow$ :$Nat$}
	\AxiomC{$ \checkmark $}
	\LeftLabel{\textsc{($Nat$)}}
	\UnaryInfC{$\emptyset \vdash 1$:$Nat$}
	\LeftLabel{\textsc{($APP^{\prime}$)}}
	\BinaryInfC{$\emptyset \vdash (fn$ $x$:$Bool.if$ $x$ $then$ $1$ $else$ $0)$ $(1)$  } 
\end{prooftree}
}}

In questo caso possiamo osservare che il termine $M$ risulta essere ben-tipato secondo le nuove regole di tipo.\\
Tuttavia, \`e evidente che l'invocazione della funzione in questione porta $M$ ad un termine stuck, questo perch\`e il sottotermine $N$ $=$ $if$ $x$ $then$ $1$ $else$ $0$ si aspetta un $Bool$, infatti le regole della semantica operazionale che gestiscono l'$if$, permettono di avanzare nella computazione.\\
Nel caso in cui la guardia non sia un valore, la computazione procede con la valutazione della guardia, altrimenti procede il ramo $TRUE$ o con il ramo $FALSE$.\\
In questo caso, poich\`e il valore passato alla funzione \`e un intero, e quindi gi\`a un valore le uniche regole applicabili sarebbero $(IF-TRUE)$ e $(IF-FALSE)$ ma poich\`e la guardia \'e 1 non sono applicabili e quindi si otterrebbe un termine STUCK.


\subsubsection*{Esercizio 3.16}
Se si aggiungessero al sistema di tipi i seguenti due assiomi

\begin{prooftree} 
	\AxiomC{}
	\AxiomC{}
	\LeftLabel{\textsc{($TRUE^{\prime}$)}}
	\BinaryInfC{$\Gamma\:\vdash$ true : $Nat$  } 
\end{prooftree}

e

\begin{prooftree} 
	\AxiomC{}
	\AxiomC{}
	\LeftLabel{\textsc{($FALSE^{\prime}$)}}
	\BinaryInfC{$\Gamma\:\vdash$ false : $Nat$  } 
\end{prooftree}

sarebbe ancora vero il teorema di safety?
\subparagraph*{Svolgimento}
La risposta \`e no:\\
Data la definizione di Teorema di Safety, \`e evidente che viene meno la stessa ipotesi ovvero che M sia un valore o un che non evolva in un termine STUCK.

Con queste regole di tipo, infatti, sarebbe permessa anche l'operazione SUM che genererebbe un termine STUCK, invalidando il teorema stesso.\
Supponiamo di avere il termine: $M$ $=$ $true+4$


\begin{prooftree} 
	\AxiomC{$ \checkmark $}
	\LeftLabel{\textsc{($Nat$)}}
	\UnaryInfC{$\empty \vdash 4$:$Nat$}
	\AxiomC{$ \checkmark $}
	\LeftLabel{\textsc{($Nat$)}}
	\UnaryInfC{$\empty \vdash true$:$Nat$}
	\LeftLabel{\textsc{($SUM$)}}
	\BinaryInfC{$\empty \vdash$ $true$ $+$ $4$  } 
\end{prooftree}

Anche in questo caso, come nel precedete, notiamo che le regole di tipo permettono la derivazione dell'albero.\\
Tuttavia, se applicassimo le regole della semantica operazionale non potremo avanzare perch\`e entrambi gli addendi sono gi\'a dei valori e quindi non posso applicare $SUM-LEFT$ o $SUM-RIGHT$ ma non posso neppure applicare $SUM$ perch\'e genero un termine STUCK in quanto $true$ non pu\'o essere sommato a $4$.

\subsubsection*{Esercizio 3.17}

Dimostrare il seguente fatto: se $\Gamma\:\vdash$ $M$ : $T$ \`e derivabile, allora  \textit{fv($M$)}  $\subseteq$ \textit{Dom($\Gamma$)}
\subparagraph*{Svolgimento}

Si procede con la dimostrazione per induzione:

\begin{itemize}


\item Caso Base 
		\begin{itemize}[label=$\star$]

		\item \textbf{True}  $\Gamma \vdash{} M : T$ \'e $\Gamma \vdash{} \true{} : \Bool$.
  		Il teorema \'e vacuamente vero per questa derivazione siccome $\not \exists{M'}. M \to{} M'$.\\
  		Il tutto si pu\`o riassumere con $fv(M)$ $=$ $fv(true)$ $=$ $\emptyset$ $\subseteq$ $\emph{Dom(}\Gamma\emph{)}$

  
 		\item \textbf{False} $\Gamma \vdash{} M : T$ \'e $\Gamma \vdash{} \false{} : \Bool$. Analogo a True.

		\item \textbf{Nat} $\Gamma \vdash{} M : T$ \'e $\Gamma \vdash{} n : \Nat$. Analogo a True.

		\item \textbf{Var}  Se $\Gamma$ $\vdash$ $x$ : $T$ , allora per la regola $(Var)$ possiamo asserire che $x$ : $T$ $\in$ $\Gamma$ e di conseguenza abbiamo che  $fv(M)$ $=$ $fv(x)$ $=$ $\{x\}$ $\subseteq$ $\emph{Dom(}\Gamma\emph{)}$

  
		\end{itemize}
		In questo passo il fatto \`e un assioma, questo perch\`e non esistono variabili libere quindi $fv(M)$ \`e $\emptyset$ e il vuoto e un sottoinsieme del $Dom(\Gamma)$

\item Caso Induttivo
		
		\begin{itemize}[label=$\star$]

		\item \textbf{SUM}  in questo caso $M$ $\equiv$ $A$ $+$ $B$ e quindi per definizione le variabili libere di $M$ sono $f$ $v(A$ $+$ $B)$ $=$ $f$ $v(A)$ $\cup$ $f$ $v(B)$.\\ Per ipotesi induttiva sappiamo che $fv(A)$ e $fv(B)$ $\subseteq$ $\emph{Dom(}\Gamma\emph{)}$.\\Poich\`e, come anticipato, $fv(M)$ $\equiv$ $fv(A)$ $\cup$ $fv(B)$ possiamo dedurre che $fv(M)$ $\subseteq$ $\emph{Dom(}\Gamma\emph{)}$.
  		
  
 		\item \textbf{MINUS} in questo caso $M$ $\equiv$ $A$ $-$ $B$ e la sua dimostrazione risulta essere analoga a quella di SUM

		\item \textbf{IF-THEN-ELSE} in questo caso $M$ $\equiv$ $if$ $C$ $then$ $A$ $else$ $B$ e quindi per definizione, le variabili libere di $M$ sono $fv(M)$ $=$ $fv(A)$ $\cup$ $fv(B)$ $\cup$ $fv(C)$.\\ Per ipotesi induttiva sappiamo che $fv(A)$, $fv(B)$ e $fv(C)$ $\subseteq$ $\emph{Dom(}\Gamma\emph{)}$.\\Poich\`e, come anticipato, $fv(M)$ $\equiv$ $fv(A)$ $\cup$ $fv(B)$ $\cup$ $fv(C)$ possiamo dedurre che $fv(M)$ $\subseteq$ $\emph{Dom(}\Gamma\emph{)}$.

		\item \textbf{FUN}  in questo caso $M$ $\equiv$ $fn$ $x.N$ e quindi per definizione le variabili libere di $M$ sono 
		$fv(M)$ $=$ $fv(N)\textbackslash \{x\}$ ($fv(M)$ $\subseteq$ $fv(N)$).\\ Per ipotesi induttiva sappiamo che $fv(N)$ $\subseteq$ $\emph{Dom(}\Gamma\emph{)}$.\\Poich\`e, come anticipato, $fv(M)$ $\equiv$ $fv(N)\textbackslash \{x\}$ possiamo dedurre che $fv(M)$ $\subseteq$ $\emph{Dom(}\Gamma\emph{)}$. 
		
		\item \textbf{APP} in questo caso $M$ $\equiv$ $A$ $B$ e quindi per definizione, le variabili libere di $M$ sono $fv(M)$ $=$ $fv(A)$ $\cup$ $fv(B)$.\\ Per ipotesi induttiva sappiamo che $fv(A)$ e $fv(B)$ $\subseteq$ $\emph{Dom(}\Gamma\emph{)}$.\\Poich\`e, come anticipato, $fv(M)$ $\equiv$ $fv(A)$ $\cup$ $fv(B)$ possiamo dedurre che $fv(M)$ $\subseteq$ $\emph{Dom(}\Gamma\emph{)}$. 
  
		\end{itemize}

\end{itemize}

\subsubsection*{Esercizio 3.18}
Ricostruire il tipo dei seguenti termini:
\begin{itemize}
\item $fn$ $x$:$T1.fn$ $y$:$T2.if$ $y$ then $x$ else true
\item $fn$ $x:Nat:\rightarrow\:$Bool.x
\item $fn$ $f:T.fn$ $x:T'.f$( if true  then  $x$  else  $f$ $x$)
\item $fn$ $f:T1.fn$ $g:T2.$ if ($f$ ($g$ true)) then $f$ ($fn$ $x:$T3.true) else $f$($fn$ $x:T4.x$)
\end{itemize}

\subparagraph*{Svolgimento}
\begin{enumerate}[label=\textbf{\alph*}), leftmargin=*]


\item $fn$ $x$:$T_1.fn$ $y$:$T_2.if$ $y$ then $x$ else true  \\
		\scalebox{.75}{
		\parbox{1cm}{
		\begin{prooftree}
			\AxiomC{y:Bool $\in$ $\Gamma$}
			\LeftLabel{\textsc{(VAR)}} 
			\LeftLabel{\textsc{(VAR)}}
			\UnaryInfC{$\Gamma$,$x$:$T_1$,$y$:$T_2$ $\vdash$ $y$ : $Bool$}
			\LeftLabel{\textsc{(VAR)}}
			\AxiomC{$x$:$T_3$ $\in$ $\Gamma$}
			\LeftLabel{\textsc{(VAR)}}
			\UnaryInfC{$\Gamma$,$x$:$T_1$,$y$:$T_2$ $\vdash$ $x$ : $T_3$}
			\LeftLabel{\textsc{(VAR)}}
			\AxiomC{$\checkmark$}
			\LeftLabel{\textsc{(TRUE)}}
			\UnaryInfC{$\Gamma$,$x$:$T_1$,$y$:$T_2$ $\vdash$ $true$ : $Bool$}
			\LeftLabel{\textsc{(IF-THEN-ELSE)}}
			\TrinaryInfC{$x$:$T_1$,$y$:$T_2$ $\vdash$ $if$ $y$ $then$ $x$ $else$ $true$ : $T_3$}
			\LeftLabel{\textsc{(FUN)}}
			\UnaryInfC{$x$:$T_1$ $\vdash$  $fn$ $y$:$T_2.if$ $y$ $then$ $x$ $else$ $true$ : $T_2 \rightarrow T_3$}
			\LeftLabel{\textsc{(FUN)}}
			\UnaryInfC{$\emptyset\:\vdash$  $fn$ $x$:$T_1.fn$ $y$:$T_2.if$ $y$ $then$ $x$ $else$ $true$ : $T_1 \rightarrow T_2 \rightarrow T_3$}
		\end{prooftree}}
		}\vspace{1cm}
		
		Il risultato finale \`e dato dal fatto che $T_1$ = $T_2$ = $T_3$ = $Bool$  e quindi il tipo finale risulta: $Bool \rightarrow Bool \rightarrow Bool$
		\vspace{1.5cm}
		
		\item $fn$ $x:Nat:\rightarrow$ $Bool.x$  \\ 
		\begin{prooftree}
		
			\AxiomC{$x:T_2$ $\in$ $\Gamma,$ $x$:$Nat:\rightarrow Bool$}
			\LeftLabel{\textsc{(VAR)}}
			\UnaryInfC{$\Gamma,$ $x$:$Nat:\rightarrow Bool$ $\vdash$  $x$:$T_2$}
			\LeftLabel{\textsc{(FUN)}}
			\UnaryInfC{$\emptyset\:\vdash$  $fn$ $x:Nat:\rightarrow$ $Bool.x$ : $T_1 \rightarrow T_2$}
		\end{prooftree} 
		
		Il risultato finale \`e dato dal fatto che $T_2$ = $Nat \rightarrow Bool$  e quindi il tipo finale risulta:\\ $(Nat \rightarrow Bool)  \rightarrow (Nat \rightarrow Bool)$
		\vspace{1.5cm}
		
		
		\item $fn$ $f:T.fn$ $x:T'.f(if$ $true$  $then$  $x$  $else$  $f$ $x)$ \\ 
		\scalebox{.95}{
		\parbox{1cm}{ \vspace{1.5cm}
		\begin{prooftree}
			
			%\AxiomC{$f$:$T$,$x$:$T_1$ $\vdash$ $f$:$T_3 \rightarrow T_2$}
			
			%%else
			
			%\AxiomC{$f$:$T$,$x$:$T_1$ $\vdash$ $f:T_4 \rightarrow T$}
			%\AxiomC{$T_1 = T_4$}
			%\LeftLabel{\textsc{(VAR)}}
			%\UnaryInfC{$f$:$T$,$x$:$T_1$ $\vdash$ $x:T_4$}
			%\LeftLabel{\textsc{(APP)}}
			%\BinaryInfC{$f$:$T$,$x$:$T_1$ $\vdash$ $f x$ : $T_3$}
			
			%%			
					
			%%then
			
			%\AxiomC{$T_1 = T_3$}
			%\LeftLabel{\textsc{ }}
			%\UnaryInfC{$f$:$T$,$x$:$T_1$ $\vdash$ $x$ : $T_3$}
						
			%%			
			%% if
			%\AxiomC{$\checkmark$}
			%\LeftLabel{\textsc{(TRUE)}}
			%\UnaryInfC{$f$:$T$,$x$:$T_1$ $\vdash$ $true$ : $Bool$}
			%%
			%\LeftLabel{\textsc{(IF-THEN-ELSE)}}
			%\TrinaryInfC{$f$:$T$,$x$:$T_1$ $\vdash$ $ if true$ then $x$ else $f$ $x$ : $T_3$}
			
			\AxiomC{A}
			\UnaryInfC{$f$:$T$,$x$:$T^{\prime}$ $\vdash$ $f$:$T^{\prime\prime\prime\prime} \rightarrow T^{\prime\prime}$}
			\AxiomC{B}
			\UnaryInfC{$f$:$T$,$x$:$T^{\prime}$ $\vdash$ $ if true$ then $x$ else $f$ $x$ : $T^{\prime\prime\prime}$}
			\LeftLabel{\textsc{(APP)}}
			\BinaryInfC{$f$:$T$,$x$:$T^{\prime}$ $\vdash$ $f$ $(if$ $true$ $then$ $x$ $else$ $f$ $x$)}
			\LeftLabel{\textsc{(FUN)}}
			\UnaryInfC{$f$:$T$ $\vdash$  $fn$ $x$:$T^{\prime}.f(if$ $true$  $then$  $x$  $else$  $f$ $x)$:$T^{\prime}\rightarrow T^{\prime\prime} $}
			\LeftLabel{\textsc{(FUN)}}
			\UnaryInfC{$\emptyset\:\vdash$  $fn$ $f:T.fn$ $x:T^{\prime}.f(if$ $true$  $then$  $x$  $else$  $f$ $x)$ : $T \rightarrow T^{\prime}\rightarrow T^{\prime\prime} $}
		\end{prooftree}}
		}\vspace{1cm}
		
		
		
		\begin{itemize}[label=$\star$]
		
		\item 	Albero A \\
		  
		\begin{prooftree}
			\AxiomC{ $f$:$T^{\prime\prime\prime\prime} \rightarrow T^{\prime\prime}$ $\vdash$ $\Gamma$}
			\LeftLabel{\textsc{(VAR)}}	
			\UnaryInfC{$f$:$T$,$x$:$T^{\prime}$ $\vdash$ $f$:$T^{\prime\prime\prime\prime} \rightarrow T^{\prime\prime}$}
		
		\end{prooftree} 
		
		\item Albero B\\
		 \scalebox{.9}{
		\parbox{1cm}{ \vspace{1.5cm}
		\begin{prooftree}
		
		
			%%			
			%% if
			\AxiomC{$\checkmark$}
			\LeftLabel{\textsc{(TRUE)}}
			\UnaryInfC{$f$:$T$,$x$:$T^{\prime}$ $\vdash$ $true$ : $Bool$}
			
			
			%%			
					
			%%then
			
			\AxiomC{$x$ : $T^{\prime\prime}$ $\in$ $\Gamma$}
			\LeftLabel{\textsc{(VAR) }}
			\UnaryInfC{$f$:$T$,$x$:$T^{\prime}$ $\vdash$ $x$ : $T^{\prime\prime}$}
						
			%%else
			
			
			\AxiomC{C}
			\LeftLabel{\textsc{(APP)}}
			\UnaryInfC{$f$:$T$,$x$:$T^{\prime}$ $\vdash$ $f x$ : $T^{\prime\prime}$}
			
			%%
			\LeftLabel{\textsc{(IF-THEN-ELSE)}}
			\TrinaryInfC{$f$:$T$,$x$:$T^{\prime}$ $\vdash$ $if$ $true$ $then$ $x$ $else$ $f$ $x$ : $T^{\prime\prime}$}
		
		\end{prooftree}}}	 
	
		
		\item Albero C (ramo else dell'if)\\
		 
		\begin{prooftree}
		
			\AxiomC{$f:T^{\prime\prime\prime}$ $\in$ $\Gamma$}
			\LeftLabel{\textsc{(VAR)}}
			\UnaryInfC{$f$:$T$,$x$:$T^{\prime}$ $\vdash$ $f:T^{\prime\prime\prime} \rightarrow T^{\prime\prime}$}
			\AxiomC{$x:T^{\prime\prime\prime}$ $\in$ $\Gamma$}
			\LeftLabel{\textsc{(VAR)}}
			\UnaryInfC{$f$:$T$,$x$:$T^{\prime}$ $\vdash$ $x:T^{\prime\prime\prime}$}
			\BinaryInfC{$f$:$T$,$x$:$T^{\prime}$ $\vdash$ $f x$ : $T^{\prime\prime}$}		
		
		\end{prooftree} 
		
		
		\end{itemize}
		
		il risultato finale \`e supportato dalle seguenti sostituzioni:
		\begin{itemize}
		
		\item $T \rightarrow T^{\prime} \rightarrow T^{\prime\prime}$
		\item $T^{\prime} \rightarrow T^{\prime\prime}$
		\item $T^{\prime\prime\prime} \rightarrow T^{\prime\prime}$
		\item $T^{\prime}$ =  $T^{\prime\prime\prime}$
		\item $T$ = $T^{\prime\prime\prime\prime}$
		\item $T^{\prime\prime\prime\prime}$ = $T^{\prime\prime\prime}$
		\item $T$ =  $T^{\prime\prime\prime}$
		\item $T^{\prime\prime}$  =  $T^{\prime\prime\prime}$
			
		
		\end{itemize}
				
		Il risultato finale \`e dato dalle sostituzioni precedenti e quindi il tipo \`e:\\ $(T \rightarrow T) \rightarrow T \rightarrow T $

		\vspace{1.5cm}
		
	
\item $fn$ $f:T1.fn$ $g:T2.if$ $(f$ $(g$ $true))$ $then$ $f$ ($fn$ $x:T3.true)$ $else$ $f(fn$ $x:T4.x)$  \\
		
		\scalebox{.75}{
		\parbox{1cm}{\vspace{1.5cm}
		\begin{prooftree}
			
				
			%guardia if
			\AxiomC{A}
			\UnaryInfC{$f$:$T_1$,$g$:$T_2$ $\vdash$ $f$ $(g$ $true))$ : $Bool$}	
			
			%then
			\AxiomC{B}
			\UnaryInfC{$f$:$T_1$,$g$:$T_2$ $\vdash$ $f$ $(fn$ $x$:$T_3.true)$:$W$}
			
			%else
			\AxiomC{C}
			\UnaryInfC{$f$:$T_1$,$g$:$T_2$ $\vdash$ $f(fn$ $x:T_4.x)$:$W$}
			
						
			
			\LeftLabel{\textsc{(IF-THEN-ELSE)}}
			\TrinaryInfC{$f$:$T_1$,$g$:$T_2$ $\vdash$ $if$ $(f$ $(g$ $true))$ $then$ $f$ ($fn$ $x:T_3.true)$ $else$ $f(fn$ $x:T_4.x)$:$W$}
			\LeftLabel{\textsc{(FUN)}}
			\UnaryInfC{$f$:$T_1$ $\vdash$ $fn$ $g:T_2.if$ $(f$ $(g$ $true))$ $then$ $f$ ($fn$ $x:T_3.true)$ $else$ $f(fn$ $x:T_4.x)$:$T_2 \rightarrow W$}
			\LeftLabel{\textsc{(FUN)}}
			\UnaryInfC{$\emptyset\:\vdash$  $fn$ $f:T_1.fn$ $g:T_2.if$ $(f$ $(g$ $true))$ $then$ $f$ ($fn$ $x:T_3.true)$ $else$ $f(fn$ $x:T_4.x)$:$T_1 \rightarrow T_2 \rightarrow W$}
		\end{prooftree}}
		}	
		
		\vspace{1cm}
		Dove gli alberi A, B e C sono:
		

		\begin{itemize}[label=$\star$]
		
		\item 	Albero A (derivazione guardia del'if)\\
		
		\scalebox{.8}{
		\parbox{1cm}{\vspace{1cm}
		\begin{prooftree}
		
			\AxiomC{$f$:$ U \rightarrow Bool $ $\in$ $f$:$T_1$,$g$:$T_2$}
			\LeftLabel{\textsc{(VAR)}}
			\UnaryInfC{$f$:$T_1$,$g$:$T_2$ $\vdash$ $f$:$ U \rightarrow Bool $}
			
			
			\AxiomC{$g$:$ Bool \rightarrow U $ $\in$ $f$:$T_1$,$g$:$T_2$}
			\LeftLabel{\textsc{(VAR)}}
			\UnaryInfC{$f$:$T_1$,$g$:$T_2$ $\vdash$ $g$:$ Bool \rightarrow U $}
			\AxiomC{$\checkmark$}
			\LeftLabel{\textsc{(TRUE)}}
			\UnaryInfC{$f$:$T_1$,$g$:$T_2$ $\vdash$ $true$:$ Bool $}
			\LeftLabel{\textsc{(APP)}}
			\BinaryInfC{$f$:$T_1$,$g$:$T_2$ $\vdash$ $g$ $true$:$U$}
			
			\LeftLabel{\textsc{(APP)}}
			\BinaryInfC{$f$:$T_1$,$g$:$T_2$ $\vdash$ $f$ $(g$ $true))$ : $Bool$}
		\end{prooftree}}	
		}	\vspace{1cm}
		
		
		\item 	Albero B (derivazione ramo then)\\
		
		\scalebox{.85}{
		\parbox{1cm}{\vspace{1cm}
		\begin{prooftree}
		
			\AxiomC{$f: Z \rightarrow W$ $\in$ $f$:$T_1$,$g$:$T_2$}
			\LeftLabel{\textsc{(VAR)}}	
			\UnaryInfC{$f$:$T_1$,$g$:$T_2$ $\vdash$ $f$ $Z \rightarrow W $}
			
			\AxiomC{$\checkmark$}
			\LeftLabel{\textsc{(TRUE)}}
			\UnaryInfC{$f$:$T_1$,$g$:$T_2$,$x$:$T_3$ $\vdash$ $true$ : $Bool$}
			\LeftLabel{\textsc{(FUN)}}
			\UnaryInfC{$f$:$T_1$,$g$:$T_2$ $\vdash$ $fn$ $x$:$T_3.true$ : $T_3 \rightarrow Bool$}
			
			\LeftLabel{\textsc{(APP)}}
			\BinaryInfC{$f$:$T_1$,$g$:$T_2$ $\vdash$ $f$ $(fn$ $x$:$T_3.true)$ :$W$}
		\end{prooftree}}	
		}\vspace{1cm}			
		
		
		
		\item 	Albero C (derivazione ramo else)\\
		
		\scalebox{.85}{
		\parbox{1cm}{\vspace{1cm}
		\begin{prooftree}
		
			\AxiomC{$f: Z_1 \rightarrow W$ $\in$ $f$:$T_1$,$g$:$T_2$}
			\LeftLabel{\textsc{(VAR)}}
			\UnaryInfC{$f$:$T_1$,$g$:$T_2$ $\vdash$ $f$ $Z_1 \rightarrow W $}		
		
			\AxiomC{$x$:$T_4$ $\in$ $\Gamma$}
			\LeftLabel{\textsc{(VAR)}}
			\UnaryInfC{$f$:$T_1$,$g$:$T_2$,$x$:$T_4$ $\vdash$ $x$ : $T_4$}
			\LeftLabel{\textsc{(FUN)}}
			\UnaryInfC{$f$:$T_1$,$g$:$T_2$ $\vdash$ $fn$ $x$:$T_4.x$ : $T_4 \rightarrow T_4$}
			
			
			\LeftLabel{\textsc{(APP)}}
			\BinaryInfC{$f$:$T_1$,$g$:$T_2$ $\vdash$ $f(fn$ $x:T_4.x)$:$W$}
		\end{prooftree}}	
		}\vspace{1cm}					
		
		
		\end{itemize}				
		
		Il risultato finale \`e supportato dalle seguenti sostituzioni:
		\begin{itemize}
		
		\item $T_1$ = $U \rightarrow Bool$
		\item $T_1$ = $Z \rightarrow Bool$
		\item $T_1$ = $Z_1 \rightarrow Bool$
		\item $T_2$ = $Bool \rightarrow U$
		\item $W$ = $Bool$
		\item $U$ = $Bool \rightarrow Bool$
		\item $T_3$ = $T_4$ = $Bool$
		
		\item quindi $T_2$ = $Bool \rightarrow (Bool \rightarrow Boool)$ ( per sostituzione infatti $f$ $(fn$ $x$:$T_3.true)$ \`e $U \rightarrow W$ con $W$ = $Bool$ e $U$ = $(Bool \rightarrow Bool))$
		\item quindi $T_1$ =  $(Bool \rightarrow Bool) \rightarrow Boool$
		\item quindi da $T_1 \rightarrow T_2 \rightarrow W$ otteniamo $(Bool \rightarrow Bool) \rightarrow Boool$ $\rightarrow$ $(Bool \rightarrow (Bool \rightarrow Boool))$ $\rightarrow$ $Bool$
		
		
		
		\end{itemize}
		e quindi il tipo del termine \`e quindi:\\
	 % $(Bool \rightarrow Bool) \rightarrow Boool$ $\rightarrow$ $(Bool \rightarrow (Bool \rightarrow Boool))$ $\rightarrow$ $Bool$\\
		$((Bool$ $\rightarrow$ $Bool)$ $\rightarrow$ $Bool)$ $\rightarrow$ $(Bool$ $\rightarrow$ $(Bool$ $\rightarrow$ $Bool))$ $\rightarrow$ $Bool$ 
		

\end{enumerate}