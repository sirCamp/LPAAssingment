 
 
\subsubsection*{Esercizio 3.1}
Provare che ogni sottotermine di un termine ben tipato e ben tipato.
\subparagraph{Svolgimento}
Per la risoluzione di questo esericizio e necessario utilizzare il metodo dell'induzione sui passi di derivazione, ovvero sull'altezza dell'albero di derivazione.

\begin{description}

 \item[Caso $0$] A questo livello il termine e congruo con il sottotermine di conseguenza se il termine e bentipato lo e anche il sottotermine

 \item[Caso $n+1$ ] A questo livello il sottotermine e dato dai sottotermini dei passi $0$...$n$ e di conseguenza se questi sottotermini sono ben tipati allora lo sara anche questo.
 Per induzione infatti sappiamo che se applico:
 \begin{description}
 \item[SUM] fdfdsf
 \end{description}

\end{description}
 
\subsubsection*{Esercizio 3.2} 
Dimostrare subject reduction per induzione sulla derivazione di $\Gamma\:\vdash$ M : T.

\subsubsection*{Esercizio 3.4}

Vale l'opposto di subject reduction, i.e. se $\Gamma\:\vdash$ $M'\: :  $T e $M\:\rightarrow\:$M', allora $\Gamma\:\vdash$ $M$ : $T$ (detto subject expansion)? Dimostrarlo oppure dare un controesempio.

\subsubsection*{Esercizio 3.5}
Se al posto della regola $(APP)\:$ si definisse la regola seguente:

\begin{prooftree} 
	\AxiomC{$\Gamma\:\vdash$ $M\: :  $T$\:\rightarrow\:$T}
	\AxiomC{$\Gamma\:\vdash$ $N\ :$T}
	
	\LeftLabel{\textsc{(APP')}}
	\BinaryInfC{$\Gamma\:\vdash$ $M$ $N$ : $T$  }
\end{prooftree} 

sarebbe ancora vero il teorema di safety?

\subsubsection*{Esercizio 3.6}
Se al posto delle regole $(APP)\:$ e $(FUN)\:$ si definissero le regole seguenti regole:
\begin{prooftree} 
	\AxiomC{$\Gamma\:\vdash$ $M\: :  $T$\:\rightarrow\:$T}
	\AxiomC{$\Gamma\:\vdash$ $N\ :$T}
	
	\LeftLabel{\textsc{(APP')}}
	\BinaryInfC{$\Gamma\:\vdash$ $M$ $N$ : $T$  }
\end{prooftree} 

e


\begin{prooftree} 
	\AxiomC{$\Gamma\:$, x : $T1\vdash$ $M\: : $T}
	\AxiomC{}
	\LeftLabel{\textsc{(FUN')}}
	\BinaryInfC{$\Gamma\:\vdash$ $fn$ x : $T1.M:\rightarrow\:$T  } 
\end{prooftree} 

sarebbe ancora vero il teorema di safety?

\subsubsection*{Esercizio 3.6}
Se si aggiungessero al sistema di tipi i seguenti due assiomi

\begin{prooftree} 
	\AxiomC{}
	\AxiomC{}
	\LeftLabel{\textsc{(TRUE')}}
	\BinaryInfC{$\Gamma\:\vdash$ true : $Nat$  } 
\end{prooftree}

e

\begin{prooftree} 
	\AxiomC{}
	\AxiomC{}
	\LeftLabel{\textsc{(FALSE')}}
	\BinaryInfC{$\Gamma\:\vdash$ false : $Nat$  } 
\end{prooftree}

sarebbe ancora vero il teorema di safety?
\subsubsection*{Esercizio 3.7}

Dimostrare il seguente fatto: se $\Gamma\:\vdash$ $M\: :  $T e derivabile allora  \textit{fv($M$)}  $\subseteq$ \textit{Dom($\Gamma$)}
\subsubsection*{Esercizio 3.8}
Ricostruire il tipo dei seguenti termini:
\begin{itemize}
\item $fn$ $x$:$T1.fn$ $y$:$T2.if$ $y$ then $x$ else true
\item $fn$ $x:Nat:\rightarrow\:$Bool.x
\item $fn$ $f:T.fn$ $x:T'.f$(if true  then  $x$  else  $f$ $x$)
\item $fn$ $f:T1.fn$ $g:T2.$if ($f$ ($g$ true))then $f$ ($fn$ $x:$T3.true) else $f$($fn$ $x:T4.x$)
\end{itemize}
