\section{I Tipi Semplici (note 45)}
 
\subsubsection*{Esercizio 3.2}
Provare che ogni sottotermine di un termine ben tipato e' ben tipato.
\subparagraph{Svolgimento}
Per la risoluzione di questo esericizio e' necessario utilizzare il metodo dell'induzione sui passi di derivazione, ovvero sull'altezza dell'albero di derivazione.

\begin{description}

 \item[Caso Base $h=1$] A questo livello il termine e il sottotermine coincidono. Quindi, il fatto che ogni sottotermine risulta essere ben tipato e' un assioma.

 \item[Caso Induttivo $h$ ] Questo caso e' composto dall'applicazione della regola e dai vari sottotermini che la riguardano. Quindi premesso che questa applicazione sia valida per $h = 0$ e che l'ipotesi sia vera anche per h allora per induzione vale anche su $h + 1$ e suoi sottotermini contenuti al passo $h$.
 
 \end{description}
 
\subsubsection*{Esercizio 3.12} 
Dimostrare subject reduction per induzione sulla derivazione di $\Gamma\:\vdash$ M : T.

\subparagraph*{Svolgimento}

Si vuole quindi dimostrare che:

\begin{center}
	Se $\Gamma \vdash{} M : T$ e $M \to{} M'$, allora $\Gamma \vdash{} M' : T$.
\end{center}
%La dimostrazione per induzione va fatta sul numero di passi che $M$ portano a $M'$.
%Dobbiamo ragionare su quella che e' la lunghezza dei cammini della derivazione:

%Caso Base l = 0:\\
%Otteniamo che $M$ = $M'$ e la tesi e' trivialmente vera.

%Caso Induttivo l+1:\\
%In questo passo ho che $M \rightarrow^{*}_l M_1 \rightarrow M'$, e per ipotesi induttiva ho che $\Gamma \vdash M_1 : T$.\\
%A questo punto ho che $M_1 \rightarrow M'$ e quindi posso applicare subject-reduction per ottenere che $\Gamma \vdash M' : T$.

%Ovviamente grazie ai teoremi di Sostituzione, Preservazione e Progressione so per certo che qualsiasi sia il contesto %il risultato non cambia

\begin{description}


 \item[TRUE] $\Gamma \vdash{} M : T$ \'e $\Gamma \vdash{} \true{} :
  \Bool$.
  Il teorema \'e vacuamente vero per questa derivazione siccome
  $\not \exists{M'}. M \to{} M'$.
  
 \item[False]
  $\Gamma \vdash{} M : T$ \'e $\Gamma \vdash{} \false{} : \Bool$.
  Analogo a True.

\item[Nat]
  $\Gamma \vdash{} M : T$ \'e $\Gamma \vdash{} n : \Nat$. Analogo a True.

\item[Var] $\Gamma \vdash{} M : T$ \'e $\Gamma \vdash{} x : T$.
  Analogo a True.

\item[Fun] $\Gamma \vdash{} M : T$ \'e
  $\Gamma \vdash{} fn x\:T_1 C : T_1 \to{} T_2$. Analogo a True.
  
  \item[Sum]
  $\Gamma \vdash{} M : T$ \'e $\Gamma \vdash{} A + B : \Nat$.
 
  Per quanto assunto dalla regola di tipo \emph{Sum}, sappiamo per certo che
  anche $A$ e $B$ hanno tipo $\Nat$ sotto contesto $\Gamma$.

  Supponiamo che $A + B \to{} M'$ (sappiamo che $M'$ \'e unico poich\'e l'esecuzione
   \'e  deterministica).Tale riduzione pu\'o essere:

  \begin{itemize}
    \item \emph{Sum}: $M' \equiv{} n$. Per la regola di tipo
      \emph{Nat}, si ha che $T \equiv{} \Nat$ e che
      $\Gamma{} \vdash{} M' : \Nat$;
    \item \emph{Sum-Left}: $M' \equiv{} A' + B$. 
    Per gli assunti della regola $\emph{Sum}$ ho che $\Gamma \vdash A : \Nat$ e per $\emph{Sum-Left}$ $A \to A'$.
    Posso quindi applicare l'ipotesi induttiva\footnote{perch \'e  la derivazione $\Gamma \vdash A :\Nat$ richiede meno passaggi}
    per ottenere il giudizio $ \Gamma \vdash A' : \Nat$.
    Sono quindi soddisfatte le premesse della regola $\emph{Sum}$ per $M'$ e quindi  $\Gamma{} \vdash{} M' : \Nat \equiv T$;

    \item \emph{Sum-Right}: $M' \equiv{} v + B'$. Il ragionamento  \'e  analogo al caso $\emph{Sum-Left}$.
  \end{itemize}

\item[Minus]
  $\Gamma \vdash{} M : T$ \'e $\Gamma \vdash{} A - B : \Nat$. Analogo a
  \emph{Sum}.

\item[IfThenElse] $\Gamma \vdash{} M : T$ \'e 
  $\Gamma \vdash{} \IF{M_1}{M_2}{M_3}$.
  
  Dal giudizio di tipo abbiamo che:

  \begin{itemize}
  	\item $\Gamma \vdash M_1 : \Bool$;
    \item $\Gamma \vdash{} M_2 : T$;
    \item $\Gamma \vdash{} M_3 : T$.
  \end{itemize}

  Supponiamo che $\IF{M_1}{M_2}{M_3} \to{} M'$ (sappiamo che $M'$ \'e unico poich\'e l'esecuzione
  \'e deterministica). Tale riduzione pu\'o essere:

  \begin{itemize}
    \item \myrule{If}: $M' \equiv{} \Gamma \vdash{} \mbox{if } M'_1
      \mbox{ then } M_2 \mbox{ else } M_3$. Dato che il giudizio $\Gamma \vdash M_1 : \Bool$ richiede una derivazione in meno e che $M_1 \to M_1'$, possiamo applicare l'ipotesi induttiva per ottenere il giudizio $\Gamma \vdash M_1' : \Bool$. Si ha quindi che la regola di tipo \myrule{IfThenElse} continua a valere per il termine $M'$ e dato che i termini $M_2$ e $M_3$ non sono cambiati, anche $M'$ ha tipo $T$;
    \item \myrule{If-True}: $M' \equiv{} M_2$. $M_2$ ha tipo $T$ e quindi anche $M'$ ha tipo $T$;
    \item \myrule{If-False}: $M' \equiv{} M_3$.  Analogo a \myrule{If-True}
  \end{itemize}
  
  
\item[App] $\Gamma \vdash{} M : T$ \'e  $\Gamma \vdash{} A \: B : T_2$.
  Dal giudizio di tipo abbiamo che:

  \begin{itemize}
    \item $\Gamma \vdash{} A : T_1 \to{} T_2$;
    \item $\Gamma \vdash{} B : T_1$.
  \end{itemize}

  Supponiamo che $A \: B \to{} M'$ (sappiamo che $M'$ \'e  unico poich\'e  l'esecuzione
  \'e  deterministica). Tale riduzione pu\'o essere:

  \begin{itemize}
    \item \myrule{Beta}: $M' \equiv{} \Gamma \vdash{}
      Z[x \coloneqq{} B]$\footnote{dato $A \equiv{} FN{x}{T_1}Z$}.
      Per la regola di tipo $\myrule{Fun}$ ho che $Z : T_2$ e quindi, per
       il Substitution Lemma, $\Gamma \vdash{} Z[x \coloneqq{} B] : T_2$. Per
      questo motivo si ha che $\Gamma \vdash{} M' : T_2$ con $T_2 \equiv T$;
    \item \myrule{App-1}: $M' \equiv{} A'\: B$. Dato che il giudizio $\Gamma \vdash A : T_1 \to T_2$ richiede una derivazione con meno passaggi e che $A \to A'$ posso applicare l'ipotesi induttiva per ottenere il giudizio $A' : T_1 \to T_2$. Il termine $M'$ soddisfa le premesse della regola di tipo \myrule{App} e quindi $M'$ ha tipo $T_2 \equiv T$.

    \item \myrule{App-2}: $M' \equiv{} v\: B'$. Analogo ad \myrule{App-1}.
  \end{itemize}

 
\end{description}




\subsubsection*{Esercizio 3.13}

Vale l'opposto di subject reduction, i.e. se $\Gamma\:\vdash$ $M'\: :  $T e $M\:\rightarrow\:$M', allora $\Gamma\:\vdash$ $M$ : $T$ (detto subject expansion)? Dimostrarlo oppure dare un controesempio.
\subparagraph*{Svolgimento}
La risposta e' no:

E sufficiente pensare a $M \rightarrow M_1$ con $\Gamma\vdash$ $M_1$ : $\sigma$ e che 
$\Gamma\nvdash$ $M$: $\sigma$\\
Quanto ripostato sopra pu\`o essere semplificato tramite questo esempio:\\
Supponimao di avere $M_1$ $\equiv$ $A$ : $T$ e $M$ $\equiv$ $if true then A else B$, ovvero il caso in cui $M$ $\rightarrow$ $M_1$ per la regola (If-True).\\
In questo caso la derivazione di $\Gamma$ $\vdash$ $M$ : $T$ ri riesce ad ottenere usando la regaola $IfThenElse$, tuttavia questo impone che nel contesto ci siano questi giudizi:
\begin{itemize}
\item $\Gamma$ $\vdash$ $true$:$Bool$ che \`e un assioma
\item $\Gamma$ $\vdash$ $A$:$T$ soddisfatto per l'ipotesi 
\item  $\Gamma$ $\vdash$ $B$:$T$ putroppo questa informazione non si riesce a repire dall'ipotesi ne dal contesto e pertatnto non c'\`e garanzia sull'uguaglianza di tipi di $A$ e $B$\\
Pertanto possiamo creare un termine $N$ come $if$ $true$ $then$ $4$ $else$ $false$, da cui si riesce ad attenere $M'$ $\equiv$ $A$ $\equiv$ $4$. Questo termine nonostante esegua risulta essere non non ben-tipato.


\end{itemize}







\subsubsection*{Esercizio 3.14}
Se al posto della regola $(APP)\:$ si definisce la regola seguente:

\begin{prooftree} 
	\AxiomC{$\Gamma\:\vdash$ $M\: :  $T$\:\rightarrow\:$T}
	\AxiomC{$\Gamma\:\vdash$ $N\ :$T}
	
	\LeftLabel{\textsc{(APP')}}
	\BinaryInfC{$\Gamma\:\vdash$ $M$ $N$ : $T$  }
\end{prooftree} 

sarebbe ancora vero il teorema di safety?
\subparagraph*{Svolgimento}
Si, lo si puo' dimostrare considerando che APP' non e' altro che una specificazione di APP: se poniamo X uguale all'insieme dei tipi di APP, Y e' un sottoinsieme di X corrispondente ai possibili tipi di APP'.

Di conseguenza il teorema di safety vale anche per APP'

\subsubsection*{Esercizio 3.15}
Se al posto delle regole $(APP)\:$ e $(FUN)\:$ si definissero le seguenti regole:
\begin{prooftree} 
	\AxiomC{$\Gamma\:\vdash$ $M\: :  $T$\:\rightarrow\:$T}
	\AxiomC{$\Gamma\:\vdash$ $N\ :$T}
	
	\LeftLabel{\textsc{(APP')}}
	\BinaryInfC{$\Gamma\:\vdash$ $M$ $N$ : $T$  }
\end{prooftree} 

e


\begin{prooftree} 
	\AxiomC{$\Gamma\:$, x : $T1\vdash$ $M\: : $T}
	\AxiomC{}
	\LeftLabel{\textsc{(FUN')}}
	\BinaryInfC{$\Gamma\:\vdash$ $fn$ x : $T1.M:T\rightarrow\:$T  } 
\end{prooftree} 

sarebbe ancora vero il teorema di safety?

\subparagraph*{Svolgimento}

La risposta e' no:\\
In questo caso la $FUN_1$ non fa alcun controllo sul tipo dell'argomento e quindi esiste la possibilita' di arrivare ad un passo della derivazione che produce un termine STUCK, invalidando il teorema stesso.\\
Supponiamo di avere il termine:\\
$M$ $=$ $(fn$ $x$:$Bool.if$ $x$ $then$ $1$ $else$ $0)$ $(1)$ 

e creiamo il suo albero di derivazione:




\begin{prooftree} 
	\AxiomC{$ \checkmark $}
	\UnaryInfC{$x$:$Bool$ $\in$  $\Gamma$ }
	\RightLabel{\textsc{($Var$)}}
	\UnaryInfC{$\Gamma$ $\vdash$ $x$:$Bool$ }
	\AxiomC{$ \checkmark $}
	\LeftLabel{\textsc{($Nat$)}}
	\UnaryInfC{$\Gamma$ $\vdash$ $1$:$Nat$}
	\AxiomC{$ \checkmark $}
	\LeftLabel{\textsc{($Nat$)}}
	\UnaryInfC{$\Gamma$ $\vdash$ $0$:$Nat$}
	\LeftLabel{\textsc{($IfThenElse$)}}
	\TrinaryInfC{$x$:$Bool$ $\vdash$ $if$ $x$ $then$ $1$ $else$ $0$ :$Nat$}
	\LeftLabel{\textsc{($FUN_1$)}}
	\UnaryInfC{$\empty \vdash fn$ $x$:$Bool.if$ $x$ $then$ $1$ $else$ $0$ :$Nat \rightarrow$ :$Nat$}
	\AxiomC{$ \checkmark $}
	\LeftLabel{\textsc{($Nat$)}}
	\UnaryInfC{$\empty \vdash 1$:$Nat$}
	\LeftLabel{\textsc{($APP_1$)}}
	\BinaryInfC{$\empty \vdash (fn$ $x$:$Bool.if$ $x$ $then$ $1$ $else$ $0)$ $(1)$  } 
\end{prooftree}


In questo caso possiamo osservare che il termine $M$ risulta essere ben-tipato secondo le nuove regole di tipo.\\
Tuttavia, \`e evidente che l'invocazione della funzione in questione porta $M$ ad un termine stuck, questo perch\`e il sottotermine $N$ $=$ $if$ $x$ $then$ $1$ $else$ $0$ si aspetta un $Bool$, infatti le regole della semantica operazionale che gestiscono l'$if$, permettono di avanzare nella computazione.\\
Nel caso in cui la guardia non sia un valore, la computazione procede con la valutazione della guardia, altrimenti procede il ramo $TRUE$ o con il ramo $FALSE$.\\
In questo caso, poich\`e il valore passato alla funzione \`e un intero, e quindi gi\`a un valore le uniche regole applicabili sarebbero $(IF-TRUE)$ e $(IF-FALSE)$ ma poich\`e la guardia \'e 1 non sono applicabili e quindi si otterrebbe un termine STUCK.


\subsubsection*{Esercizio 3.16}
Se si aggiungessero al sistema di tipi i seguenti due assiomi

\begin{prooftree} 
	\AxiomC{}
	\AxiomC{}
	\LeftLabel{\textsc{(TRUE')}}
	\BinaryInfC{$\Gamma\:\vdash$ true : $Nat$  } 
\end{prooftree}

e

\begin{prooftree} 
	\AxiomC{}
	\AxiomC{}
	\LeftLabel{\textsc{(FALSE')}}
	\BinaryInfC{$\Gamma\:\vdash$ false : $Nat$  } 
\end{prooftree}

sarebbe ancora vero il teorema di safety?
\subparagraph*{Svolgimento}
La risposta e' no:\\
Data la definizione di Teorema di Safety, e' evidente che viene meno la stessa ipotesi ovvero che M sia un valore o un che non evolva in un termine STUCK.

Con queste regole di tipo, infatti, sarebbe permessa anche l'operazione SUM che genererebbe un termine STUCK, invalidando il teorema stesso.\
Supponiamo di avere il termine: $M$ $=$ $true+4$


\begin{prooftree} 
	\AxiomC{$ \checkmark $}
	\LeftLabel{\textsc{($Nat$)}}
	\UnaryInfC{$\empty \vdash 4$:$Nat$}
	\AxiomC{$ \checkmark $}
	\LeftLabel{\textsc{($Nat$)}}
	\UnaryInfC{$\empty \vdash true$:$Nat$}
	\LeftLabel{\textsc{($SUM$)}}
	\BinaryInfC{$\empty \vdash$ $true$ $+$ $4$  } 
\end{prooftree}

Anche in questo caso, come nel precendete, notiamo che le regole di tipo permettono la derivazione dell'albero.\\
Tuttavia, se applicassimo le regole della semantica operazionale non potremo avanzare perch\`e entrambi gli addendi sono gi\'a dei valori e quindi non posso applicare $SUM-LEFT$ o $SUM-RIGHT$ ma non posso neppure applicare $SUM$ perch\'e genero un termine STUCK in quanto $true$ non pu\'o essere sommato a $4$.

\subsubsection*{Esercizio 3.17}

Dimostrare il seguente fatto: se $\Gamma\:\vdash$ $M\: :  $T e derivabile allora  \textit{fv($M$)}  $\subseteq$ \textit{Dom($\Gamma$)}
\subparagraph*{Svolgimento}

Si procede con la dimostrazione per induzione:

\begin{description}

\item[Caso Base]

\item[TRUE] $\Gamma \vdash{} M : T$ \'e $\Gamma \vdash{} \true{} :
  \Bool$.
  Il teorema \'e vacuamente vero per questa derivazione siccome
  $\not \exists{M'}. M \to{} M'$.
  
 \item[False]
  $\Gamma \vdash{} M : T$ \'e $\Gamma \vdash{} \false{} : \Bool$.
  Analogo a True.

\item[Nat]
  $\Gamma \vdash{} M : T$ \'e $\Gamma \vdash{} n : \Nat$. Analogo a True.

\item[Var] $\Gamma \vdash{} M : T$ \'e $\Gamma \vdash{} x : T$.
  Analogo a True.
  
 In questo passo il fatto e' un assioma, questo perche' non esistono variabili libere quindi $fv(M)$ e' $\emptyset$ e il vuoto e un sottoinsieme del $Dom(\Gamma)$

\item[Caso Induttivo]

Al passo h grazie al lemma di inversione e al caso base sappiamo per ipotesi che il fatto e' vero e quindi per induzione lo e' anche al passo h +1 

\end{description}

\subsubsection*{Esercizio 3.18}
Ricostruire il tipo dei seguenti termini:
\begin{itemize}
\item $fn$ $x$:$T1.fn$ $y$:$T2.if$ $y$ then $x$ else true
\item $fn$ $x:Nat:\rightarrow\:$Bool.x
\item $fn$ $f:T.fn$ $x:T'.f$( if true  then  $x$  else  $f$ $x$)
\item $fn$ $f:T1.fn$ $g:T2.$ if ($f$ ($g$ true)) then $f$ ($fn$ $x:$T3.true) else $f$($fn$ $x:T4.x$)
\end{itemize}

\subparagraph*{Svolgimento}
\begin{enumerate}[label=\alph*), leftmargin=*]


\item $fn$ $x$:$T1.fn$ $y$:$T2.if$ $y$ then $x$ else true  \\
		\scalebox{.75}{
		\parbox{1cm}{
		\begin{prooftree}
			\AxiomC{y:Bool $\in$ $\Gamma$}
			\LeftLabel{\textsc{(VAR)}} 
			\LeftLabel{\textsc{(VAR)}}
			\UnaryInfC{$\Gamma$,$x$:$T1$, $\vdash$ $y$ : $Bool$}
			\LeftLabel{\textsc{(VAR)}}
			\AxiomC{x:B$x$:$Nat:\rightarrow Bool$ $\in$ $\Gamma$}
			\LeftLabel{\textsc{(VAR)}}
			\UnaryInfC{$\Gamma$,$x$:$T1$, $\vdash$ $x$ : $Bool$}
			\LeftLabel{\textsc{(VAR)}}
			\AxiomC{$\checkmark$}
			\LeftLabel{\textsc{(TRUE)}}
			\UnaryInfC{$\Gamma$,$x$:$T1$, $\vdash$ $true$ : $Bool$}
			\LeftLabel{\textsc{(IF-THEN-ELSE)}}
			\TrinaryInfC{$x$:$T1$,$y$:$T2$ $\vdash$ $if$ $y$ then $x$ else true}
			\LeftLabel{\textsc{(FUN)}}
			\UnaryInfC{$x$:$T1$ $\vdash$  $fn$ $y$:$T2.if$ $y$ then $x$ else true}
			\LeftLabel{\textsc{(FUN)}}
			\UnaryInfC{$\emptyset\:\vdash$  $fn$ $x$:$T1.fn$ $y$:$T2.if$ $y$ then $x$ else true}
		\end{prooftree}}
		}
		
		Il risultato finale e' $Bool \rightarrow (Bool \rightarrow Bool)$
		
\item $fn$ $x:Nat:\rightarrow\:$Bool.x  \\
		\scalebox{.75}{
		\parbox{1cm}{
		\begin{prooftree}
		
			\AxiomC{$x$:$Nat:\rightarrow Bool$ $\in$ $\Gamma$}
			\LeftLabel{\textsc{(VAR)}}
			\UnaryInfC{$x$:$Nat:\rightarrow Bool$ $\vdash$  $x$:$T$}
			\LeftLabel{\textsc{(FUN)}}
			\UnaryInfC{$\emptyset\:\vdash$  $fn$ $x:Nat:\rightarrow\:$Bool.x}
		\end{prooftree}}
		}
		
		Il risultato finale e' $(Nat \rightarrow Bool) \rightarrow (Nat \rightarrow Bool)$
		
		
\item $fn$ $f:T.fn$ $x:T'.f$(if true  then  $x$  else  $f$ $x$)  \\
		\scalebox{.75}{
		\parbox{1cm}{
		\begin{prooftree}
			\AxiomC{$f$:$T$,$x$:$T_1$ $\vdash$ $f$:$T_3 \rightarrow T_2$}
			
			%%else
			
			\AxiomC{$f$:$T$,$x$:$T_1$ $\vdash$ $f:T_4 \rightarrow T$}
			\AxiomC{$T_1 = T_4$}
			\LeftLabel{\textsc{(VAR)}}
			\UnaryInfC{$f$:$T$,$x$:$T_1$ $\vdash$ $x:T_4$}
			\LeftLabel{\textsc{(APP)}}
			\BinaryInfC{$f$:$T$,$x$:$T_1$ $\vdash$ $f x$ : $T_3$}
			
			%%			
					
			%%then
			
			\AxiomC{$T_1 = T_3$}
			\LeftLabel{\textsc{ }}
			\UnaryInfC{$f$:$T$,$x$:$T_1$ $\vdash$ $x$ : $T_3$}
						
			%%			
			%% if
			\AxiomC{$\checkmark$}
			\LeftLabel{\textsc{(TRUE)}}
			\UnaryInfC{$f$:$T$,$x$:$T_1$ $\vdash$ $true$ : $Bool$}
			%%
			\LeftLabel{\textsc{(IF-THEN-ELSE)}}
			\TrinaryInfC{$f$:$T$,$x$:$T_1$ $\vdash$ $ if true$ then $x$ else $f$ $x$ : $T_3$}
			
			\LeftLabel{\textsc{(APP)}}
			\BinaryInfC{$f$:$T$,$x$:$T_1$ $\vdash$ $f$ $(if true$ then $x$ else $f$ $x$)}
			\LeftLabel{\textsc{(FUN)}}
			\UnaryInfC{$f$:$T$ $\vdash$  $fn$ $x$:$T_1.f$ $(if true$ then $x$ else $f$ $x$)}
			\LeftLabel{\textsc{(FUN)}}
			\UnaryInfC{$\emptyset\:\vdash$  $fn$ $f:T.fn$ $x:T_1.f$(if true  then  $x$  else  $f$ $x$)}
		\end{prooftree}}
		}
		
		Il risultato finale e' dato dal fatto che $T = T_1 = T_" = T_3 = T_4$ quindi $(T \rightarrow T) \rightarrow (T \rightarrow T \rightarrow T)$
		
	
\item $fn$ $f:T1.fn$ $g:T2.$if ($f$ ($g$ true)) then $f$ ($fn$ $x:$T3.true) else $f$($fn$ $x:T4.x$)  \\
		\scalebox{.35}{
		\parbox{1cm}{
		\begin{prooftree}
			
			%%else
			\AxiomC{$x$:$T_4$ $\in$ $\Gamma$}
			\LeftLabel{\textsc{(VAR)}}
			\UnaryInfC{$f$:$T_1$,$g$:$T_2$,$x$:$T_4$ $\vdash$ x : $T_4$}
			\LeftLabel{\textsc{(FUN)}}
			\UnaryInfC{$f$:$T_1$,$g$:$T_2$ $\vdash$ $fn$ $x$:$T_4$.x : $T_4 \rightarrow T_4$}
			\AxiomC{$f$:$T_1$,$g$:$T_2$ $\vdash$ $f$ $T_5 \rightarrow T_1 $}
			\LeftLabel{\textsc{(APP)}}
			\BinaryInfC{$f$:$T_1$,$g$:$T_2$ $\vdash$ $f$($fn$ $x:T_4.x$)}
			
			%%			
					
			%%then
			\AxiomC{$\checkmark$}
			\LeftLabel{\textsc{(TRUE)}}
			\UnaryInfC{$f$:$T_1$,$g$:$T_2$,$x$:$T_3$ $\vdash$ true : $Bool$}
			\LeftLabel{\textsc{(FUN)}}
			\UnaryInfC{$f$:$T_1$,$g$:$T_2$ $\vdash$ $fn$ $x$:$T_3$.true : $T_3 \rightarrow Bool$}
			\AxiomC{$f$:$T_1$,$g$:$T_2$ $\vdash$ $f$ $T_5 \rightarrow T_1 $}
			\LeftLabel{\textsc{(APP)}}
			\BinaryInfC{$f$:$T_1$,$g$:$T_2$ $\vdash$ $f$ ($fn$ $x$:$T_3$.true)}
						
			%%			
			%% if
			
			\AxiomC{$f$:$T_1$,$g$:$T_2$ $\vdash$ $g$:$ Bool \rightarrow T_2 $}
			\AxiomC{$\checkmark$}
			\LeftLabel{\textsc{(TRUE)}}
			\UnaryInfC{$f$:$T_1$,$g$:$T_2$ $\vdash$ $true$:$ Bool $}
			\LeftLabel{\textsc{(APP)}}
			\BinaryInfC{$f$:$T_1$,$g$:$T_2$ $\vdash$ $g true$:$ T_2$}
			\AxiomC{$f$:$T_1$,$g$:$T_2$ $\vdash$ $f$:$ T_2 \rightarrow Bool $}
			\LeftLabel{\textsc{(APP)}}
			\BinaryInfC{$f$:$T_1$,$g$:$T_2$ $\vdash$ $f$ ($g$ true)) : $Bool$}
			%%
						
			\LeftLabel{\textsc{(IF-THEN-ELSE)}}
			\TrinaryInfC{$f$:$T_1$,$g$:$T_2$ $\vdash$ $if$ ($f$ ($g$ true)) then $f$ ($fn$ $x$:$T_3$.true) else $f$($fn$ $x:T_4.x$)}
			\LeftLabel{\textsc{(FUN)}}
			\UnaryInfC{$f$:$T_1$ $\vdash$ $fn$ $g:T_2.$if ($f$ ($g$ true)) then $f$ ($fn$ $x$:$T_3$.true) else $f$($fn$ $x:T_4.x$)}
			\LeftLabel{\textsc{(FUN)}}
			\UnaryInfC{$\emptyset\:\vdash$  $fn$ $f:T_1.fn$ $g:T_2.$if ($f$ ($g$ true)) then $f$ ($fn$ $x$:$T_3$.true) else $f$($fn$ $x:T_4.x$)}
		\end{prooftree}}
		}	
		
		Il risultato finale e' dato dal fatto che:
		\begin{itemize}
		
	
		\item $T_4 = Bool$
		\item $T_1 = T_2 \rightarrow Bool$
		\item $T_5 = T_3 \rightarrow Bool$
		\item $T_4 = Bool$
		\item $T_3 = T_4 = Bool$
		\item $T_5 = Bool \rightarrow Bool$
		\item $T_2 = (Bool \rightarrow Bool)$
		\item $T_1 = (Bool\rightarrow Bool) \rightarrow Bool$
		
		\end{itemize}
		e quindi il tipo del termine e' quindi $(Bool\rightarrow Bool) \rightarrow Bool$
		

\end{enumerate}