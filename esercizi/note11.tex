\section{Featherweight Java (note 11)}
\subsection*{Esercizio 8.1} 
Si noti che una class table puo' contenere definizioni di classi mutuamente ricorsive. Scrivere un esempio.

\textbf{{\color{red} DA FARE}}

\subsection*{Esercizio 8.2} 
Descrivere la semantica operazionale dei seguenti termini:
\begin{itemize}
	\item new Pair(new A(), new B()).snd
	\item (Pair)new Pair(new A(), new B())
	\item new Pair(new A(), new B()).setfst(new B())
	\item ((Pair) (new Pair(new Pair(new A(), new B()), new A()).fst)).snd
	\item  (B) ((A)new C())
\end{itemize} 

\textbf{{\color{red} DA FARE}}

\subsection*{Esercizio 8.3} 
Scrivere un programma con override di un metodo e descriverne la valutazione, evidenziando il binding dinamico per la chiamata del metodo riscritto.

\textbf{{\color{red} DA FARE}}

\subsection*{Esercizio 8.4} 
Perche' c'e' una regola di tipo sia per l'upcast che per il downcast, mentre c'e' la sola regola di valutazione per upcast, nella semantica operazionale?
\textbf{{\color{red} DA FARE}}

\subsection*{Esercizio 8.5} 
Ha senso aggiungere a FJ la regola di subtyping per i tipi freccia A $\rightarrow$ B?

\textbf{{\color{red} DA FARE}}

\subsection*{Esercizio 8.12} 
Aggiungere a FJ il termine ClassCastException: come cambia la semantica operazionale? Le regole di tipo? Il teorema di Safety e i teoremi di Preservazione e Progressione? Aggiungere in seguito anche la possibilita' di gestire le eccezioni.

\textbf{{\color{red} DA FARE}}