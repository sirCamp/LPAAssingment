\section{Subtyping (note 9)}
\subsection*{Esercizio 6.1}
Scrivere le derivazioni dei giudizi:
\begin{itemize}
	\item \{l:\{a:Nat, b:Nat\}, l':\{m:Nat\}\} <: \{l:\{a:Nat\}, l':\{\}\}
	\item \{l:\{a:Nat, b:Nat\}, l':\{m:Nat\}\} <: \{l:\{a:Nat\}, l':\{m:Nat\}\}	
	\item \{l:\{a:Nat, b:Nat\}, l':\{m:Nat\}\} <: \{l:\{a:Nat\}\}
\end{itemize} 

\subsubsection*{Svolgimento}
 
\textbf{A): \{l:\{a:Nat, b:Nat\}, l':\{m:Nat\}\} <: \{l:\{a:Nat\}, l':\{\}\}}
	
\begin{prooftree} 
	\AxiomC{(SUB-WIDTH)}
	\LeftLabel{\textsc{}}
	\UnaryInfC{$\emptyset\:\vdash$ l:\{a:Nat, b:Nat\}<: l:\{a:Nat\}}
	\AxiomC{(SUB-WIDTH)}
	\LeftLabel{\textsc{}}
	\UnaryInfC{$\emptyset\:\vdash$ l':\{m:Nat\}<: l':\{\}}
	\LeftLabel{\textsc{(SUB-DEPTH)}}
	\BinaryInfC{$\emptyset\:\vdash$ \{l:\{a:Nat, b:Nat\}, l':\{m:Nat\}\} <: \{l:\{a:Nat\}, l':\{\}\}}
\end{prooftree}
 
\textbf{B): \{l:\{a:Nat, b:Nat\}, l':\{m:Nat\}\} <: \{l:\{a:Nat\}, l':\{m:Nat\}\}}
	
\begin{prooftree} 
	\AxiomC{(SUB-WIDTH)}
	\LeftLabel{\textsc{}}
	\UnaryInfC{$\emptyset\:\vdash$ l:\{a:Nat, b:Nat\}<: l:\{a:Nat\}}
	\AxiomC{(REFLEX)}
	\LeftLabel{\textsc{}}
	\UnaryInfC{$\emptyset\:\vdash$ l':\{m:Nat\}<: l':\{m:Nat\}}
	\LeftLabel{\textsc{(SUB-DEPTH)}}
	\BinaryInfC{$\emptyset\:\vdash$ \{l:\{a:Nat, b:Nat\}, l':\{m:Nat\}\} <: \{l:\{a:Nat\}, l':\{m:Nat\}\}}
\end{prooftree}

\textbf{C): \{l:\{a:Nat, b:Nat\}, l':\{m:Nat\}\} <: \{l:\{a:Nat\}\}}
	
\scalebox{.85}{
\parbox{1cm}{
\begin{prooftree} 	
	\AxiomC{(SUB-WIDTH)}
	\LeftLabel{\textsc{}}
	\UnaryInfC{$\emptyset\:\vdash$ l:\{a:Nat, b:Nat\} <: l:\{a:Nat\}}
	\AxiomC{(SUB-WIDTH)}
	\LeftLabel{\textsc{}}
	\UnaryInfC{$\emptyset\:\vdash$ l':\{m:Nat\}<: l':\{\}}
	\LeftLabel{\textsc{(SUB-DEPTH)}}
	\BinaryInfC{$\emptyset\:\vdash$ \{l:\{a:Nat, b:Nat\}, l':\{m:Nat\}\} <: \{l:\{a:Nat\}, l':\{\}\}}
	\AxiomC{(SUB-WIDTH)}
	\LeftLabel{\textsc{}}
	\UnaryInfC{$\emptyset\:\vdash$ \{l:\{a:Nat\}, l':\{\}\} <: \{l:\{a:Nat\}\}}
	\LeftLabel{\textsc{(TRANS)}}
	\BinaryInfC{$\emptyset\:\vdash$ \{l:\{a:Nat, b:Nat\}, l':\{m:Nat\}\} <: \{l:\{a:Nat\}\}}
\end{prooftree}}}

\subsection*{Esercizio 6.2}
Si scriva la derivazione di \{a:Nat, b:Bool, c:Nat\} <: \{b:Bool\} 

\subsubsection*{Svolgimento}
\scalebox{.85}{
\parbox{1cm}{
 \begin{prooftree} 
	\AxiomC{\{a:Nat, b:Bool, c:Nat\} \`e permutazione di \{b:Bool, a:Nat, c:Nat\}}
	\LeftLabel{\textsc{PERMUTE}}
	\UnaryInfC{$\emptyset\:\vdash$ \{a:Nat, b:Bool, c:Nat\} <: \{b:Bool, a:Nat, c:Nat\}}
	\AxiomC{(SUB-WIDTH)}
	\LeftLabel{\textsc{}}
	\UnaryInfC{$\emptyset\:\vdash$ \{b:Bool, a:Nat, c:Nat\} <: \{b:Bool\}}
	\LeftLabel{\textsc{(TRANS)}}
	\BinaryInfC{$\emptyset\:\vdash$ \{a:Nat, b:Bool, c:Nat\} <: \{b:Bool\}}
\end{prooftree} }}

\subsection*{Esercizio 6.3}

Dare la derivazione del giudizio $\emptyset\:\vdash$ (fn r:\{l:Nat\}.r.l + 2) \{l= 0, l'= 1\}:Nat. Esiste una sola derivazione di questo giudizio? 

\subsubsection*{Svolgimento}
\scalebox{.75}{
\parbox{1cm}{
 \begin{prooftree} 
	\AxiomC{r.l:Nat $\in\:\Gamma$}	
	\LeftLabel{\textsc{(VAR)}}
	\UnaryInfC{$\emptyset$, r:\{l:Nat\} $\vdash$ r.l:Nat}
	\AxiomC{(NAT)}	
	\LeftLabel{\textsc{}}
	\UnaryInfC{$\emptyset$, r:\{l:Nat\} $\vdash$ 2:Nat}
	\LeftLabel{\textsc{(SUM)}}
	\BinaryInfC{$\emptyset$, r:\{l:Nat\} $\vdash$ (r.l + 2):Nat}
	\LeftLabel{\textsc{(FUN)}}
	\UnaryInfC{$\emptyset\:\vdash$ (fn r:\{l:Nat\}.r.l + 2):\{l:Nat\} $\rightarrow$ Nat}
	\AxiomC{(NAT)}
	\LeftLabel{\textsc{}}
	\UnaryInfC{$\emptyset\:\vdash$ 0:Nat}
	\AxiomC{(NAT)}
	\LeftLabel{\textsc{}}
	\UnaryInfC{$\emptyset\:\vdash$ 1:Nat}
	\LeftLabel{\textsc{(TYPE-RECORD)}}
	\BinaryInfC{$\emptyset\:\vdash$ \{l=0, l'=1\}:\{l:Nat, l':Nat\}}
	\AxiomC{(SUBWIDTH)}
	\LeftLabel{\textsc{}}
	\UnaryInfC{\{l:Nat, l':Nat\}<:\{l:Nat\}}
	\LeftLabel{\textsc{(SUBSUMPTION)}}
	\BinaryInfC{$\emptyset\:\vdash$ \{l=0, l'=1\}:\{l:Nat\}}
	\LeftLabel{\textsc{(APP)}}
	\BinaryInfC{$\emptyset\:\vdash$ (fn r:\{l:Nat\}.r.l + 2) \{l= 0, l'= 1\}:Nat}
\end{prooftree} }}

Esistono altre derivazioni di questo giudizio; infatti, in ogni momento posso applicare la regola SUBSUMPTION.
Esempio di alternativa:
 
 \begin{prooftree} 
	\AxiomC{(A)}
	\LeftLabel{\textsc{}}
	\UnaryInfC{$\emptyset\:\vdash$ (fn r:\{l:Nat\}.r.l + 2) \{l= 0, l'= 1\}:Nat}
	\AxiomC{(REFLEX)}
	\LeftLabel{\textsc{}}
	\UnaryInfC{Nat<:Nat}
	\LeftLabel{\textsc{(SUBSUMPTION)}} 
	\BinaryInfC{$\emptyset\:\vdash$ (fn r:\{l:Nat\}.r.l + 2) \{l= 0, l'= 1\}:Nat}
\end{prooftree}   

Dove A \`e la derivazione vista precedentemente. Potendo applicare SUBSUMPTION in qualsiasi punto della derivazione, anche pi\`u volte, ne consegue che il numero di derivazioni possibili \`e infinito. 

\subsection*{Esercizio 6.4}
Quale potrebbe essere la relazione di sottotipo dei variant types? 
 
\subsubsection*{Svolgimento} 
 \begin{prooftree} 
	\AxiomC{$\Gamma$ $\vdash$ M : $T_j$}
	\AxiomC{j $\in$ 1..n+k}	
	\LeftLabel{\textsc{(SUBTYPE VARIANT)}}
	\BinaryInfC{$\Gamma$ $\vdash$ $\langle$ $l_j$ = M $\rangle$ : $\langle$ $l_i$ : $T_i$\textsuperscript{i $\in$ 1..n} $\rangle$} 
\end{prooftree} 

In questo modo, modificando TYPE VARIANT, si ottiene il subtyping in larghezza (WIDTH). Per ottenere il subtyping in profondit\`a (DEPTH) basta usare SUBSUMPTION.
Questa regola, basa il sottotipaggio sul fatto che il typevariant sottotipo pu\`o avere un numero $\geq$ di elementi del sopra-tipo, l'importante e che i tipi di ogni elemento siano compresi uguali a $T_i $ con $i$ $\in$ $1,...,n $ dove $n$ \`e il numero di elementi del sopra-tipo.\\

Inoltre, sulla base del fatto che i variant-type sono simili ai record, \`e possibile ipotizzare un subtyping basato sulle reogle SUB-WITH e SUB-DEPTH. I variant-type, infatti, hanno la stessa struttura, la differenza risiede nel fatto che i campi di quest'ultimi vengono acceduti tramite \textit{pattern matching} e non tramite etichette.\\
Pertanto si possono avere anche le regole:\\

 \begin{prooftree} 
	\AxiomC{}	
	\LeftLabel{\textsc{(SUB-WIDTH VARIANT)}}
	\UnaryInfC{$\langle l_i$ : $T_i$ $^{i=1,...,n}\rangle $ $<:$ $\langle l_i$ : $T_i$ $^{i=1,...,n+k}\rangle $} 
\end{prooftree} 


 \begin{prooftree} 
	\AxiomC{$\forall$ $i$ $\in$ $1,...,n$ $S$ $<:$ $T$}	
	\LeftLabel{\textsc{(SUB-DEPTH VARIANT)}}
	\UnaryInfC{$\langle l_i$ : $S_i$ $^{i=1,...,n}\rangle $ $<:$ $\langle l_i$ : $T_i$ $^{i=1,...,n}\rangle $} 
\end{prooftree} 


 \begin{prooftree} 
	\AxiomC{$\langle k_i$ : $S_j$ $^{j=1,...,n}\rangle $ \`e una permutazione di $\langle l_i$ : $T_i$ $^{i=1,...,n}\rangle $ }	
	\LeftLabel{\textsc{(SUB-PERMUTE VARIANT)}}
	\UnaryInfC{$\langle k_i$ : $S_j$ $^{j=1,...,n}\rangle $ $<:$ $\langle l_i$ : $T_i$ $^{i=1,...,n}\rangle $} 
\end{prooftree} 

\subsection*{Esercizio 6.5}
Quali propriet\`a del sistema di tipi si perderebbero se avessimo definito la relazione di
subtyping con una regola di troppo? E se l'avessimo definita con una regola in meno? Se avessimo definito le regole in modo tale che \{l : Nat\} <: \{l : Nat, l' : Nat\}, quale propriet\`a del sistema non sarebbe pi\`u vera? Identificarla e darne un controesempio.
\subsection*{Svolgimento}

\subparagraph*{Regola di troppo}

Aggiungendo una regola in pi\`u, come ad esempio la regola per che rendere derivabile il giudizio:

$\{l$:$Nat\}$ <: $\{l$:$Nat,l_1$:$Nat\}$ 

ovvero la regola

\begin{prooftree} 
	\AxiomC{}	
	\LeftLabel{\textsc{(SUB-WIDTH-INVERSE)}}
	\UnaryInfC{$\{l_i $ : $T_i$ $i \in 1,...,n\} <$ :$ \{l_i$ : $T_i$ $i \in 1,...,n+k\}$} 
\end{prooftree} 

Tale regola pu\`o portare alla generazione di un termine STUCK invalidando il teorema di Safety.

\subparagraph*{Regola in meno}

Gli effetti che si otterrebbero con una regola in meno sono diversi e dipendono dalla regola che viene tolta.
\begin{itemize}
\item (SUBSUMPTION): la rimozione di questa regola comporta la perdita della possibilit\`a di usare un sottotipo dove viene utilizzato il sopra-tipo.
\item (REFLEX): la rimozione di questa regola comporterebbe degli assurdi, come ad esempio l'impossibilit\`a di poter derivare la regola SUB-DEPTH per record i cui tipi dei valori siano uguali, in quanto un tipo $T$ generico non sarebbe pi\`u sotto-tipo di se stesso.In sostanza avremo una situazione di questo tipo: $T \not <$: $T$.
\item (TRANS):  la rimozione di questa regola comporterebbe la rimozione della possibilit\`a di rendere transitiva la propriet\`a e la relazione di subtyping impedendo di avere relazioni di subtyping a pi\`u livelli. Poniamo $U$ e $S$ sottotipi di un tipo generico $T$, se mancasse tale regola non si riuscirebbe a derivare un giudizio del tipo $S<$:$T$ sulla base di $S <$:$U$ e $U<$:$T$. In sostanza la derivazione tramite l'utilizzo di tipi intermedi non sarebbe pi\`u sopportata e quindi la gerarchia di tipo verrebbe meno.
\item (SUB-WIDTH):
la rimozione di questa regola renderebbe limiterebbe la relazione di subtyping per quanto riguarda i record. Questa regola infatti permette di una relazione di subtyping tra il record $\{l$:$Nat,l_1$:$Nat\}$ e il record $\{l$:$Nat\}$ permettendo di utilizzare il primo (specifico) al posto del secondo (generale). La mancanza di tale regola, come detto, limita fortemente il subtyping per i record perch\`e obbliga sempre l'utilizzo di record della stessa \emph{larghezza}, tuttavia la relazione di subtyping tra record, continua ad esistere in altre forme, come quelle garantite dall SUB-DEPTH.
\item (SUB-DEPTH): 
la rimozione di questa regola, come nel caso precedente, limita la relazione di subtyping dei record, anche se pi\`u debolmente rispetto alla precedente. Senza questa regola, la relazione di subtyping tra record non potrebbe pi\`u appoggiarsi sulla relazione di tipo esistente tra i corrispettivi tipi dei valori dei record. Supponendo di avere $S <$: $T$, un record   $\{l$:$S\}$ e un record $\{l$:$T\}$, nonostante il primo record possa potenzialmente essere utilizzato ovunque venga utilizzato il secondo, la mancanza di questa regola ne impedisce di derivare il giudizio che ne garantirebbe il funzionamento.
\item (PERMUTE):
la rimozione di questa regola, elimina la possibilit\`a di avere giudizi come ad esempio $\{a$:$Nat,b$:$Nat\}$ <: $\{b$:$Nat\}$, questo perch\`e senza questa regola la relazione di suptyping risulta essere posizionale e non gestirebbe il sottotipo con permutazione degli elementi del record.
\item (ARROW):
la rimozione di questa regola le funzioni accetterebbero solo il tipo $T$ scritto nella signatura della funzione stessa senza accettare un sottotipo del tipo specificato. Inoltre il tipo di ritorno della funzione sarebbe sempre quello pi\'u specifico. In sostanza il sottotipaggio per le funzioni non potrebbe esistere.
\end{itemize}
\subparagraph*{Regola invertita}

Se fosse definita la relazione secondo quanto scritto, esisterebbe la regola di SUB-WIDTH-INVERSE ovvero:

\begin{prooftree} 
	\AxiomC{}	
	\LeftLabel{\textsc{(SUB-WIDTH-INVERSE)}}
	\UnaryInfC{$\{l_i $ : $T_i$ $i \in 1,...,n\} <$ :$ \{l_i$ : $T_i$ $i \in 1,...,n+k\}$} 
\end{prooftree} 

Tale regola permetterebbe di usare record del tipo $\{l$:$Nat\}$ al posto di $\{l$:$Nat,l_1$:$Nat\}$ cosi facendo si potrebbero creare situazioni come ad esempio:

$(fn$ $x:\{l$:$Nat,l_1$:$Nat\}.x.l_1 + 1)$ $(\{l=1\})$

Questa funzione andrebbe a generare un termine STUCK invalidando di fatto il teorema di Safety:

\emph{Sia $M$ un termine chiuso e ben tipato. Allora $M$ non evolve ad un termine stuck, ma $\exists v$ tale che $M \rightarrow \ast$ $v$ oppure $M \rightarrow \ast$  throw v.}


\subsection*{Esercizio 6.12}
Ridimostrare il teorema di progressione, preservazione, il substitution lemma e il teorema
di safety per il linguaggio con i record e il subtyping. 

\subsection*{Svolgimento}

(Lemmi di inversione)
\begin{itemize}
	\item Se $\Gamma \vdash x : T$ allora $\exists S$ tale che $x : S \in \Gamma $e $S <: T$.
	\item Se $\Gamma \vdash fn x:S1M : T$ \`e derivabile, allora $\exists T_1, T_2$ tali che $T = T_1 \rightarrow T_2$ , con $T_1 <: S_1$ e
	$\Gamma, x : S_1 \vdash M : T_2$.
	\item Se $\Gamma \vdash \{k_r = M_r\ ^{r \in 1\dots m}\} : T$ \`e derivabile allora
	$\exists T_i, i = 1, \dots, n$ tali che $T = \{\ell_i : T_i\ ^{i \in 1\dots n}\}$,
	con $\{\ell_i\ ^{i \in 1 \dots n}\} \subseteq \{k_r\ ^{r \in 1 \dots m} \}$
	e $\Gamma \vdash M_r : T_i$ per ogni etichetta comune $k_r = \ell_i$.
\end{itemize}

Inoltre sempre a lezione \`e stato definito il lemma di inversione della relazione di subtyping:
(Lemma di inversione della relazione di subtyping)
	\begin{enumerate}
	\item Se $S <: T_1 \rightarrow T_2$ allora $S$ \`e della forma $S_1 \rightarrow S_2$ con $T_1 <: S_1$ e $S_2 <: T_2$.
	\item Se $S <: \{\ell_i : T_i\ ^{i \in 1 \dots n} \}$ allora $S$ \`e della forma
	$\{k_j : S_j\ ^{j \in 1 \dots m} \}$ tale che $\{\ell_i\ ^{i \in 1 \dots n}\} \subseteq \{k_j\ ^{j \in 1 \dots m} \}$ e $S_j <: T_i$ per ogni etichetta comune
	$\ell_i = k_j$
	\end{enumerate}

(Lemma delle Forme Canoniche)
\begin{itemize}
 \item Se $v$ \`e un valore di tipo $T_1 \rightarrow T_2$ allora $v$ \`e della forma $fn x:S1M$.
 \item Se $v$ \`e un valore di tipo $\{\ell_i:T_i\ ^{i \in 1 \dots n}\}$, allora $v$ \`e della forma $\{k_j = v_j\ ^{j \in 1 \dots m} \}$
 tale che $\{\ell_i\ ^{i \in 1 \dots n} \} \subseteq \{k_j\ ^{j \in 1 \dots m}\}$
\end{itemize}

\paragraph{Dimostrazione del substitution Lemma}
Per provare il substitution Lemma devo estendere la definizione induttiva di $fv(M)$ e la definizione di sostituzione per i tipi 
record e i tipi select.
(Variabili libere)
	Vengono riportati solo i casi che estendono la definizione fornita in Note $2$:
	\begin{itemize}
		\item $fv(\{\ell_i = M_i\ ^{i \in 1 \dots n} \}) = \bigcup_{i = 1 \dots n} fv(M_i)$
		\item $fv(M.l) = fv(M)$
	\end{itemize}


(Sostituzione)
	Vengono riportati solo i casi che estendono la definizione fornita in Note $2$:
	\begin{itemize}
		\item $\{\ell_i = M_i\ ^{i \in 1 \dots n} \} \{x := N\} = \{\ell_i = M_i \{x \coloneqq N\}\ ^{i \in 1 \dots n} \}$
		\item $M.l\{x \coloneqq N \} = M\{x \coloneqq N \}.l$
	\end{itemize}



(Substitution Lemma)
Se $\Gamma, x : S \vdash M : T$ e $\Gamma \vdash N : S$, allora $\Gamma \vdash M\{x := N\}:T.$	

\begin{proof}
Procedo per induzione sulla lunghezza della derivazione del giudizio
$\Gamma, x : S \vdash M : T$. La dimostrazione procede distinguendo l'ultima
regola di tipo che \`e stata applicata per derivare tale giudizio:
\begin{itemize}
	\item \textbf{Casi Base} (h = 1): Quelli in cui \`e stato applicato un assioma per le regole dei tipi.
	\begin{itemize}
		\item $\boxed{\myrule{True}}$. 
		$\Gamma, x : S \vdash \true : T \equiv \Bool$. La tesi $\Gamma \vdash \true\{x \coloneqq N\} : \Bool$, visto
		che $\true \{x \coloneqq N \} \equiv \true$, si ottiene che la tesi \`e:
		$\Gamma \vdash \true: \Bool$ che \`e vera per l'assioma \myrule{True}.
		\item $\boxed{\myrule{False}}$: analogo al caso \myrule{True}.
		\item $\boxed{\myrule{Nat}}$: analogo al caso \myrule{True}.
		\item $\boxed{\myrule{Var}}$. Questo caso differisce leggermente da quello visto a lezione, in quanto il lemma di inversione
		per le variabili \`e stato modificato e asserisce che se $\Gamma \vdash x : T$, allora $x: S \in \Gamma$ con $S <: T$. 
		Tuttavia, se il giudizio con il tipo preciso non \`e nel contesto (ovvero $S \neq T$), l'albero non pu\`o essere composto dal solo assioma \myrule{Var} ma
		deve essere presente anche l'applicazione di \myrule{Subsumption}.
		Quindi, per ricondurre questo caso a quello mostrato a lezione dimostro un lemma:
		\vspace{10pt}
		
			Se il giudizio $\Gamma \vdash z : T$ \`e stato ottenuto dall'assioma \myrule{Var}, allora $z : T \in \Gamma$.
			(Vale il lemma di inversione del capitolo $45$)
			\begin{proof}
			In generale dall'assunzione $\Gamma \vdash z : T$ segue per il lemma di inversione (di note 9) che $\exists S$ tale che $z: S \in \Gamma$ e  $S <:T$. 
			Se $S <:T$ possono darsi solo due casi:
			\begin{itemize}
				\item $\boxed{S = T}$ la regola $\myrule{Var}$ pu\`o essere applicata perch\`e il giudizio $z : T \in \Gamma$
				\item $\boxed{S \neq T}$ la regola \myrule{Var} non pu\`o essere applicata 
				perch\`e il giudizio $z : T \notin \Gamma$. 
			\end{itemize}
			\end{proof}
		
		Quindi per questo caso vale il lemma di inversione (pi\`u forte) del capitolo $45$. E quindi la dimostrazione fornita a lezione
		risulta essere ancora valida in questa estensione del linguaggio.
		
		Ovvero se $\Gamma , x : S \vdash y : T$ viene derivato con la regola \myrule{Var} possono verificarsi due casi:
		\begin{enumerate}
			\item $y = x$:  $\Gamma , x : S \vdash x : T$. Per il lemma di inversione (delle note 45), il giudizio deriva da un'asserzione presente nel contesto e quindi per forza $S = T$. Si ha quindi che considerando la sostituzione $\Gamma \vdash x\{x:=N\}:T $. Ma $N$ ha tipo $S$ e $S = T$, quindi la tesi continua a valere.
			\item $y \neq x$: in questo caso la sostituzione $\Gamma \vdash y\{x:=N\}:T $ non altera il termine, in pi\`u per l'assioma \myrule{Var} $y : T \in \Gamma, x :S$ e quindi \`e derivabile il giudizio $\Gamma \vdash y :T$.
		\end{enumerate}\vspace{.5em}

		\noindent \textbf{Ipotesi induttiva}: Da $\Gamma, x : S \vdash_k M : T$ e $\Gamma \vdash N : S$ implica che 
		$\Gamma \vdash M\{x \coloneqq N\} : T$.
		
		\item \textbf{Passo Induttivo}:  
		I casi \myrule{Sum}, \myrule{Minus}, \myrule{Fun} sono analoghi a quelli visti a lezione. Visto che questo teorema
		vuole dimostrare giudizi del tipo $\Gamma \vdash M : T$ le regole di sottotipo non potranno essere mai l'ultima regola 
		applicata (poich\`e dimostrano giudizi solo del tipo $\Gamma \vdash M <: T$) e quindi non devono essere considerate. Queste hanno infatti un ruolo
		indiretto nel caso \myrule{Subsumption}.
		\begin{itemize}
			\item $\boxed{\myrule{Sum}}$: Invariato rispetto a quanto visto a lezione.
			\item $\boxed{\myrule{Minus}}$: Invariato rispetto a quanto visto a lezione.
			\item $\boxed{\myrule{Fun}}$: Invariato rispetto a quanto visto a lezione.
			\item $\boxed{\myrule{App}}$: Invariato rispetto a quanto visto a lezione
			\item $\boxed{\myrule{Type-Record}}$
			In questo caso $M \equiv \{\ell_i = M_i\ ^{i \in 1 \dots n} \}$. Assumo le due ipotesi per questo termine
			quindi 
			\begin{enumerate}
				\item $\Gamma, x : S \vdash_{k+1} \{\ell_i = M_i\ ^{i \in 1 \dots n}\} : \{\ell_i : T_i\}$
				\item $\Gamma \vdash N : S$
			\end{enumerate}
			Siccome l'ultima regola applicata \`e \myrule{Type-Record} le sue premesse (al passo precedente) devono essere verificate quindi
			$\forall i \in 1 \dots n$ $ \Gamma, x : S \vdash_{k_i} M_i : T_i$. 
			Siccome le derivazioni dei giudizi per i sottotermini sono strettamente inferiori (di almeno un'unit\`a) rispetto
			a quella di partenza e vale $\Gamma \vdash N : S$, ottengo per ipotesi induttiva (applicata agli $n$ $M_i$) che
			$\forall i \in 1 \dots n$ $\Gamma \vdash M_i \{x \coloneqq N\} : T_i$.
			Applicando la regola $\myrule{Type-Record}$ ottengo
			$\Gamma \vdash \{\ell_i : M_i\{x \coloneqq N\}\ ^{i \in 1 \dots n} \} : \{\ell_i : T_i\ ^{i \in 1 \dots n}\}$ che \`e proprio la tesi da dimostrare visto che per definizione:
			
			$$
			\{\ell_i : M_i\ ^{i \in 1 \dots n}\}\{x \coloneqq N \} = \{\ell_i : M_i\{x \coloneqq N \}\ ^{i \in 1 \dots n}\}
			$$
			
			\item $\boxed{\myrule{Type-Select}}$:
			$M \equiv M.\ell_j$
			Assumo che valgano le due ipotesi ovvero:
			\begin{enumerate}
				\item $\Gamma, x : S  \vdash_{k+1} M.\ell_j : T$, lo rinomino $T_j$ per comodit\`a.
				\item $\Gamma \vdash N : S$ 
			\end{enumerate}
			Ora siccome vale la prima ipotesi e l'ultima regola applicata \`e \myrule{Type-Select} le premesse devono essere soddisfatte
			quindi
			$\Gamma, x : S \vdash M : \{\ell_i : T_i\ ^{i \in 1 \dots n}\}$ con $j \in 1 \dots n$.
			Siccome questo giudizio ha una derivazione di lunghezza strettamente inferiore rispetto a quello di partenza  e vale
			$\Gamma \vdash N : S$ posso applicare l'ipotesi induttiva e ottenere che
			$\Gamma \vdash M\{x \coloneqq N \} : \{\ell_i : T_i\ ^{i \in 1 \dots n}\}$ quindi riapplicando \myrule{Type-Select} ottengo:
			$$
			\Gamma \vdash M\{x \coloneqq N \}.\ell_j : T_j
			$$
			che \`e proprio la definizione di $M.\ell_j \{x \coloneqq N \}$
			
			\item $\boxed{\myrule{Subsumption}}$: 
			Assumo le due ipotesi per questa derivazione ovvero:
			\begin{enumerate}
				\item $\Gamma, x : S \vdash_{k+1} M : T$
				\item $\Gamma \vdash N : S$
			\end{enumerate}
			Visto che l'ultima regola applicata \`e stata $\myrule{Subsumption}$ posso dire che valgono le sue premesse:
			
		    %\prftree[r]
			$\myrule{Subsumption}$ 	$\Gamma, x : S \vdash_k M : U$ $U <: T$  $\Gamma$, $x : S$ $\vdash_{k+1}$ $M$ : $T$
		
			Ora siccome vale $\Gamma, x : S \vdash_k M$ : $U$ e $\Gamma \vdash N : S$ posso applicare l'ipotesi induttiva 
			e ottenere che $\Gamma \vdash M\{ x \coloneqq N \} : U $. 
			E visto che $U$ \`e un sottotipo di $T$ l'asserto \`e dimostrato per la regola \myrule{Subsumption}.\\
			
			$\myrule{Subsumption}$ $\Gamma$ $\vdash$ $M\{x \coloneqq N \}$ : $U$  $U <: T$ $\Gamma$ $\vdash$ $M\{x \coloneqq N\}$: $T$
			
		\end{itemize}
	\end{itemize} 
\end{itemize}
	
	
\end{proof}


\paragraph{Dimostrazione del teorema di preservazione (per induzione su $M \to M'$)}

	Se $\Gamma \vdash M : T$ e $M \longrightarrow M'$, allora
	$\Gamma \vdash M':T$


Procedo per induzione sull'altezza della derivazione di $M \longrightarrow M'$.
Inizio dai casi base, che sono quelli in cui $M \longrightarrow M'$ viene da un assioma.
%% TO DO controllare le regole di derivazione dei record %%
\begin{itemize}
	\item $\boxed{\text{\textbf{Caso Base}}}$
	I casi base per le regole di derivazione \myrule{Sum}, \myrule{Minus}, \myrule{If-True}, \myrule{If-False} rimangono
	invariati rispetto a quelli mostrati a lezione. Il caso \myrule{Beta} \`e leggermente diverso rispetto a quello 
	mostrato a lezione per via del lemma di inversione modificato, e infine va aggiunto il nuovo assioma \myrule{Select}.
	\begin{itemize}
		
		\item $\boxed{\myrule{Beta}}$
		$M\equiv (fn x:S1 M_1\ v) : T \longrightarrow M_1 \{x \coloneqq v\}$  
		Dall'ipotesi $\Gamma \vdash fn x:S1 M_1\ v : T$ segue per il lemma di inversione 
		che esiste $T_A$ tale che $\Gamma \vdash v : T_A$ e $\Gamma \vdash fn x:S1M_1: T_A \rightarrow T$.
		
		Da $\Gamma \vdash fn x:S1M_1 : T_A \rightarrow T$ segue per il lemma di inversione di note 9:
		\begin{enumerate}
			\item $T_A <: S_1$
			\item $\Gamma, x : S_1 \vdash M_1 : T$
			\label{ex9-12:enum:SecondaIpotesiLemmaSostituzione}
		\end{enumerate}
			
		Visto che
		$$
		%\prftree[l]
		{\myrule{Subsumption}}
		{\Gamma \vdash v : T_A}
		{T_A <: S_1}
		{\Gamma \vdash v : S_1}
		$$
		ho ricavato che $\Gamma \vdash v : S_1$. Considerato che vale \ref{ex9-12:enum:SecondaIpotesiLemmaSostituzione} posso applicare
		il lemma di sostituzione e ottenere la tesi $\Gamma \vdash M_1 \{x \coloneqq v \} : T$.
		
		\item $\boxed{\myrule{Select}}$ $\{\ell_i = v_i^{i \in 1 \dots n}\}.\ell_j \longrightarrow v_j$, grazie all'assioma $\myrule{Select}$ (con $j \in 1 \dots n$). Per ipotesi
		$\Gamma \vdash \{\ell_i = v_i^{i \in 1 \dots n}\}.\ell_j$ ha tipo $T$. Per comodit\`a e in linea 
		con la regola $\myrule{Type-Select}$
		rinomino $T$ in $T_j$. Quindi per il lemma di inversione della regola $\myrule{Type-Select}$ ottengo che:
		$\Gamma \vdash M : \{\ell_i : T_i\ 	^{i\in 1 \dots n}\}$ con $j \in 1 \dots n$. Ora, siccome per la regola \myrule{Select} $v_j$ \`e stato ottenuto dalla
		selezione del valore della label $\ell_j$ che ha tipo $T_j$, deduco che $v_j$
		ha il tipo $T_j$ cercato. Visto che $T_j$ non \`e altro che un alias per $T$
		la tesi \`e dimostrata.
	\end{itemize}
	\item $\boxed{\text{\textbf{Passo Induttivo}}}$
	Procedo ora con i casi induttivi, quelli cio\`e che in cui $M \longrightarrow M'$ \`e stato derivato
	con una derivazione di altezza $k+1$. Distinguendo
	i vari casi a seconda dell'ultima regola usata:
	i casi \myrule{Sum-Left}, \myrule{Sum-Right}, \myrule{Minus-Left}, \myrule{Minus-Right}, \myrule{If}
	sono invariati visto che vale il lemma di inversione di note 2.
	\begin{itemize}
	
		\item $\boxed{\myrule{Eval-Select}}$ $M \equiv N.\ell_j \longrightarrow N'.\ell_j \equiv M'$.
		Assumo le due ipotesi per questo termine:
		\begin{enumerate}
			\item $	N.\ell_j \longrightarrow N'.\ell_j$:
			Visto che ho assunto che $N.\ell_j \longrightarrow N'.\ell_j$ \`e stato derivato con un albero di derivazione alto
			$k+1$ e l'ultima regola applicata \`e stata \myrule{Eval-Select} la premessa della regola $N \longrightarrow N'$ deve
			essere verificata. Graficamente:
			$$
			%\prftree[r]
			{\myrule{Eval-Select}}
			{N \longrightarrow N'}
			{N.\ell_j \longrightarrow N'.\ell_j}
			$$
			
			\item $\Gamma \vdash N.\ell_j : T$ rinomino $T$ in $T_j$. Quindi per il lemma di inversione della regola \myrule{Type-Select}
			$\Gamma \vdash N : \{\ell_i : T_i \ ^{i \in 1 \dots n}\}$ con $j \in 1 \dots n$.
		\end{enumerate}
		Quindi applicando l'ipotesi induttiva sulla derivazione $N \longrightarrow N'$ (che \`e alta $k$), si ottiene che:
		$$
		\Gamma \vdash N' : \{\ell_i : T_i\ ^{i \in 1 \dots n } \}
		$$
		Applicando la regola \myrule{Type-Select} ottengo $N'.\ell_j : T_j$ che \`e la tesi visto che $T_j$ \`e un altro nome per $T$.
		
		\item $\boxed{\myrule{Eval-Record}}$ 
		Assumo come di consueto le due ipotesi:
		\begin{enumerate}
			\item 
			$$
			%\prftree[r]
			{\myrule{Eval-Record}}
			{M_x \longrightarrow M_x'}
			{
				\{\ell_i : v_i\ ^{i \in 1 \dots x-1}, \ell_x : M_x, \ell_j : M_j\ ^{j \in x+1 \dots m}\} \longrightarrow
				\{\ell_i : v_i\ ^{i \in 1 \dots x-1}, \ell_x : M'_x, \ell_j : M_j\ ^{j \in x+1 \dots m}\}
			}
			$$
			Con $M_x \to M'_x$ che viene derivato con un albero di derivazione di altezza inferiore rispetto a quello per $M \to M'$.
			Si pu\`o quindi applicare l'ipotesi induttiva ad $M'_x$, ottenendo che \`e derivabile il giudizio $\Gamma \vdash M'_x : T_x$.
			\item $\{\ell_i : v_i\ ^{i \in 1 \dots x-1}, \ell_x : M_x, \ell_j : M_j\ ^{j \in x+1 \dots m}\} : T$ \`e derivabile.
			Quindi per il lemma di inversione $\exists T_i, \: i = 1 \dots n$ tale che 
			$T = \{\ell_i : T_i\ ^{i \in 1 \dots n} \}$ con $\{\ell_i\ ^{i \in 1 \dots n} \} \subseteq \{k_r\ ^{r \in 1 \dots m} \}$ 
			e per ogni etichetta in comune $k_r = \ell_i$. $\Gamma \vdash M_r : T_i$.
			Sostanzialmente stiamo dicendo che $M$ pu\`o essere un'istanza di un sottotipo $S$ di $T$ e non necessariamente il tipo $T$.
		\end{enumerate}
	% Quindi se sai che M ha fatto un passo ed \`e andato in M' perch\`e M_x \to M'_x puoi applicare l'ipotesi induttiva (perch\`e M_x \to M'_x \`e pi\`u corta)per dire che \Gamma M'_x : T_x \`e ancora valido. (Che poi M'_x sia un T_x o un S_x con S_x <: T_x te ne puoi fregare IMHO)
	
	Dal momento che $M' = \{\ell_i : v_i\ ^{i \in 1 \dots x-1}, \ell_x : M'_x, \ell_j : M_j\ ^{j \in x+1 \dots m}\}$ posso applicare la regola di tipo \myrule{Type-Record} per derivare il giudizio $\Gamma \vdash M' : T$ con $T = \{ \ell_i : T_i \:^{i \in 1 \ldots n} \}$.
	Sono infatti soddisfatte le premesse della regola perch\`e i termini $M_i$ associati alle $\ell_i$ con $i \in 1 \ldots n,\: i \neq x$ non sono cambiati e quindi anche il loro tipo non \`e cambiato, mentre per ipotesi induttiva $\Gamma \vdash M'_j : T_j$.
	La tesi \`e quindi provata.
	\end{itemize}
\end{itemize}

\paragraph{Dimostrazione del teorema di preservazione (per induzione su $\Gamma \vdash M : T$)}

	Se $\Gamma \vdash M : T$ e $M \longrightarrow M'$, allora
	$\Gamma \vdash M':T$

\begin{proof}
Procedo in analogia all'esercizio 3.12, che dimostra $\Gamma \vdash M : T$ per
induzione sull'altezza dell'albero della prova.
\begin{itemize}
	\item $\boxed{\text{\textbf{Casi Base}}} (h = 1)$: Casi in cui \`e stato applicato uno degli assiomi. Si veda
	l'esercizio 3.12.
	\item $\boxed{\text{\textbf{Passo Induttivo}}}$ Ora siccome l'ipotesi \`e verificata
	per i casi base dimostro che vale anche per i casi induttivi a seconda dell'ultima regola usata. 
	Rispetto all'esercizio 3.12 sono state aggiunte solo
	3 regole di tipo:
	\begin{itemize}
		\item $\boxed{\myrule{Type-Record}}$ assumendo le due ipotesi, ovvero:
		\begin{enumerate}
			\item $\Gamma \vdash M : T$ visto che l'ultima regola applicata \`e
			 \myrule{Type-Record}, allora
			 $M \equiv \{\ell_i = M_i\ ^{i \in 1 \dots n} \}$ e
			 $T = \{\ell_i : T_i\ ^{i \in 1 \dots n}\}$. Dal momento che la sua
			 premessa deve essere verificata si ha quindi
			 $\forall i \in 1 \dots n$. $\Gamma \vdash M_i : T_i$.
			
			\item $M \longrightarrow M'$
		\end{enumerate}
		
		Per il lemma delle forme canoniche dalla premessa 1 segue che $M$ potrebbe essere un valore, ma in questo caso
		la seconda ipotesi non verrebbe verificata.
		
		Visto che $M$ non \`e un valore, l'unica regola che fa ridurre un $M$ di tipo
		$ \{\ell_i : M_i\ ^{i \in 1 \dots n} \}$ \`e:
		$$
		%\prftree[l]
		{\myrule{Eval-Record}}
		{M_j \longrightarrow M_j'}
		{
			\{\ell_i = v_i\ ^{i \in 1 \dots j-1}, \ell_j = M_j, \ell_p = M_p\ ^{p \in j+1 \dots n}\}
			\longrightarrow 
			\{\ell_i = v_i\ ^{i \in 1 \dots j-1}, \ell_j = M_j', \ell_p = M_p\ ^{p \in j+1 \dots n}\}
		}
		$$
		Siccome il giudizio $\Gamma \vdash M_j : T_j$ \`e pi\`u corto di un passo rispetto
		a $\Gamma \vdash M_j : T_j$ e $M_j \longrightarrow M_j'$,
		allora per ipotesi induttiva concludo che $\Gamma \vdash M_j' : T_j$.
		Ora devo dimostrare che
		$\{\ell_i = v_i\ ^{i \in 1 \dots j-1}, \ell_j = M_j', \ell_p = M_p\ ^{p \in j+1 \dots n}\}$ ha tipo $T$.
		%Basta applicare al fatto $\forall i \in 1 \dots n.\ $ $\Gamma \vdash M_i : T_i$ e $\Gamma \vdash M_j':T$ la regola di 
		%\myrule{Type-Record} per concludere la tesi.
		Dal momento che $M' = \{\ell_i : v_i\ ^{i \in 1 \dots j-1}, \ell_j : M'_j, \ell_p : M_p\ ^{p \in j+1 \dots m}\}$ posso applicare la regola di tipo \myrule{Type-Record} per derivare il giudizio $\Gamma \vdash M' : T$ con $T = \{ \ell_i : T_i \:^{i \in 1 \ldots n} \}$.
		Sono infatti soddisfatte le premesse della regola perch\`e i termini $M_i$ associati alle $\ell_i$ con $i \in 1 \ldots n,\: i \neq j$ non sono cambiati e quindi anche il loro tipo non \`e cambiato, mentre per ipotesi induttiva $\Gamma \vdash M'_j : T_j$.
		La tesi \`e quindi provata.

		\item $\boxed{\myrule{Type-Select}}$ $M = N.\ell_j$
		\begin{enumerate}
			\item $\Gamma \vdash N.\ell_j : T_j$. Visto che l'ultima regola applicata \`e stata \myrule{Type-Select} le sue premesse devono essere
			verificate quindi $\Gamma \vdash N : \{\ell_i : T_i\ ^{i \in 1 \dots n} \}$ con $j \in 1 \dots n$.
			\item $M \longrightarrow M'$. Per questi tipi possono valere due opzioni:
				\begin{enumerate}[label=\alph*)]
					\item $N = \{\ell_i = v_i\ ^{i \in 1 \dots n} \}$ \`e un valore. Ma questo
					caso \`e immediato (non ha bisogno dell'ipotesi induttiva) perch\`e per
					l'assioma $\myrule{Select}$ si conclude che $N.\ell_j \longrightarrow v_j$,
					per quanto mostrato dall'ipotesi 1, ha tipo $T_j$.
					\item $N$ non \`e un valore valutato e quindi $M'= N'.\ell_j$.
				\end{enumerate} 
		\end{enumerate}
		Considero dunque solo il caso in cui $N$ non \`e un valore.
		
		Visto che la lunghezza della derivazione del giudizio di tipo per $N$ \`e strettamente inferiore di quella 
		per $N.\ell_j$ e vale $ N.\ell_j \longrightarrow N'.\ell_j$ (perch\`e $N$ non \`e un valore) vale anche la premessa
		alla regola $\myrule{Eval-Select}$ e quindi $N \longrightarrow N'$.
		Quindi posso applicare l'ipotesi induttiva e concludere che $N'$ ha lo stesso
		tipo di $N$, ovvero $\{\ell_i : T_i\ ^{i \in 1 \dots n} \}$.
		Applicando la regola $\myrule{Type-Select}$ ed usando
		$N' : \{\ell_i : T_i\ ^{i \in 1 \dots n} \}$ come premessa ottengo il risultato
		cercato, ovvero $N'.\ell_j : T_j$.
	
		\item $\boxed{\myrule{Subsumption}}$ $M = \{\ell_i = M_i\ ^{i \in 1 \dots n} \}$ Assumo le due ipotesi:
		\begin{enumerate}
			\item $\Gamma \vdash M : T$
			\item $M \longrightarrow M'$
		\end{enumerate} 
		Visto che l'ultima regola applicata \`e stata \myrule{Subsumption} posso assumere che le sue premesse siano verificate:
		$$
		%\prftree[l]
		{\myrule{Subsumption}}
		{\Gamma \vdash M : S}
		{S <: T}
		{\Gamma \vdash M : T}
		$$
		Visto che il giudizio $\Gamma \vdash M : S$ \`e pi\`u corto di almeno un'unit\`a rispetto a $\Gamma \vdash M : T$ e $M \longrightarrow M'$
		allora per ipotesi induttiva concludo che $\Gamma \vdash M' : S$. 
		Quindi ho ottenuto le premesse per poter applicare la regola $\myrule{Subsumption}$ a $M'$:
		$$
		%\prftree[l]
		{\myrule{Subsumption}}
		{\Gamma \vdash M' : S}
		{S <: T (\text{vero per quanto mostrato prima})}
		{\Gamma \vdash M' : T}
		$$
		$\Gamma \vdash M' : T$ era la tesi da dimostrare, perci\`o la dimostrazione del
		teorema di preservazione \`e conclusa.
	
	\end{itemize}
\end{itemize}
\end{proof}


\paragraph{Dimostrazione del teorema di progressione}
(Progressione).
	 Sia $M$ un termine chiuso e ben tipato, i.e. $\emptyset \vdash M : T $, allora o $M$ \`e un
	valore, oppure esiste un termine $M'$ tale che $M \longrightarrow M'$.

\begin{proof}
Procedo per induzione sull'altezza dell'albero di prova che dimostra
$\varnothing \vdash M : T$.
\begin{itemize}
	\item $\boxed{\text{\textbf{Casi Base}} (h = 1)}$: I casi base rimangono
	quelli mostrati a lezione, visto che gli assiomi per i giudizi di tipo non
	sono stati estesi. \vspace{.5em}

	\noindent \textbf{Ipotesi induttiva} (h = k): Se $\varnothing \vdash_{k} \bar{M} : T$ allora $\bar{M}$ \`e un valore oppure $\exists
	\bar{M'}: \bar{M} \longrightarrow \bar{M'}$
	\item $\boxed{\text{\textbf{Passo induttivo}} (h = k+1)}$ I casi che rimangono invariati rispetto a lezione vengono omessi. Per via
	del lemma di inversione delle forme canoniche per il subtyping, la
	dimostrazione per $\myrule{Type-Select}$ e $\myrule{Type-Record}$
	\`e leggermente differente a quella mostrata nell'esercizio 6.3. Mentre la regola \myrule{Subsumption} \`e ``inedita":
	

	\begin{itemize}
		\item \myrule{Type-Record} Se $M = \{ \ell_i = M_i \:^{i = 1 \ldots n} \}$ e $\emptyset \vdash M : T$, sappiamo che per la regola \myrule{Type-Record} valgono i giudizi di tipo
		$\emptyset \vdash M_i : T_i \:\ \forall \: i = 1\ldots n$ e che gli alberi di derivazione relativi ai vari giudizi sono di altezza inferiore all'albero del giudizio principale.
		\`e quindi possibile applicare l'ipotesi induttiva sui sotto-termini $M_i$, i quali o sono un valore $v_i$ di tipo $T_i$ oppure possono avanzare in un termine $M_i'$:
		\begin{itemize}
			\item Se sono tutti dei valori si ha $M = \{ \ell_i = v_i \:^{i = 1 \ldots n} \}$, ovvero $M$ \`e un valore record di tipo $\{ \ell_i : T_i \:^{i = 1 \ldots n} \}$ e quindi il teorema di Progressione continua a valere banalmente.
			\item Se c'\`e almeno un $M_i$ che non \`e un valore, \`e possibile identificare il termine $M_j$ di indice minimo che non \`e un valore e per il quale continua a valere il giudizio di tipo $\emptyset \vdash M_j : T_j$. Vale quindi l'ipotesi induttiva, ovvero $\exists M_j'$ tale che $M_j \to M_j'$. Posso quindi applicare la regola \myrule{Eval-Record} per far avanzare il termine $M$ al termine $M'$, dove al posto di $M_j$ compare $M_j'$. Anche in questo caso il teorema di Progressione continua a valere.
		\end{itemize}
	
		\item \myrule{Type-Select} Se $M = N.\ell_j$ e $\emptyset \vdash M : T$, sappiamo che per la regola \myrule{Type-Select} vale il giudizio di tipo $\emptyset \vdash N : \{ \ell_i : T_i \:^{i = 1 \ldots n}\}$ (con un albero di derivazione pi\`u piccolo) e che $j \in \{1 \ldots n\}$.
		Si ha quindi che $N$ o \`e un buon valore finale oppure $\exists N' : N \longrightarrow N'$.
		\begin{itemize}
			\item Se $N$ \`e un valore $v$ per il lemma delle forme canoniche di note 9 ottengo che
			$v$ \`e della forma $\{k_q = v_q \:^{q \in 1 \dots m} \}$ tale che $\{\ell_i\ ^{i \in 1 \dots n} \} \subseteq \{k_q\ ^{q \in 1 \dots m}\}$.
			Siccome l'etichetta $\ell_j \in \{k_q\ ^ {q \in 1 \dots m}\}$ allora $\exists$ un indice $p \in 1 \dots m : \ell_j = k_p$ e quindi $M = N.\ell_j$ si riscrive grazie alla regola \myrule{Select} nel valore $v_p$.
			
			\item Un record non ancora completamente valutato, ovvero $N = \{ \ell_i = v_i \:^{i = 1 \ldots x-1}$, $ l_x = M_x, l_k = M_k \:^{k = x+1\ldots n}\}, \exists M_x'| M_x \to M_x'$ e quindi per \myrule{Eval-Record} $N \to N'$ e pertanto anche $M \to M' = N'.\ell_j$.
			
			\item Un termine generico di tipo record, ovvero $N = \IF{M_1}{M_2}{M_3}$ oppure $N = A \: B$. In entrambi i casi vale il giudizio $\emptyset \vdash N : \{ \ell_i : T_i \:^{i = 1 \ldots n}\}$, cio\`e $N$ \`e un termine chiuso, ben tipato e non \`e un valore. 
			Quindi per ipotesi induttiva ho che $\exists N'$ tale che $N \to N'$ e quindi per \myrule{Eval-Select} $\exists M' = N'.\ell_j$ tale che $M \to M'$.
			Pertanto il teorema di Progressione continua a valere.
			
		\end{itemize}
		
		
		
		
		\item \myrule{Subsumption}
		$$
		%\prftree[r]
		{\myrule{Subsumption}}
		{
			%\prftree[r]
			{}
			{\vdots}
			{\varnothing \vdash_{k} M : S}
		}
		{
			%\prftree[r]
			{}
			{\vdots}
			{S <: T}
		}
		{\varnothing \vdash_{k+1} M : T}
		$$
		Ora, siccome la derivazione del giudizio $\varnothing \vdash M  : S$ \`e pi\`u
		corta di (almeno) un passo di derivazione rispetto a quella di partenza,
		posso concludere per ipotesi induttiva che $M$ o \`e un buon valore finale o
		che $\exists M': M \longrightarrow M'$. Siccome questo $M$ corrisponde a
		quello di partenza, la tesi \`e dimostrata.

	\end{itemize}

\end{itemize}
\end{proof}

\paragraph{Teorema di Safety}
\begin{centering}
	Se $\emptyset \vdash M :T$ e $M \to^* M'$ con $M'$ tale che $M' \not\to$, allora $M'$ \`e un valore.
\end{centering}

\noindent La dimostrazione \`e una diretta conseguenza dei due teoremi precedenti: per il teorema di Progressione anche $\emptyset \vdash M': T$ \`e derivabile e, per il teorema di Subject-Reduction $M'$ deve essere un valore, perch\`e altrimenti esisterebbe un $M''$ al quale pu\`o ridursi.


\subsection*{Esercizio 6.16}
Trovare due termini M e N tali che M $\rightarrow$ N, $\Gamma\vdash$ M : T, $\Gamma\vdash$ N : S con S<::T e T$\nless$::S, cio\`e esibire un caso in cui il tipo di un termine decresce durante la computazione. 
\subsubsection*{Svolgimento}
Una possibile soluzione potrebbe essere la seguente:

$ (fn$ $x$:$\{l$:$Nat\}.\{l_i = x.l_i^{i \in 1,...,x.length},l_j^{j=x.lenght+1} = 0\})$ $({l=4})$

Questa funzione, in sostanza non fa altro che ritornare un record con gli stessi campi del primo pi\`u un ulteriore campo.
Cos\`i facendo il tipo del parametro della funzione \`e $T$  ma accetta qualsiasi tipo $S$ tale che $S$ <: $T$, tuttavia il tipo di ritorno \`e sempre pi\`u specifico ( a causa della nuova etichetta aggiunta) e di conseguenza, per come abbiamo definito la regola di subtyping, il tipo di ritorno $U$ non sar\`a mai sopra-tipo di $T$.

Quindi abbiamo trovato due termini, $M$:$T$ e $N$:$S$ tali che $M \rightarrow N$ con le caratteristiche richieste.

Inoltre, per come definita, questa funzione rispetta quanto richiesto anche nel subtyping non algoritmico.