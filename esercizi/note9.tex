\section{Subtyping (note 9)}
\subsection*{Esercizio 7.1}
Scrivere le derivazioni dei giudizi:
\begin{itemize}
	\item \{l:\{a:Nat, b:Nat\}, l':\{m:Nat\}\} <: \{l:\{a:Nat\}, l':\{\}\}
	\item \{l:\{a:Nat, b:Nat\}, l':\{m:Nat\}\} <: \{l:\{a:Nat\}, l':\{m:Nat\}\}	
	\item \{l:\{a:Nat, b:Nat\}, l':\{m:Nat\}\} <: \{l:\{a:Nat\}\}
\end{itemize} 

\subsubsection*{Svolgimento}
 
\textbf{A): \{l:\{a:Nat, b:Nat\}, l':\{m:Nat\}\} <: \{l:\{a:Nat\}, l':\{\}\}}
	
\begin{prooftree} 
	\AxiomC{(SUB-WIDTH)}
	\LeftLabel{\textsc{}}
	\UnaryInfC{$\emptyset\:\vdash$ l:\{a:Nat, b:Nat\}<: l:\{a:Nat\}}
	\AxiomC{(SUB-WIDTH)}
	\LeftLabel{\textsc{}}
	\UnaryInfC{$\emptyset\:\vdash$ l':\{m:Nat\}<: l':\{\}}
	\LeftLabel{\textsc{(SUB-DEPTH)}}
	\BinaryInfC{$\emptyset\:\vdash$ \{l:\{a:Nat, b:Nat\}, l':\{m:Nat\}\} <: \{l:\{a:Nat\}, l':\{\}\}}
\end{prooftree}
 
\textbf{B): \{l:\{a:Nat, b:Nat\}, l':\{m:Nat\}\} <: \{l:\{a:Nat\}, l':\{m:Nat\}\}}
	
\begin{prooftree} 
	\AxiomC{(SUB-WIDTH)}
	\LeftLabel{\textsc{}}
	\UnaryInfC{$\emptyset\:\vdash$ l:\{a:Nat, b:Nat\}<: l:\{a:Nat\}}
	\AxiomC{(REFLEX)}
	\LeftLabel{\textsc{}}
	\UnaryInfC{$\emptyset\:\vdash$ l':\{m:Nat\}<: l':\{m:Nat\}}
	\LeftLabel{\textsc{(SUB-DEPTH)}}
	\BinaryInfC{$\emptyset\:\vdash$ \{l:\{a:Nat, b:Nat\}, l':\{m:Nat\}\} <: \{l:\{a:Nat\}, l':\{m:Nat\}\}}
\end{prooftree}

\textbf{C): \{l:\{a:Nat, b:Nat\}, l':\{m:Nat\}\} <: \{l:\{a:Nat\}\}}
	
\scalebox{.85}{
\parbox{1cm}{
\begin{prooftree} 	
	\AxiomC{(SUB-WIDTH)}
	\LeftLabel{\textsc{}}
	\UnaryInfC{$\emptyset\:\vdash$ l:\{a:Nat, b:Nat\} <: l:\{a:Nat\}}
	\AxiomC{(SUB-WIDTH)}
	\LeftLabel{\textsc{}}
	\UnaryInfC{$\emptyset\:\vdash$ l':\{m:Nat\}<: l':\{\}}
	\LeftLabel{\textsc{(SUB-DEPTH)}}
	\BinaryInfC{$\emptyset\:\vdash$ \{l:\{a:Nat, b:Nat\}, l':\{m:Nat\}\} <: \{l:\{a:Nat\}, l':\{\}\}}
	\AxiomC{(SUB-WIDTH)}
	\LeftLabel{\textsc{}}
	\UnaryInfC{$\emptyset\:\vdash$ \{l:\{a:Nat\}, l':\{\}\} <: \{l:\{a:Nat\}\}}
	\LeftLabel{\textsc{(TRANS)}}
	\BinaryInfC{$\emptyset\:\vdash$ \{l:\{a:Nat, b:Nat\}, l':\{m:Nat\}\} <: \{l:\{a:Nat\}\}}
\end{prooftree}}}

\subsection*{Esercizio 7.2}
Si scriva la derivazione di \{a:Nat, b:Bool, c:Nat\} <: \{b:Bool\} 

\subsubsection*{Svolgimento}
\scalebox{.85}{
\parbox{1cm}{
 \begin{prooftree} 
	\AxiomC{\{a:Nat, b:Bool, c:Nat\} e' permutazione di \{b:Bool, a:Nat, c:Nat\}}
	\LeftLabel{\textsc{PERMUTE}}
	\UnaryInfC{$\emptyset\:\vdash$ \{a:Nat, b:Bool, c:Nat\} <: \{b:Bool, a:Nat, c:Nat\}}
	\AxiomC{(SUB-WIDTH)}
	\LeftLabel{\textsc{}}
	\UnaryInfC{$\emptyset\:\vdash$ \{b:Bool, a:Nat, c:Nat\} <: \{b:Bool\}}
	\LeftLabel{\textsc{(TRANS)}}
	\BinaryInfC{$\emptyset\:\vdash$ \{a:Nat, b:Bool, c:Nat\} <: \{b:Bool\}}
\end{prooftree} }}

\subsection*{Esercizio 7.3}

Dare la derivazione del giudizio $\emptyset\:\vdash$ (fn r:\{l:Nat\}.r.l + 2) \{l= 0, l'= 1\}:Nat. Esiste una sola derivazione di questo giudizio? 

\subsubsection*{Svolgimento}
\scalebox{.75}{
\parbox{1cm}{
 \begin{prooftree} 
	\AxiomC{r.l:Nat $\in\:\Gamma$}	
	\LeftLabel{\textsc{(VAR)}}
	\UnaryInfC{$\emptyset$, r:\{l:Nat\} $\vdash$ r.l:Nat}
	\AxiomC{(NAT)}	
	\LeftLabel{\textsc{}}
	\UnaryInfC{$\emptyset$, r:\{l:Nat\} $\vdash$ 2:Nat}
	\LeftLabel{\textsc{(SUM)}}
	\BinaryInfC{$\emptyset$, r:\{l:Nat\} $\vdash$ (r.l + 2):Nat}
	\LeftLabel{\textsc{(FUN)}}
	\UnaryInfC{$\emptyset\:\vdash$ (fn r:\{l:Nat\}.r.l + 2):\{l:Nat\} $\rightarrow$ Nat}
	\AxiomC{(NAT)}
	\LeftLabel{\textsc{}}
	\UnaryInfC{$\emptyset\:\vdash$ 0:Nat}
	\AxiomC{(NAT)}
	\LeftLabel{\textsc{}}
	\UnaryInfC{$\emptyset\:\vdash$ 1:Nat}
	\LeftLabel{\textsc{(TYPE-RECORD)}}
	\BinaryInfC{$\emptyset\:\vdash$ \{l=0, l'=1\}:\{l:Nat, l':Nat\}}
	\AxiomC{(SUBWIDTH)}
	\LeftLabel{\textsc{}}
	\UnaryInfC{\{l:Nat, l':Nat\}<:\{l:Nat\}}
	\LeftLabel{\textsc{(SUBSUMPTION)}}
	\BinaryInfC{$\emptyset\:\vdash$ \{l=0, l'=1\}:\{l:Nat\}}
	\LeftLabel{\textsc{(APP)}}
	\BinaryInfC{$\emptyset\:\vdash$ (fn r:\{l:Nat\}.r.l + 2) \{l= 0, l'= 1\}:Nat}
\end{prooftree} }}

\subsection*{Esercizio 7.4}
Quale potrebbbe essere la relazione di sottotipo dei variant types? 
 
\subsubsection*{Svolgimento} 
 \begin{prooftree} 
	\AxiomC{$\Gamma$ $\vdash$ M : $T_j$}
	\AxiomC{j $\in$ 1..n+k}	
	\LeftLabel{\textsc{(SUBTYPE VARIANT)}}
	\BinaryInfC{$\Gamma$ $\vdash$ <$l_j$ = M> : <$l_i$ : $T_i$\textsuperscript{i $\in$ 1..n}>} 
\end{prooftree} 

In questo modo, modificando TYPE VARIANT, si ottiene il subtyping in larghezza (WIDTH). Per ottenere il subtyping in profondita' (DEPTH) basta usare SUBSUMPTION.

\subsection*{Esercizio 7.5}
Quali proprieta' del sistema di tipi si perderebbero se avessimo definito la relazione di
subtyping con una regola di troppo? E se l'avessimo definita con una regola in meno? E se l'avessimo definita con una regola in meno? Se avessimo definito le regole in modo tale che \{l : Nat\} <: \{l : Nat, l' : Nat\}, quale proprieta' del sistema non sarebbe piu' vera? Identificarla e darne un controesempio.
\textbf{{\color{red} DA FARE}}

\subsection*{Esercizio 7.12}
Ridimostrare il teorema di progressione, preservazione, il substitution lemma e il teorema
di safety per il linguaggio con i record e il subtyping. 
\textbf{{\color{red} DA FARE}}

\subsection*{Esercizio 7.16}
Trovare due termini M e N tali che M $\rightarrow$ N, $\Gamma\vdash$ M : T, $\Gamma\vdash$ N : S con S<::T e T<::S, cioe' esibire un caso in cui il tipo di un termine decresce durante la computazione. 
\textbf{{\color{red} DA FARE}}