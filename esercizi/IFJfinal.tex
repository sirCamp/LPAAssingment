\section{Imperative Featherweight Java (note 12)}
\vspace{1cm}

\subsection*{\fbox{Esercizio 8.1}}
Scrivere una regola per eliminare i riferimenti non pi\`u utilizzati.


Svolgimento:


I riferimenti agli oggetti che non vengono pi\`u utilizzati all'interno di un termine e, possono essere eliminati.

consideriamo la seguente regola:

\begin{prooftree}
	\AxiomC{o $\notin$ fref($e$)}
	\AxiomC{o $\notin$ Codom($\sigma$)}	
	\BinaryInfC{$\langle\sigma\:[o\mapsto(C,f\widetilde{:}v)]\:,\:e\rangle$ $\rightarrow$ $\langle\sigma\:,\:e\rangle$}
	\end{prooftree}

\vspace{0,5cm}

\begin{itemize}
\item fref($e$) = Insieme dei riferimenti a tutti gli oggetti utilizzati nel termine $e$
\item Codom($\sigma$) = Insieme degli oggetti che sono riferiti da altri oggetti presenti nella memoria.
\end{itemize}

\vspace{1,5cm}



\subsection*{\fbox{Esercizio 8.2}}
Descrivere il comportamento del seguente programma:
 
\begin{lstlisting}
class D extends Object {

	Object f; 
	
	D(Object f) {super(); this.f=f;}
 
	Object m() {return this;}
	
}


class C extends D {

	C(Object f) {super();}
 
	Object m() {return this;}

}

Object z=new Object();

C x=new C(z);

C y=new D(x);

x.m(); x=y; x.m()

y.f=new Object();

z=null; x.f=z; y.f.m();

\end{lstlisting}

Risoluzione:

Definiamo la configurazione iniziale sostituendo allo store $\sigma$ l'insieme vuoto($\emptyset$) ed all'espressione $e$ l'insieme delle istruzioni del programma. Otteniamo dunque la seguente configurazione iniziale:


$\langle\emptyset\:,\:Object\:z=new\:Object();\:C\:x=new\:C(z);\:.....\rangle$

\begin{itemize}

\vspace{0,5cm}
\item Creazione oggetto $o_1$ da memorizzare nella variabile z

$\rightarrow$ $\langle\emptyset\:,\:Object\:z=o_1;\:C\:x=new\:C(z);\:.....\rangle$

\vspace{0,5cm}

\begin{prooftree}
	\AxiomC{$o_1 \notin Dom(\emptyset)$}
	\AxiomC{field(Object)= $\emptyset$}
	\AxiomC{$|\emptyset| = |\emptyset|$ }
	\RightLabel{(NEW)}
	\TrinaryInfC{$\langle\emptyset,\:new Object()\rangle$ $\rightarrow$ $\langle\emptyset\:[o_1\mapsto (Object)]\:,\:o_1\rangle$}
	\LeftLabel{(CONG)}
	\UnaryInfC{$\langle\emptyset\:,E[t]\rangle$ $\rightarrow$ $\langle\emptyset\:[o_1\mapsto (Object)]\:,\:E[o_1]\rangle$}
	\end{prooftree}
\vspace{1cm}

		\begin{itemize}
		\item[-] E[] $\equiv$ Object z = []
		\item[-] $t$ $\equiv$ new Object() 
		\end{itemize}

\vspace{0,5cm}
\item Associazione dell'oggetto $o_1$ a z

$\rightarrow$ $\langle\emptyset\:\:[o_1\:\mapsto\:(Object)]\:\:,\:C\:x=new\:C(z);\:.....\rangle$

\begin{prooftree}
	\AxiomC{$z\:\notin Dom(\sigma)$ }
	\RightLabel{(NEW)}
	\UnaryInfC{$\langle\sigma\:,\:Object\:z=\:o_1\:;\:e\rangle$ $\rightarrow$ $\langle\sigma\:[z\mapsto o_1]\:,\:e\rangle$}
	\LeftLabel{(CONG)}
	\UnaryInfC{$\langle\sigma\:,E[t]\rangle$ $\rightarrow$ $\langle\sigma\:[z\:\mapsto\:o_1]\:,\:E[o_1]\rangle$}
	\end{prooftree}
\vspace{1cm}

		\begin{itemize}
		\item[-] E[] $\equiv$ []
		\item[-] $t$ $\equiv$ Object z=$o_1$ 
		\item[-] $\sigma\:\equiv\:\emptyset\:\:[o_1\:\mapsto\:(Object)]$
		\end{itemize}


\vspace{0,5cm}
\item Dereferenzazione della varibile z per creare l'oggetto $o_2$

$\rightarrow$ $\langle\emptyset\:\:[o_1\:\mapsto\:Object]\:[z\:\mapsto:o_1]\:,\:C\:x=new\:C(o_1);\:.....\rangle$

\vspace{0,5cm}
\item Creazione dell'oggetto $o_2$ da associare nella variabile x

$\rightarrow$ $\langle\emptyset\:\:[o_1\:\mapsto\:Object]\:[z\:\mapsto\:o_1]\:[o_2\:\mapsto\:(C,f:o_1)]\:,\:C\:x=new\:C(z);\:.....\rangle$

\vspace{0,5cm}
\item Associazione dell'oggetto $o_2$ ad x

$\rightarrow$ $\langle\emptyset\:\:[o_1\:\mapsto\:Object]\:[z\:\mapsto\:o_1]\:[o_2\:\mapsto\:(C,f:o_1)]\:[x\:\mapsto\:o_2]\:,\:C\:y=new\:D(x);\:.....\rangle$

\vspace{0,5cm}
\item Dereferenzazione della variabile x per creare l'oggetto $o_2$

$\rightarrow$ $\langle\emptyset\:\:[o_1\:\mapsto\:Object]\:[z\:\mapsto\:o_1]\:[o_2\:\mapsto\:(C,f:o_1)]\:[x\:\mapsto\:o_2]\:,\:C\:y=new\:D(o_2);\:.....\rangle$

\vspace{0,5cm}
\item Creazione dell'oggetto $o_3$ di tipo D da associare nella variabile y di tipo C. La classe C \`e un sottotipo della classe D, per\`o non aggiunge nessun nuovo campo dati e nessun nuovo metodo, effettua solamente l'override del metodo m().

$\rightarrow$ $\langle\emptyset\:\:[o_1\:\mapsto\:Object]\:[z\:\mapsto\:o_1]\:[o_2\:\mapsto\:(C,f:o_1)]\:[x\:\mapsto\:o_2]\:[o_3\:\mapsto\:(D,f:o_2)]\:,\:C\:y=o_3;\:.....\rangle$

\vspace{0,5cm}
\item Associazione dell'oggetto $o_3$ ad y

$\rightarrow$ $\langle\emptyset\:\:[o_1\:\mapsto\:Object]\:[z\:\mapsto\:o_1]\:[o_2\:\mapsto\:(C,f:o_1)]\:[x\:\mapsto\:o_2]\:[o_3\:\mapsto\:(D,f:o_2)]\:[y\:\mapsto\:o_3]\:,\:x.m();.....\rangle$

\vspace{0,5cm}
\item Dereferenzazione di x per effettuare l'invocazione del metodo m().

$\rightarrow$ $\langle\emptyset\:\:[o_1\:\mapsto\:Object]\:[z\:\mapsto\:o_1]\:[o_2\:\mapsto\:(C,f:o_1)]\:[x\:\mapsto\:o_2]\:[o_3\:\mapsto\:(D,f:o_2)]\:[y\:\mapsto\:o_3]\:,\:o_2.m();.....\rangle$

\vspace{0,5cm}
\item Invocazione del metodo m() della classe C perch\`e la funzione mbody effettua la scelta di tale metodo in base alla class-table dell'oggetto di invocazione, il metodo restituisce l'oggetto stesso quindi $o_2$.

Lo store rimane inalterato e si ha solo il passaggio all'istruzione successiva.

$\rightarrow$ $\langle\emptyset\:\:[o_1\:\mapsto\:Object]\:[z\:\mapsto\:o_1]\:[o_2\:\mapsto\:(C,f:o_1)]\:[x\:\mapsto\:o_2]\:[o_3\:\mapsto\:(D,f:o_2)]\:[y\:\mapsto\:o_3]\:,\:x=y;.....\rangle$

\vspace{0,5cm}
\item Dereferenzazione della variabile y.

$\rightarrow$ $\langle\emptyset\:\:[o_1\:\mapsto\:Object]\:[z\:\mapsto\:o_1]\:[o_2\:\mapsto\:(C,f:o_1)]\:[x\:\mapsto\:o_2]\:[o_3\:\mapsto\:(D,f:o_2)]\:[y\:\mapsto\:o_3]\:,\:x=o_3;.....\rangle$

\vspace{0,5cm}
\item Associazione dell'oggetto $o_3$ di tipo D alla variabile x. Si effetua una modifica allo store in quanto la variabile x gi\`a esisteva ed era associata all'oggetto $o_2$.

$\rightarrow$ $\langle\emptyset\:\:[o_1\:\mapsto\:Object]\:[z\:\mapsto\:o_1]\:[o_2\:\mapsto\:(C,f:o_1)]\:\underline{[x\:\mapsto\:o_3]}\:[o_3\:\mapsto\:(D,f:o_2)]\:[y\:\mapsto\:o_3]\:,\:x.m();.....\rangle$

\vspace{0,5cm}
\item Dereferenzazione di x per effettuare l'invocazione del metodo m().

$\rightarrow$ $\langle\emptyset\:\:[o_1\:\mapsto\:Object]\:[z\:\mapsto\:o_1]\:[o_2\:\mapsto\:(C,f:o_1)]\:[x\:\mapsto\:o_3]\:[o_3\:\mapsto\:(D,f:o_2)]\:[y\:\mapsto\:o_3]\:,\:o_3.m();.....\rangle$

\vspace{0,5cm}
\item Invocazione del metodo m() della classe D perch\`e la funzione mbody effettua la scelta di tale metodo in base alla class-table dell'oggetto di invocazione, il metodo restituisce l'oggetto stesso quindi $o_3$.

Lo store rimane inalterato e si ha solo il passaggio all'istruzione successiva.

$\rightarrow$ $\langle\emptyset\:\:[o_1\:\mapsto\:Object]\:[z\:\mapsto\:o_1]\:[o_2\:\mapsto\:(C,f:o_1)]\:[x\:\mapsto\:o_3]\:[o_3\:\mapsto\:(D,f:o_2)]\:[y\:\mapsto\:o_3]\:,\:y.f=new\:Object();.....\rangle$

\vspace{0,5cm}
\item Dereferenzazione di y

$\rightarrow$ $\langle\emptyset\:\:[o_1\:\mapsto\:Object]\:[z\:\mapsto\:o_1]\:[o_2\:\mapsto\:(C,f:o_1)]\:[x\:\mapsto\:o_3]\:[o_3\:\mapsto\:(D,f:o_2)]\:[y\:\mapsto\:o_3]\:,\:o_3.f=new\:Object();.....\rangle$

\vspace{0,5cm}
\item Creazione dell'oggetto $o_4$ di tipo Object da associare alla struttura dati f dell'oggetto $o_3$, associato alla variabile y, che referenziava l'oggetto $o_2$.

$\rightarrow$ $\langle\emptyset\:\:[o_1\:\mapsto\:Object]\:[z\:\mapsto\:o_1]\:[o_2\:\mapsto\:(C,f:o_1)]\:[x\:\mapsto\:o_3]\:\underline{[o_3\:\mapsto\:(D,f:o_4)]}\:[y\:\mapsto\:o_3]\:[o_4\:\mapsto\:Object]\:,\:o_3.f=o_4;.....\rangle$

\vspace{0,5cm}
\item Associazione alla struttura dati dell'oggetto $o_3$ associato alla variabile y.

Lo store rimane inalterato e si ha solo il passaggio all'istruzione successiva.

$\rightarrow$ $\langle\emptyset\:\:[o_1\:\mapsto\:Object]\:[z\:\mapsto\:o_1]\:[o_2\:\mapsto\:(C,f:o_1)]\:[x\:\mapsto\:o_3]\:[o_3\:\mapsto\:(D,f:o_4)]\:[y\:\mapsto\:o_3]\:[o_4\:\mapsto\:Object]\:,\:z=null;.....\rangle$

\vspace{0,5cm}
\item Associazione di \textbf{null} alla variabile z. 

$\rightarrow$ $\langle\emptyset\:\:[o_1\:\mapsto\:Object]\:\underline{[z\mapsto\:null]}\:[o_2\:\mapsto\:(C,f:o_1)]\:[x\:\mapsto\:o_3]\:[o_3\:\mapsto\:(D,f:o_4)]\:[y\:\mapsto\:o_3]\:[o_4\:\mapsto\:Object]\:,\:x.f=z;.....\rangle$

\vspace{0,5cm}
\item Dereferenzazione di z.

$\rightarrow$ $\langle\emptyset\:\:[o_1\:\mapsto\:Object]\:\underline{[z\mapsto\:null]}\:[o_2\:\mapsto\:(C,f:o_1)]\:[x\:\mapsto\:o_3]\:[o_3\:\mapsto\:(D,f:o_4)]\:[y\:\mapsto\:o_3]\:[o_4\:\mapsto\:Object]\:,\:x.f=null;.....\rangle$

\vspace{0,5cm}
\item Dereferenzazione di x.

$\rightarrow$ $\langle\emptyset\:\:[o_1\:\mapsto\:Object]\:\underline{[z\mapsto\:null]}\:[o_2\:\mapsto\:(C,f:o_1)]\:[x\:\mapsto\:o_3]\:[o_3\:\mapsto\:(D,f:o_4)]\:[y\:\mapsto\:o_3]\:[o_4\:\mapsto\:Object]\:,\:o_3.f=null;.....\rangle$

\vspace{0,5cm}
\item Assegnazione di $null$ ad $o_3.f$. Considerando che la variabile y si riferisce all'oggetto $o_3$, abbiamo che anche y.f = $null$

$\rightarrow$ $\langle\emptyset\:\:[o_1\:\mapsto\:Object]\:[z\mapsto\:null]\:[o_2\:\mapsto\:(C,f:o_1)]\:[x\:\mapsto\:o_3]\:[o_3\:\mapsto\:(D,f:null)]\:[y\:\mapsto\:o_3]\:[o_4\:\mapsto\:Object]\:,\:y.f.m();\rangle$

\vspace{0,5cm}
\item Dereferenzazione di y.

$\rightarrow$ $\langle\emptyset\:\:[o_1\:\mapsto\:Object]\:[z\mapsto\:null]\:[o_2\:\mapsto\:(C,f:o_1)]\:[x\:\mapsto\:o_3]\:[o_3\:\mapsto\:(D,f:null)]\:[y\:\mapsto\:o_3]\:[o_4\:\mapsto\:Object]\:,\:o_3.f.m();\rangle$

\vspace{0,5cm}
\item Il programma termina restituendo NPE, in quanto si tenta di invocare il metodo m() su $null$.

\end{itemize}


\vspace{1,5cm}
\subsection*{\fbox{Esercizio 8.3}}
\
\\
\
\vspace{0,6cm}
Dare una derivazione di tipo per il termine C x=$o_1$; $o_1$.f=y; x=$o_2$.
\vspace{0,6cm}

\begin{prooftree}
	\AxiomC{$o_1\::\:F\:\in\:\Gamma$}
	\RightLabel{(OID)}
	\UnaryInfC{$\Gamma\:\vdash\:o_1:F$}
	\AxiomC{(A)}
	\RightLabel{(CLASS)}
	\UnaryInfC{$F\:<:\:C $}
	\AxiomC{(B)}
	\AxiomC{(C)}
	\UnaryInfC{$\Gamma\:,\:x=o_2\::T$}
	\RightLabel{(SEQ)}
	\BinaryInfC{$\Gamma\:,\:x:C\:\vdash\:o_1.f=y,\:x=o_2\::\:T$}
	\LeftLabel{(DICHIAR)}
	\TrinaryInfC{$C\:x=o_1;\:o_1.f=y;\:x=o_2\::\:T$}
	\end{prooftree}
	
\vspace{2,5cm}

\begin{itemize}

\item[(A)]
\
\
($CT(F)\:=\:class\:F\:extends\:C\:\{..\}$)
\vspace{1cm}

\item[(B)]
\
\\
\
\vspace{0,3cm}
\begin{bprooftree}
	\AxiomC{$o_1\::\:F\:\in\:\Gamma$}
	\RightLabel{(Oid)}
	\UnaryInfC{$\Gamma\:,\:x:C\:\vdash\:o_1\::\:F$}
	\AxiomC{$f\:\in\:field(N)$}
	\RightLabel{(Fld)}
	\BinaryInfC{$\Gamma\:,\:x:C\:\vdash\:o_1.f\::\:N$}
	\AxiomC{$CT(N)..$}
	\RightLabel{(CL)}
	\UnaryInfC{$N\:<:\:M $}
	\AxiomC{$y\::\:M\:\in\:\Gamma$}
	\RightLabel{(Var)}
	\UnaryInfC{$\Gamma\:,\:x:C\:\vdash\:y\::\:M$}
	\LeftLabel{(Fld Ass)}
	\TrinaryInfC{$\Gamma\:,\:x:C\:\vdash\:o_1.f=y\::\:M$}
	\end{bprooftree}

\vspace{1cm}

\item[(C)]
\
\\
\\
\\
\
\vspace{0,8cm}
\begin{bprooftree}
	\AxiomC{$o_2\::\:T\:\in\:\Gamma$}
	\RightLabel{(OID)}
	\UnaryInfC{$\Gamma\:,\:x:C\:\vdash\:o_2\::\:T$}
	\AxiomC{$CT(T)= class T\:extends\:C\{..\}$}
	\RightLabel{(CL)}
	\UnaryInfC{$T\:<:\:C $}
	\AxiomC{$x\::\:C\:\in\:\Gamma$}
	\RightLabel{(Var)}
	\UnaryInfC{$\Gamma\:\vdash\:x\::\:C$}
	\LeftLabel{(Ass)}
	\TrinaryInfC{$\Gamma\:,\:x:C\:\vdash\:x=o_2\::\:T$}
	\end{bprooftree}

\end{itemize}


