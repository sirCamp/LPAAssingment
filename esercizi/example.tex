 
\begin{align*}
	x \in Var & &\\
	n \in Num & &\\
	Termini \: M, N &::= x &\text{ variabili} \\
								&|\: n \:|\: \text{true} \:|\: \text{false} &\text{ costanti intere e booleane} \\
								&|\: M + M \:|\: M - M &\text{ operazioni intere} \\
								&|\: \text{if} \: M \: &\text{then} \: M \: \text{else} \: M \text{ condizionale} \\
								&|\: \text{fn } x.M &\text{ dichiarazione di una funzione} \\
								&|\: M \: M &\text{ applicazione di una funzione}
\end{align*}

$$M = (\fn x.\iif x \text{ then true else false})\: (\text{if false then fn }y.\text{ true else } 4)$$

\noindent L'unica regola che posso applicare è \textsc{App2}, perché la prima parte del termine è già un valore.

$$(\fn x.\iif x \text{ then true else false}) \:(\text{if false then fn }y.\text{ true else } 4) \rightarrow (\fn x.\iif x \text{ then true else false})\: 4$$

\noindent Adesso posso applicare \textsc{Beta}:

$$(\fn x.\iif x \text{ then true else false})\: 4 \rightarrow \iif 4 \text{ then true else false}$$

\noindent ottenendo un termine stuck perché la guardia dell'\text{if} non è booleana.

I hope that you find this template both visually appealing and useful. \\

\begin{prooftree}
	\AxiomC{$\checkmark$}
	\LeftLabel{\textsc{Sum}}
	\UnaryInfC{$2+1 \rightarrow 3$}
	\LeftLabel{\textsc{(Minus-Right)}}
	\UnaryInfC{$5-(2+1) \rightarrow 5 -3$}
	\LeftLabel{\textsc{(App2)}}
	\UnaryInfC{$(\fn x.x+x+x) \: \: 5-(2+1) \rightarrow (\fn x.x+x+x)\: (5-3)$}
\end{prooftree}