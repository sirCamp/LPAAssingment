\section{Il mini linguaggio funzionale (note 2)}
\subsection*{Esercizio 1.1}
La relazione di riduzione data definisce una strategia efficiente per la valutazione del termine if-then-else, che permette cioè di valutare unicamente il ramo scelto dalla valutazione della guardia
booleana. Ridefinire la semantica operazionale del linguaggio in modo che adotti una strategia non efficiente per il costrutto if-then-else, valutando entambi i rami del costrutto condizionale.

\subsubsection*{Svolgimento}
Per ottenere una semantica meno efficiente, si cambiano gli assiomi (IF-TRUE) e (IF-FALSE) e si aggiungono le regole (THEN) ed (ELSE). (IF) si mantiene tale.

\paragraph{Assiomi}
\begin{prooftree}
	\AxiomC{}
	\LeftLabel{\textsc{(IF-TRUE)}}
	\UnaryInfC{$if \: TRUE \: then \: v_1 \: else \: v_2 \rightarrow \: v_1$}
\end{prooftree}

\begin{prooftree}
	\AxiomC{}
	\LeftLabel{\textsc{(IF-FALSE)}}
	\UnaryInfC{$if \: FALSE \: then \: v_1 \: else \: v_2 \rightarrow \: v_2$}
\end{prooftree}

\paragraph{Regole}
\begin{prooftree}
	\AxiomC{$M_2 \rightarrow \: M_2'$}
	\LeftLabel{\textsc{(THEN)}}
	\UnaryInfC{$if \: v_1 \: then \: M_2 \: else \: M_3 \rightarrow \: 
	if \: v_1 \: then \: M_2' \: else \: M_3$}
\end{prooftree}

\begin{prooftree}
	\AxiomC{$M_3 \rightarrow \: M_3'$}
	\LeftLabel{\textsc{(ELSE)}}
	\UnaryInfC{$if \: v_1 \: then \: v_2 \: else \: M_3 \rightarrow \: 
	if \: v_1 \: then \: v_2 \: else \: M_3'$} 
\end{prooftree} 

\subsection*{Esercizio 1.4}
La valutazione, cioè l’esecuzione del programma, è deterministica. Se $M \rightarrow$ M' e
$M \rightarrow$ M'' allora M'=M''.\\
Dimostrare la proposizione precedente per induzione sulla struttura del termine M.

\subsubsection*{Dimostrazione}
Si dimostra per induzione sulla derivazione.

Casi base:
\begin{itemize}
	\item M=true \\
	\indent	Se true $\rightarrow$ M' e true $\rightarrow$ M'' $\implies$ M' = M''\\
	\indent Vero perchè true è un valore finale;
	
 \item M=false \\
	\indent	Se false $\rightarrow$ M' e false $\rightarrow$ M'' $\implies$ M' = M''\\
	\indent Vero perchè false è un valore finale;

\item M=fn x.M \\
	\indent	Se fn x.M $\rightarrow$ M' e fn x.M $\rightarrow$ M'' $\implies$ M' = M''\\
	\indent Vero perchè fn x.M è un valore finale;
	\item M=n \\
	\indent	Se n $\rightarrow$ M' e n $\rightarrow$ M'' $\implies$ M' = M''\\
	\indent Vero perchè n è un valore finale.
\end{itemize}
 
Passo induttivo:
\begin{itemize}
	\item Tesi: $M=M_1+M_2 \:\rightarrow\: M', \:M_1+M_2 \:\rightarrow$ M'' $\implies$ M'=M''\\
	\indent Per ipotesi induttiva: 
	\begin{itemize}
		\item Se $M_1 \:\rightarrow \: M_1'$ e $M_1 \:\rightarrow \: M_1''$ allora $M_1'=M_1''$;
		\item Se $M_2 \:\rightarrow \: M_2'$ e $M_2 \:\rightarrow \: M_2''$ allora $M_2'=M_2''$.
	\end{itemize} 
	3 Casi possibili:
	\begin{enumerate}[label=\alph*)]
		\item Derivazione tramite l'uso della regola (SUM-LEFT): 
		\begin{prooftree}
					\AxiomC{$M_1\:\rightarrow\:M_1'$}
					\LeftLabel{\textsc{(SUM-LEFT)}}
					\UnaryInfC{$M_1+M_2 \:\rightarrow \: M_1'+M_2=M'$}
				\end{prooftree} 
				\indent In questo caso $M_1'=M_1''$ per ipotesi 
						induttiva e quindi anche $M'=M''$;\\
				\indent La derivazione risulta deterministica applicando questa regola.
		\item Derivazione tramite l'uso della regola (SUM-RIGHT): 
		\begin{prooftree}
					\AxiomC{$M_2\:\rightarrow\:M_2'$}
					\LeftLabel{\textsc{(SUM-RIGHT)}}
					\UnaryInfC{$v_1+M_2 \:\rightarrow \: v_1+M_2'=M'$}
				\end{prooftree} 
				\indent In questo caso $M_2'=M_2''$ per ipotesi 
						induttiva e quindi anche $M'=M''$;\\
				\indent La derivazione risulta deterministica applicando questa regola.
		\item Derivazione tramite l'uso della regola (SUM): 
		\begin{prooftree}
					\AxiomC{con n'=$n_1+n_2$}
					\LeftLabel{\textsc{(SUM)}}
					\UnaryInfC{$n_1+n_2 \:\rightarrow \: n'$}
				\end{prooftree} 
				\indent Il risultato della somma è univoco e di conseguenza la derivazione è deterministica. 
	\end{enumerate}
	In tutti e 3 i casi, i soli possibili, la derivazione è deterministica, quindi la tesi 
	è dimostrata.\\
	
	
	\item Tesi: $M=M_1-M_2 \:\rightarrow\: M', \:M_1-M_2 \:\rightarrow$ M'' $\implies$ M'=M''\\
	\indent Per ipotesi induttiva: 
	\begin{itemize}
		\item Se $M_1 \:\rightarrow \: M_1'$ e $M_1 \:\rightarrow \: M_1''$ allora $M_1'=M_1''$;
		\item Se $M_2 \:\rightarrow \: M_2'$ e $M_2 \:\rightarrow \: M_2''$ allora $M_2'=M_2''$.
	\end{itemize} 
	3 Casi possibili:
	\begin{enumerate}[label=\alph*)]
		\item Derivazione tramite l'uso della regola (MINUS-LEFT): 
		\begin{prooftree}
					\AxiomC{$M_1\:\rightarrow\:M_1'$}
					\LeftLabel{\textsc{(MINUS-LEFT)}}
					\UnaryInfC{$M_1-M_2 \:\rightarrow \: M_1'-M_2=M'$}
				\end{prooftree} 
				\indent In questo caso $M_1'=M_1''$ per ipotesi 
						induttiva e quindi anche $M'=M''$;\\
				\indent La derivazione risulta deterministica applicando questa regola.
		\item Derivazione tramite l'uso della regola (MINUS-RIGHT): 
		\begin{prooftree}
					\AxiomC{$M_2\:\rightarrow\:M_2'$}
					\LeftLabel{\textsc{(MINUS-RIGHT)}}
					\UnaryInfC{$v_1-M_2 \:\rightarrow \: v_1-M_2'=M'$}
				\end{prooftree} 
				\indent In questo caso $M_2'=M_2''$ per ipotesi 
						induttiva e quindi anche $M'=M''$;\\
				\indent La derivazione risulta deterministica applicando questa regola.
		\item Derivazione tramite l'uso della regola (MINUS): 
		\begin{prooftree}
					\AxiomC{con n'=$n_1-n_2$}
					\LeftLabel{\textsc{(MINUS)}}
					\UnaryInfC{$n_1-n_2 \:\rightarrow \: n'$}
				\end{prooftree} 
				\indent Il risultato della differenza è univoco e di conseguenza la derivazione è deterministica. 
	\end{enumerate}
	In tutti e 3 i casi, i soli possibili, la derivazione è deterministica, quindi la tesi 
	è dimostrata.\\
	
	
	\item Tesi: $M=if\:M_1\:then\:M_2\:else\:M_3\:\rightarrow\: M',\:if\:M_1\:then\:M_2\:else\:M_3\:\rightarrow\: M''\:\implies$ M'=M''\\
	\indent Per ipotesi induttiva: 
	\begin{itemize}
		\item Se $M_1 \:\rightarrow \: M_1'$ e $M_1 \:\rightarrow \: M_1''$ allora $M_1'=M_1''$.
	\end{itemize}  
	3 Casi possibili:
	\begin{enumerate}[label=\alph*)]
		\item Derivazione tramite l'uso della regola (IF): 
		\begin{prooftree}
					\AxiomC{$M_1\:\rightarrow\:M_1'$}
					\LeftLabel{\textsc{(IF)}}
					\UnaryInfC{$if\:M_1\:then\:M_2\:else\:M_3\:\rightarrow\:if\:M_1'\:then\:M_2\:else\:M_3=M'$}
				\end{prooftree} 
				\indent In questo caso $M_1'=M_1''$ per ipotesi 
						induttiva e quindi anche $M'=M''$;\\
				\indent La derivazione risulta deterministica applicando questa regola.
		\item Derivazione tramite l'uso dell'assioma (IF-TRUE): 
		\begin{prooftree}
					\AxiomC{}
					\LeftLabel{\textsc{(IF-TRUE)}}
					\UnaryInfC{$if\:true\:then\:M_2\:else\:M_3\:\rightarrow\:M_2=M'$}
				\end{prooftree} 
				\indent In questo caso $M_2=M'=M''$ per applicazione dell'assioma e 
				la derivazione risulta deterministica.  
		\item Derivazione tramite l'uso della regola (IF-FALSE): 
		\begin{prooftree}
					\AxiomC{}
					\LeftLabel{\textsc{(IF-FAlSE)}}
					\UnaryInfC{$if\:false\:then\:M_2\:else\:M_3\:\rightarrow\:M_3=M'$}
				\end{prooftree} 
				\indent In questo caso $M_3=M'=M''$ per applicazione dell'assioma e 
				la derivazione risulta deterministica.  
	\end{enumerate} 
	In tutti e 3 i casi, i soli possibili, la derivazione è deterministica, quindi la tesi 
	è dimostrata.\\	
	
	
		\item Tesi: $M=M_1\:M_2 \:\rightarrow\: M', \:M_1\:M_2 \:\rightarrow$ M'' $\implies$ M'=M''\\
	\indent Per ipotesi induttiva: 
	\begin{itemize}
		\item Se $M_1 \:\rightarrow \: M_1'$ e $M_1 \:\rightarrow \: M_1''$ allora $M_1'=M_1''$;
		\item Se $M_2 \:\rightarrow \: M_2'$ e $M_2 \:\rightarrow \: M_2''$ allora $M_2'=M_2''$.
	\end{itemize}  
	2 Casi possibili:
	\begin{enumerate}[label=\alph*)]
		\item Derivazione tramite l'uso della regola (APP1): 
		\begin{prooftree}
					\AxiomC{$M_1\:\rightarrow\:M_1'$}
					\LeftLabel{\textsc{(APP1)}}
					\UnaryInfC{$M_1\:M_2 \:\rightarrow \: M_1'\:M_2=M'$}
				\end{prooftree} 
				\indent In questo caso $M_1'=M_1''$ per ipotesi 
						induttiva e quindi anche $M'=M''$;\\
				\indent La derivazione risulta deterministica applicando questa regola.
		\item Derivazione tramite l'uso della regola (APP2): 
		\begin{prooftree}
					\AxiomC{$M_2\:\rightarrow\:M_2'$}
					\LeftLabel{\textsc{(APP2)}}
					\UnaryInfC{$v_1\:M_2 \:\rightarrow \: v_1\:M_2'=M'$}
				\end{prooftree} 
				\indent In questo caso $M_2'=M_2''$ per ipotesi 
						induttiva e quindi anche $M'=M''$;\\
				\indent La derivazione risulta deterministica applicando questa regola. 
	\end{enumerate} 
	In tutti e 2 i casi, i soli possibili, la derivazione è deterministica, quindi la tesi 
	è dimostrata.\\	
	
	
		\item Tesi: $M=fn\:x.M\:v\:\rightarrow \:M\{x:=v\}=M'$, $fn\:x.M\:v\:\rightarrow \:M\{x:=v\}=M''\implies$ M'=M''\\  
 	Derivazione tramite l'uso dell'assioma (BETA): 
		\begin{prooftree}
					\AxiomC{}
					\LeftLabel{\textsc{(BETA)}}
					\UnaryInfC{$fn\:x.M\:v\:\rightarrow \:M\{x:=v\}=M'$}
				\end{prooftree} 
				\indent Tramite l'applicazione dell'assioma si ottiene M{x:=v}=M'=M''.\\
				\indent La derivazione risulta quindi deterministica e la tesi risulta dimostrata.\\	
\end{itemize}

La derivazione è determinstica all'applicazione di tutte le regole e gli assiomi del linguaggio, di conseguenza è deterministica l'esecuzione del programma.\\ 
\begin{flushright}
$\Box$
\end{flushright}

\subsection*{Esercizio 1.5}
Descrivere la valutazione del termine ((fn x.3) (fn y.y)) ((fn z.if z then 1 else 0) (false)).
Modificare le regole di valutazione in modo tale che, mantenendo una strategia call-by-value, il termine precedente evolva in un termine stuck in meno passi di riduzione. Scrivere le regole di valutazione della strategia call-by-name e valutare il termine precedente secondo questa strategia.\\

\subsubsection*{Svolgimento}
Valutazione tramite strategia call-by-value: \\
\textbf{passo 1}
\begin{prooftree} 
	\AxiomC{}
	\LeftLabel{\textsc{(BETA)}}
	\UnaryInfC{$(\fn x.3)(\fn y.y)\: \rightarrow \:3$}
	\LeftLabel{\textsc{(APP1)}}
	\UnaryInfC{$((\fn x.3)(\fn y.y))\:((\fn z.if\:z\:then\:1\:else\:0)(false))\: \rightarrow 
	\:3\:((\fn z.if\:z\:then\:1\:else\:0)(false))$}
\end{prooftree}
\textbf{passo 2}
\begin{prooftree} 
	\AxiomC{}
	\LeftLabel{\textsc{(BETA)}}
	\UnaryInfC{$(\fn z.if\:z\:then\:1\:else\:0)(false)\: \rightarrow \:if\:false\:then\:1\:else\:0$}
	\LeftLabel{\textsc{(APP2)}}
	\UnaryInfC{$3\:((\fn z.if\:z\:then\:1\:else\:0)(false))\: \rightarrow 
	\:3\:(if\:false\:then\:1\:else\:0)$}
\end{prooftree}
\textbf{passo 3}
\begin{prooftree} 
	\AxiomC{}
	\LeftLabel{\textsc{(IF-FALSE)}}
	\UnaryInfC{$if\:false\:then\:1\:else\:0\: \rightarrow \:0$}
	\LeftLabel{\textsc{(APP2)}}
	\UnaryInfC{$3\:(if\:false\:then\:1\:else\:0)\: \rightarrow 
	\:3\:0$}
\end{prooftree}

3 0 costituisce uno stuck: non esistono regole o assiomi per continuare l'esecuzione di tale programma. Esecuzione terminata dopo 3 passi di derivazione. \\
Si può mantenere la strategia CBV e terminare l'esecuzione in un numero minore di passi modificando (APP2):
 
\begin{prooftree}
	\AxiomC{$N\:\rightarrow\:N'$}
	\LeftLabel{\textsc{(APP2)}}
	\UnaryInfC{$(\fn x.M)\: N\:\rightarrow\:\fn x.M\:N'$} 
\end{prooftree}

In questo modo l'esecuzione del programma terminerebbe dopo un solo passo. Infatti, ci si ritroverebbe in una forma in cui non è possibile applicare alcuna regola o assioma (3 non è una funzione).\\

Con una strategia CBN (Call-by-name) non ho (APP2) e la funzione (BETA) è la seguente:

\begin{prooftree}
	\AxiomC{$\checkmark$}
	\LeftLabel{\textsc{(BETA)}} 
	\UnaryInfC{$(\fn x.M)\: N\: \rightarrow \:M\{x:=N\}$}
\end{prooftree}

Con questa regola mostriamo l'esecuzione del programma:\\

\textbf{passo 1}
\begin{prooftree} 
	\AxiomC{}
	\LeftLabel{\textsc{(BETA)}}
	\UnaryInfC{$(\fn x.3)(\fn y.y)\: \rightarrow \:3$}
	\LeftLabel{\textsc{(APP1)}}
	\UnaryInfC{$((\fn x.3)(\fn y.y))\:((\fn z.if\:z\:then\:1\:else\:0)(false))\: \rightarrow 
	\:3\:((\fn z.if\:z\:then\:1\:else\:0)(false))$}
\end{prooftree}

Stuck! Non esistono regole o assiomi applicabili: APP1 non è applicabile in quanto 3 non è funzione.



\subsection*{Esercizio 1.6}
Consideriamo le seguenti definizioni in Scala:\\
\begin{lstlisting}
def square(x:Int):Int = x*x  
def sumOfSquare(x:Int,y:Int):Int = square(x)+square(y)
\end{lstlisting}
Descrivere i passi di riduzione dell'espressione \textbf{sumOfSquare(3,4)} secondo una strategia call-by-value analoga a quella definita nella sezione precedente.\\ 
Descrivere inoltre la riduzione della stesa espressione secondo la strategia call-by-name. \\
Descrivere i passi di riduzione dell'espressione \textbf{sumOfSquare(3,2+2)} secondo le strategie call-by-value e call-by-name.

\subsubsection*{Svolgimento}
Caso \textbf{sumOfSquare(3,4)}.\\
CBV si comporta come CBN.\\

\textbf{passo 1}
\begin{prooftree} 
	\AxiomC{}
	\LeftLabel{\textsc{(BETA)}}
	\UnaryInfC{sumOfSquare(3, 4) $\rightarrow$ square(3) + square(4)} 
\end{prooftree}

\textbf{passo 2}
\begin{prooftree} 
	\AxiomC{}
	\LeftLabel{\textsc{(BETA)}}
	\UnaryInfC{square(3) $\rightarrow$ 3*3} 
	\LeftLabel{\textsc{(SUM-L)}}
	\UnaryInfC{square(3) + square(4) $\rightarrow$ 3*3 + square(4)} 
\end{prooftree}

\textbf{passo 3}
\begin{prooftree} 
	\AxiomC{}
	\LeftLabel{\textsc{(MULT)}}
	\UnaryInfC{3*3 $\rightarrow$ 9} 
	\LeftLabel{\textsc{(SUM-L)}}
	\UnaryInfC{3*3 + square(4) $\rightarrow$ 9 + square(4)} 
\end{prooftree}

\textbf{passo 4}
\begin{prooftree} 
	\AxiomC{}
	\LeftLabel{\textsc{(BETA)}}
	\UnaryInfC{square(4) $\rightarrow$ 4*4} 
	\LeftLabel{\textsc{(SUM-R)}}
	\UnaryInfC{9 + square(4) $\rightarrow$ 9 + 4*4} 
\end{prooftree}

\textbf{passo 5}
\begin{prooftree} 
	\AxiomC{}
	\LeftLabel{\textsc{(MULT)}}
	\UnaryInfC{4*4 $\rightarrow$ 16} 
	\LeftLabel{\textsc{(SUM-R)}}
	\UnaryInfC{9 + 4*4 $\rightarrow$ 9 + 16} 
\end{prooftree}

\textbf{passo 6}
\begin{prooftree} 
	\AxiomC{} 
	\LeftLabel{\textsc{(SUM)}}
	\UnaryInfC{9 + 16 $\rightarrow$ 25} 
\end{prooftree}

Caso \textbf{sumOfSquare(3,2+2)}\\
CBV:\\
\textbf{passo 1}
\begin{prooftree} 
	\AxiomC{}
	\LeftLabel{\textsc{(SUM)}}
	\UnaryInfC{2+2 $\rightarrow$ 4} 
	\LeftLabel{\textsc{(APP2)}}
	\UnaryInfC{sumOfSquare(3, 2+2) $\rightarrow$ sumOfSquare(3, 4)} 
\end{prooftree}

I passi successivi sono gli stessi del caso precedente, quindi i passi di esecuzione sono 7.\\

CBN:\\
\textbf{passo 1}
\begin{prooftree} 
	\AxiomC{} 
	\LeftLabel{\textsc{(BETA)}}
	\UnaryInfC{sumOfSquare(3, 2+2) $\rightarrow$ square(3) + square(2+2)} 
\end{prooftree}

\textbf{passo 2}
\begin{prooftree} 
	\AxiomC{} 
	\LeftLabel{\textsc{(BETA)}}
	\UnaryInfC{square(3) $\rightarrow$ 3*3}
	\LeftLabel{\textsc{(SUM-L)}}
	\UnaryInfC{square(3) + square(2+2) $\rightarrow$ 3*3 + square(2+2)} 
\end{prooftree}

\textbf{passo 3}
\begin{prooftree} 
	\AxiomC{} 
	\LeftLabel{\textsc{(MULT)}}
	\UnaryInfC{3*3 $\rightarrow$ 9}
	\LeftLabel{\textsc{(SUM-L)}}
	\UnaryInfC{3*3 + square(2+2) $\rightarrow$ 9 + square(2+2)} 
\end{prooftree}

\textbf{passo 4}
\begin{prooftree} 
	\AxiomC{} 
	\LeftLabel{\textsc{(BETA)}}
	\UnaryInfC{square(2+2) $\rightarrow$ (2+2)*(2+2)}
	\LeftLabel{\textsc{(SUM-R)}}
	\UnaryInfC{9 + square(2+2) $\rightarrow$ 9 + (2+2)*(2+2)} 
\end{prooftree}

\textbf{passo 5}
\begin{prooftree} 
	\AxiomC{} 
	\LeftLabel{\textsc{(SUM)}}
	\UnaryInfC{2+2 $\rightarrow$ 4}
	\LeftLabel{\textsc{(MULT-L)}}
	\UnaryInfC{(2+2)*(2+2) $\rightarrow$ (4)*(2+2)}
	\LeftLabel{\textsc{(SUM-R)}}
	\UnaryInfC{9 + (2+2)*(2+2) $\rightarrow$ 9 + (4)*(2+2)} 
\end{prooftree}

\textbf{passo 6}
\begin{prooftree} 
	\AxiomC{} 
	\LeftLabel{\textsc{(SUM)}}
	\UnaryInfC{2+2 $\rightarrow$ 4}
	\LeftLabel{\textsc{(MULT-R)}}
	\UnaryInfC{4*(2+2) $\rightarrow$ 4*4}
	\LeftLabel{\textsc{(SUM-R)}}
	\UnaryInfC{9 + 4*(2+2) $\rightarrow$ 9 + 4*4} 
\end{prooftree}

\textbf{passo 7}
\begin{prooftree} 
	\AxiomC{}  
	\LeftLabel{\textsc{(MULT)}}
	\UnaryInfC{4*4 $\rightarrow$ 16}
	\LeftLabel{\textsc{(SUM-R)}}
	\UnaryInfC{9 + 4*4 $\rightarrow$ 9 + 16} 
\end{prooftree}

\textbf{passo 8}
\begin{prooftree} 
	\AxiomC{}   
	\LeftLabel{\textsc{(SUM)}}
	\UnaryInfC{9 + 16 $\rightarrow$ 25} 
\end{prooftree}

I passi di esecuzione sono 8 in questo caso (CBN).

\subsection*{Esercizio 1.7}
Si consideri la seguente definizione in Scala:

\begin{lstlisting}
def test(x:Int, y:Int):Int = x*x
\end{lstlisting}
confrontare la velocit\`a (cio\`e il numero di passi) di riduzione delle seguenti espressioni secondo le strategie CBV e CBN, indicando quale delle due \`e pi\`u veloce. 
\begin{enumerate}
	\item test(2,3)
	\item test(3+4,8)
	\item test(7,2*4)
	\item test(3+4,2*4)
\end{enumerate}

\subsubsection*{Svolgimento}
\begin{enumerate}
	\item CBV=CBN: entrambi terminano l'esecuzione in 2 passi; 
	\item CBV<CBN: CBV termina prima perch\`e valuta l'espressione 3+4 subito (una sola volta). CBN invece invoca prima la funzione, sostituendo ad x l'espressione. Essendo x presente 2 volte nella funzione, la valutazione viene effettuata 2 volte. 
	\item CBV>CBN: CBN termina prima perch\`e non valuta l'espressione 2*4 che non \`e utilizzata nella funzione test.
	\item CBV=CBN: entrambi terminano l'esecuzione in 4 passi; la differenza sta nel fatto che con CBV si svolgono prima le due operazioni (somma e moltiplicazione) e poi si chiama la funzione. Con CBN invece, si evita di calcolare 2*4, tuttavia si calcola 3+4 due volte.
\end{enumerate}



\subsection*{Esercizio 1.8}
Definire in Scala una funzione and che si comporta come il costrutto logico $x\&\&y$. 

\subsubsection*{Svolgimento}

\begin{lstlisting}
// and logico. b1 e b2 passati by-name 
def and(b1:=> Boolean, b2:=> Boolean): Boolean =
	if(!b1) false
	else if(b2) true
		 else false; 
\end{lstlisting}

Il parametro b1 viene passato by-name nella nostra soluzione. Tuttavia, esso pu\`o essere passato anche by-value visto che in ogni caso viene valutato. 










