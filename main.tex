\documentclass[12pt]{article}
\usepackage[english]{babel}
\usepackage{natbib}
\usepackage{url}
\usepackage[utf8x]{inputenc}
\usepackage{amsmath}
\usepackage{graphicx}
\graphicspath{{images/}}
\usepackage{parskip}
\usepackage{fancyhdr}
\usepackage{vmargin}

\setmarginsrb{1.5 cm}{2.5 cm}{1.5 cm}{2.5 cm}{1 cm}{1.5 cm}{1 cm}{1.5 cm}

\usepackage{eurosym}                %simbolo dell'euro
\usepackage{listings}
\usepackage[usenames,dvipsnames,svgnames,table]{xcolor}
\input{listings}

%\tightlist per compatibilità con pandoc
\providecommand{\tightlist}{%
	\setlength{\itemsep}{0pt}\setlength{\parskip}{0pt}}


\usepackage[labelfont=bf]{caption}

\usepackage[protrusion=true,expansion=true]{microtype} % Better typography
\usepackage{graphicx} % Required for including pictures
\usepackage{wrapfig} % Allows in-line images
 

\usepackage{subfig}
\usepackage{hyperref}
\usepackage{placeins}
\usepackage{sourcecodepro}
\usepackage{hyperref}                   % collegamenti ipertestuali

\usepackage[colorinlistoftodos,prependcaption]{todonotes} %todo

\usepackage{amssymb,amsmath,amsthm,amsfonts}
\usepackage{mathtools}


\usepackage{float}
\usepackage{algorithm}
\usepackage{algpseudocode} % https://en.wikibooks.org/wiki/LaTeX/Algorithms#Typesetting_using_the_algorithmicx_package
\usepackage{amsmath} 
\usepackage{amssymb}  %$\mathbb{N}$ per il simbolo dei numeri naturali 

\usepackage{enumerate} % permette di personalizzare enumerate
\usepackage{enumitem}
\usepackage{xmpincl}	%Aggiunge metadati sulla licenza CC
\usepackage{xspace}

\usepackage{bussproofs}

\usepackage{listings}
\usepackage{color}

\definecolor{dkgreen}{rgb}{0,0.6,0}
\definecolor{gray}{rgb}{0.5,0.5,0.5}
\definecolor{mauve}{rgb}{0.58,0,0.82}

\lstset{frame=tb,
  language=Java,
  aboveskip=3mm,
  belowskip=3mm,
  showstringspaces=false,
  columns=flexible,
  basicstyle={\small\ttfamily},
  numbers=none,
  numberstyle=\tiny\color{gray},
  keywordstyle=\color{blue},
  commentstyle=\color{dkgreen},
  stringstyle=\color{mauve},
  breaklines=true,
  breakatwhitespace=true,
  tabsize=3
}

\hypersetup{
    colorlinks=false,
    pdfborder={0 0 0},
}
   
\makeatletter
\renewcommand\@biblabel[1]{\textbf{#1.}} % Change the square brackets for each bibliography item from '[1]' to '1.'
\renewcommand{\@listI}{\itemsep=0pt} % Reduce the space between items in the itemize and enumerate environments and the bibliography

\renewcommand{\maketitle}{ % Customize the title - do not edit title and author name here, see the TITLE block below
	\begin{flushright} % Right align
		{\LARGE\@title} % Increase the font size of the title
		
		\vspace{50pt} % Some vertical space between the title and author name
		
		{\large\@author} % Author name
		\\\@date % Date
		
		\vspace{100pt} % Some vertical space between the author block and abstract
	\end{flushright}
}

%% breakablealgorithm http://tex.stackexchange.com/questions/33866/algorithm-tag-and-page-break
\makeatletter
\newenvironment{breakablealgorithm}
{% \begin{breakablealgorithm}
	\begin{center}
		\refstepcounter{algorithm}% New algorithm
		\hrule height.8pt depth0pt \kern2pt% \@fs@pre for \@fs@ruled
		\renewcommand{\caption}[2][\relax]{% Make a new \caption
			{\raggedright\textbf{\ALG@name~\thealgorithm} ##2\par}%
			\ifx\relax##1\relax % #1 is \relax
			\addcontentsline{loa}{algorithm}{\protect\numberline{\thealgorithm}##2}%
			\else % #1 is not \relax
			\addcontentsline{loa}{algorithm}{\protect\numberline{\thealgorithm}##1}%
			\fi
			\kern2pt\hrule\kern2pt
		}
	}{% \end{breakablealgorithm}
	\kern2pt\hrule\relax% \@fs@post for \@fs@ruled
\end{center}
}
\makeatother

\makeatletter % trattino con punto sopra
\newcommand{\dotminus}{\mathbin{\text{\@dotminus}}}

\newcommand{\@dotminus}{%
	\ooalign{\hidewidth\raise1ex\hbox{.}\hidewidth\cr$\m@th-$\cr}%
}
\makeatother

\DeclarePairedDelimiter{\ceil}{\lceil}{\rceil}
\DeclarePairedDelimiter{\floor}{\lfloor}{\rfloor}


\let\oldtext\text
\renewcommand{\text}[1]{\oldtext{\normalfont\sffamily #1}}
\newcommand{\false}{\text{ false }}
\newcommand{\true}{\text{ true }}
\newcommand{\fn}{\text{fn }}
\newcommand{\iif}{\text{ if }}
\newcommand{\then}{\text{ then }}
\newcommand{\eelse}{\text{ else }}
\newcommand{\Nat}{\text{ Nat }}
\newcommand{\Bool}{\text{ Bool }}


\newcommand{\IF}[3]{\ensuremath{\text{if}\ #1\ \text{then}\ #2\ \text{else}\ #3}}

\newcommand{\vbar}{\:|\:}
\newcommand{\proj}{.\_}

\newcommand{\throw}[1]{\ensuremath{\text{throw} \: #1}}
\newcommand{\trycatch}[2]{\ensuremath{\text{try} \: #1 \: \text{catch} \: #2}}
\newcommand{\TExc}{\ensuremath{T_{exc}}}
\newcommand{\myrule}[1]{\textsc{(#1)}}

\newcommand{\class}[3]{\text{class} \: #1 \: \text{extends} \: #2 \: \{\: #3 \:\}}
\newcommand{\construct}[3]{\ensuremath{#1(#2)\{\: #3\:\}}}
\newcommand{\this}{\text{this}}
\newcommand{\super}{\text{super}}

\newcommand{\method}[4]{\ensuremath{#1 \: #2(#3) \: \{ \: \text{return} \: #4;\: \} }}
\newcommand{\new}[2]{\ensuremath{\text{new}\: #1(  #2  )}}

\newcommand{\stype}{\ensuremath{<:}}
\newcommand{\fields}[1]{fields(#1)}
\newcommand{\mbody}[1]{mbody(#1)}
\newcommand{\mtype}[1]{mtype(#1)}
\newcommand{\IN}[1]{\text{in} \: #1}


% Shorcut per prooftree
\newcommand{\LL}[1]{\LeftLabel{\myrule{#1}}}
\newcommand{\AC}[1]{\AxiomC{#1}}
\newenvironment{bprooftree}
{\leavevmode\hbox\bgroup}
{\DisplayProof\egroup}

\title{Esercizi - Aspetti avanzati di Ling. di prog.}                                % Title
\author{Stefano Campese, Luca Costantino ed Enrico Savoca}                               % Author
\date{3 Dec 2016}                                         % Date

\makeatletter
\let\thetitle\@title
\let\theauthor\@author
\let\thedate\@date
\makeatother

\pagestyle{fancy}
\fancyhf{}
\rhead{\theauthor}
\lhead{\thetitle}
\cfoot{\thepage}

\begin{document}

%%%%%%%%%%%%%%%%%%%%%%%%%%%%%%%%%%%%%%%%%%%%%%%%%%%%%%%%%%%%%%%%%%%%%%%%%%%%%%%%%%%%%%%%%

\begin{titlepage}
    \centering
    \vspace*{0.5 cm}
    \includegraphics[scale = 0.75]{logo.png}\\[1.0 cm]  % University Logo
    \textsc{\LARGE Aspetti Avanzati dei Linguaggi \newline\newline di Programmazione}\\[2.0 cm]  % University Name
    %\textsc{\Large CSX-325}\\[0.5 cm]               % Course Code
    %\rule{\linewidth}{0.2 mm} \\[0.4 cm]
    { \huge \bfseries \thetitle}\\
    \rule{\linewidth}{0.2 mm} \\[1.5 cm]
    
    \begin{minipage}{0.4\textwidth}
        \begin{flushleft} \large
            \emph{Autori:}\\
            Stefano Campese\\
            Luca Costantino\\
            Enrico Savoca\\
        \end{flushleft}
    \end{minipage}~
    
    
    
    
    
    
    
\end{titlepage}

Il documento è stato realizzato da Stefano Campese, Luca Costantino ed Enrico Savoca. Lo scopo di esso è quello di raccogliere in un unico posto le soluzioni di tutti gli esercizi del corso di "Aspetti avanzati dei linguaggi di programmazione". Gli esercizi sono stati svolti da tutti gli autori del documento e la versione ritenuta corretta di ognuno di essi è stata riportata nel presente testo.

%%%%%%%%%%%%%%%%%%%%%%%%%%%%%%%%%%%%%%%%%%%%%%%%%%%%%%%%%%%%%%%%%%%%%%%%%%%%%%%%%%%%%%%%% 

\tableofcontents 
\pagebreak

%%%%%%%%%%%%%%%%%%%%%%%%%%%%%%%%%%%%%%%%%%%%%%%%%%%%%%%%%%%%%%%%%%%%%%%%%%%%%%%%%%%%%%%%%
  
\section{Introduzione}
Il documento è stato realizzato da Stefano Campese ed Enrico Savoca. Lo scopo di esso è quello di raccogliere in un unico posto le soluzioni di tutti gli esercizi del corso di "Aspetti avanzati dei linguaggi di programmazione". Gli esercizi sono stati svolti da entrambi gli autori e la versione ritenuta corretta di ognuno di essi è stata riportata nel presente testo.
\subsection{Il mini linguaggio funzionale (note 3)}
\subsubsection*{Esempi di derivazione svolti}
 
\begin{enumerate}[label=\alph*)]
	\item $\emptyset\:\vdash$  if true then 5 + 7 else 2 :Nat
	\item $\emptyset\:\vdash$  ($\fn$ x:T.x) x :T$\rightarrow$T
	\item $\emptyset\:\vdash$  ($\fn$ x:Bool.x) true :Bool
	\item f:Bool$\rightarrow$ Bool $\vdash$ f (if false then true else false) :Bool
	\item f:Bool$\rightarrow$ Bool $\vdash\:\fn$ x:Bool.f(if x then false else x) :Bool$\rightarrow$Bool
\end{enumerate}

\subparagraph{Svolgimento}

\begin{enumerate}[label=\alph*)]
	\item 	
		\begin{prooftree}
			\AxiomC{$\checkmark$}
			\LeftLabel{\textsc{(TRUE)}}
			\UnaryInfC{$\emptyset\:\vdash$ true: Bool}
			\AxiomC{$\checkmark$}
			\LeftLabel{\textsc{(NAT)}}
			\UnaryInfC{$\emptyset\:\vdash$ 5 :Nat}
			\AxiomC{$\checkmark$}
			\LeftLabel{\textsc{(NAT)}}
			\UnaryInfC{$\emptyset\:\vdash$ 7 :Nat}
			\LeftLabel{\textsc{(SUM)}}
			\BinaryInfC{$\emptyset\:\vdash$ 5 + 7 :Nat}
			\AxiomC{$\checkmark$}
			\LeftLabel{\textsc{(NAT)}}
			\UnaryInfC{$\emptyset\:\vdash$  2 :Nat}
			\LeftLabel{\textsc{(IF-THEN-ELSE)}}
			\TrinaryInfC{$\emptyset\:\vdash$  if true then 5 + 7 else 2 :Nat}
		\end{prooftree} 


	\item  
		\begin{prooftree} 
			\AxiomC{x:T $\in$ x:T}
			\LeftLabel{\textsc{(VAR)}}
			\UnaryInfC{x:T $\vdash$  x :T}
			\LeftLabel{\textsc{(FUN)}}
			\UnaryInfC{$\emptyset\:\vdash$  ($\fn$ x:T.x) x :T$\rightarrow$T}
		\end{prooftree} 
	 
	 
	\item 
		\begin{prooftree} 
			\AxiomC{x:Bool $\in$ x:Bool}
			\LeftLabel{\textsc{(VAR)}}
			\UnaryInfC{x:Bool $\vdash$  x :Bool}
			\LeftLabel{\textsc{(FUN)}}
		    \UnaryInfC{$\emptyset\:\vdash\:\fn$ x:Bool.x :Bool$/rightarrow					$Bool}
			\AxiomC{$\checkmark$}
			\LeftLabel{\textsc{(TRUE)}}
			\UnaryInfC{$\emptyset\:\vdash$ true :Bool}
			\LeftLabel{\textsc{(APP)}}
			\BinaryInfC{$\emptyset\:\vdash$  ($\fn$ x:Bool.x) true :Bool}
		\end{prooftree} 
		
		
	\item 
		Sia $\Gamma$ = f:Bool$\rightarrow$Bool \\
		\begin{prooftree}  
			\AxiomC{f:Bool$\rightarrow$ Bool $\in$ $\Gamma$}
			\LeftLabel{\textsc{(VAR)}}
			\UnaryInfC{$\Gamma$ $\vdash$ f :Bool $\rightarrow$ Bool} 
			\AxiomC{$\checkmark$}
			\LeftLabel{\textsc{(FALSE)}}
			\UnaryInfC{$\Gamma$ $\vdash$ false :Bool}
			\AxiomC{$\checkmark$}
			\LeftLabel{\textsc{(TRUE)}}
			\UnaryInfC{$\Gamma$ $\vdash$ true :Bool}
			\AxiomC{$\checkmark$}
			\LeftLabel{\textsc{(FALSE)}}
			\UnaryInfC{$\Gamma$ $\vdash$ false :Bool}
			\LeftLabel{\textsc{(IF-THEN-ELSE)}}
			\TrinaryInfC{$\Gamma$ $\vdash$ if false then true else false :Bool}
			\LeftLabel{\textsc{(APP)}}
			\BinaryInfC{$\Gamma$ $\vdash$ f (if false then true else 							false):Bool}
		\end{prooftree} 
	
	
	\item 
		Sia $\Gamma$ = f:Bool$\rightarrow$Bool \\
		\begin{prooftree}  
			\AxiomC{f:Bool $\rightarrow$ Bool $\in$ $\Gamma$, x:Bool}
			\LeftLabel{\textsc{(VAR)}}
			\UnaryInfC{$\Gamma$, x:Bool $\vdash$ f: Bool $\rightarrow$ Bool} 
			\LeftLabel{\textsc{(FUN)}}
			\UnaryInfC{$\Gamma$ $\vdash\:\fn$ x:Bool.f: ? $\rightarrow$ Bool 
			$\rightarrow$ Bool} 
			\AxiomC{x:Bool $\in$ $\Gamma$}
			\LeftLabel{\textsc{(VAR)}}
			\UnaryInfC{$\Gamma$ $\vdash$ x :?}
			\AxiomC{$\checkmark$}
			\LeftLabel{\textsc{(TRUE)}}
			\UnaryInfC{$\Gamma$ $\vdash$ true :Bool}
			\AxiomC{$\checkmark$}
			\LeftLabel{\textsc{(FALSE)}}
			\UnaryInfC{$\Gamma$ $\vdash$ false :Bool}
			\LeftLabel{\textsc{(IF-THEN-ELSE)}}
			\TrinaryInfC{$\Gamma$ $\vdash$ (if x then false else x):? 
			$\rightarrow$ Bool}
			\LeftLabel{\textsc{(APP)}}
			\BinaryInfC{$\Gamma\:\vdash\:\fn$ x:Bool.f(if x then false 
			else x)
			 :Bool$\rightarrow$Bool}
		\end{prooftree} 
		Quindi ?=Bool.	
\end{enumerate}
 

\subsubsection*{Esercizio 2.1}
Trovare un contesto $\Gamma$ tale che $\Gamma$' f x y : Bool sia derivabile.

\subparagraph{Svolgimento}
 	\begin{prooftree} 
		\AxiomC{f:$T_2\:\rightarrow\:T_1\:\rightarrow$ Bool $\in\:\Gamma$}
		\LeftLabel{\textsc{(VAR)}}
		\UnaryInfC{$\Gamma\:\vdash$ f :$T_2\:\rightarrow\:T_1\:\rightarrow$
		 Bool}
		\AxiomC{x:$T_2\:\in$ x:$T_2$}
		\LeftLabel{\textsc{(VAR)}}
		\UnaryInfC{$\Gamma\:\vdash$ x :$T_2$}
		\LeftLabel{\textsc{(APP)}}
		\BinaryInfC{$\Gamma\:\vdash$ f x :$T_1\:\rightarrow$ Bool}
		\AxiomC{y :$T_1$ $\in$ $\Gamma$}
		\LeftLabel{\textsc{(VAR)}}
		\UnaryInfC{$\Gamma\:\vdash$ y :$T_1$}
		\LeftLabel{\textsc{(APP)}}
		\BinaryInfC{$\Gamma\:\vdash$ f x y :Bool}
	\end{prooftree} 
	
	Esiste un contesto per questo programma, in quanto il programma si puo' ottenere applicando le regole e costruendo l'albero di derivazione. \\
	Fanno parte del contesto le seguenti regole di tipo:
\begin{itemize}
	\item f:$T_2\:\rightarrow\:T_1\:\rightarrow$ Bool
	\item x :$T_2$
	\item y :$T_1$
\end{itemize}

Caso f (x y):\\
 	\begin{prooftree} 
		\AxiomC{f :$T_1\:\rightarrow$ Bool $\:\in\:\Gamma$}
		\LeftLabel{\textsc{(VAR)}}
		\UnaryInfC{$\Gamma\:\vdash$ f :$T_1\:\rightarrow$ Bool}
		\AxiomC{x:$T_2\:\rightarrow\:T_1\:\in\:\Gamma$}
		\LeftLabel{\textsc{(VAR)}}
		\UnaryInfC{$\Gamma\:\vdash$ x :$T_2$}
		\AxiomC{y:$T_2\:\in\:\Gamma$}
		\LeftLabel{\textsc{(VAR)}}
		\UnaryInfC{$\Gamma\:\vdash$ y :$T_2$}
		\LeftLabel{\textsc{(APP)}}
		\BinaryInfC{$\Gamma\:\vdash$ x y :$T_1$}
		\LeftLabel{\textsc{(APP)}}
		\BinaryInfC{$\Gamma\:\vdash$ f (x y) :Bool}
	\end{prooftree} 
	
	Esiste un contesto anche per questo programma. \\
	Fanno parte del contesto le seguenti regole di tipo:
\begin{itemize}
	\item f:$T_1\:\rightarrow\:$ Bool
	\item x :$T_2\:\rightarrow\:T_1$
	\item y :$T_2$
\end{itemize}


\subsubsection*{Esercizio 2.2}
Il giudizio $\Gamma$' x x : T e' derivabile? Se si', trovare una derivazione per qualche $\Gamma$, T, altrimenti provare che non e' derivabile.

\subparagraph{Svolgimento}
 	\begin{prooftree} 
		\AxiomC{$\Gamma\:\vdash$ x :$T_1\:\rightarrow\:$T}
		\AxiomC{$\Gamma\:\vdash$ x :$T_1$}
		\LeftLabel{\textsc{(APP)}}
		\BinaryInfC{$\Gamma\:\vdash$ x x :T}
	\end{prooftree} 

No, non e' derivabile. Sarebbe necessario avere un sistema di tipi che ammetta il tipo ricorsivo. \\



\section{I tipi semplici (note 3)}
\subsection*{Esempi di derivazione svolti}
 
\begin{enumerate}[label=\alph*)]
	\item $\emptyset\:\vdash$  if true then 5 + 7 else 2 :Nat
	\item $\emptyset\:\vdash$  $\fn$ x:T.x :T$\rightarrow$T
	\item $\emptyset\:\vdash$  ($\fn$ x:Bool.x) true :Bool
	\item f:Bool$\rightarrow$ Bool $\vdash$ f (if false then true else false) :Bool
	\item f:Bool$\rightarrow$ Bool $\vdash\:\fn$ x:Bool.f(if x then false else x) :Bool$\rightarrow$Bool
\end{enumerate}

\subsubsection*{Svolgimento}

\begin{enumerate}[label=\alph*), leftmargin=*]
	\item $\emptyset\:\vdash$  if true then 5 + 7 else 2 :Nat\\
		\scalebox{.75}{
		\parbox{1cm}{
    	\begin{prooftree}
			\AxiomC{$\checkmark$}
			\LeftLabel{\textsc{(TRUE)}}
			\UnaryInfC{$\emptyset\:\vdash$ true: Bool}
			\AxiomC{$\checkmark$}
			\LeftLabel{\textsc{(NAT)}}
			\UnaryInfC{$\emptyset\:\vdash$ 5 :Nat}
			\AxiomC{$\checkmark$}
			\LeftLabel{\textsc{(NAT)}}
			\UnaryInfC{$\emptyset\:\vdash$ 7 :Nat}
			\LeftLabel{\textsc{(SUM)}}
			\BinaryInfC{$\emptyset\:\vdash$ 5 + 7 :Nat}
			\AxiomC{$\checkmark$}
			\LeftLabel{\textsc{(NAT)}}
			\UnaryInfC{$\emptyset\:\vdash$  2 :Nat}
			\LeftLabel{\textsc{(IF-THEN-ELSE)}}
			\TrinaryInfC{$\emptyset\:\vdash$  if true then 5 + 7 else 2 :Nat}
		\end{prooftree}}}

    \vspace*{1 cm}

	\item $\emptyset\:\vdash$  $\fn$ x:T.x :T$\rightarrow$T  \\
		\scalebox{.75}{
		\parbox{1cm}{
		\begin{prooftree} 
			\AxiomC{x:T $\in$ x:T}
			\LeftLabel{\textsc{(VAR)}}
			\UnaryInfC{x:T $\vdash$  x :T}
			\LeftLabel{\textsc{(FUN)}}
			\UnaryInfC{$\emptyset\:\vdash$  $\fn$ x:T.x :T$\rightarrow$T}
		\end{prooftree}}
		}
	 
	 
    \vspace*{1 cm}
	\item $\emptyset\:\vdash$  ($\fn$ x:Bool.x) true :Bool\\
		\scalebox{.75}{
		\parbox{1cm}{
		\begin{prooftree} 
			\AxiomC{x:Bool $\in$ x:Bool}
			\LeftLabel{\textsc{(VAR)}}
			\UnaryInfC{x:Bool $\vdash$  x :Bool}
			\LeftLabel{\textsc{(FUN)}}
		    \UnaryInfC{$\emptyset\:\vdash\:\fn$ x:Bool.x :Bool$\rightarrow					$Bool}
			\AxiomC{$\checkmark$}
			\LeftLabel{\textsc{(TRUE)}}
			\UnaryInfC{$\emptyset\:\vdash$ true :Bool}
			\LeftLabel{\textsc{(APP)}}
			\BinaryInfC{$\emptyset\:\vdash$  ($\fn$ x:Bool.x) true :Bool}
		\end{prooftree}}
		}
		
	
    \vspace*{1 cm}	
	\item f:Bool$\rightarrow$ Bool $\vdash$ f (if false then true else false) :Bool\\
		Sia $\Gamma$ = f:Bool$\rightarrow$Bool \\
		\scalebox{.75}{
		\parbox{1cm}{
		\begin{prooftree}  
			\AxiomC{f:Bool$\rightarrow$ Bool $\in$ $\Gamma$}
			\LeftLabel{\textsc{(VAR)}}
			\UnaryInfC{$\Gamma$ $\vdash$ f :Bool $\rightarrow$ Bool} 
			\AxiomC{$\checkmark$}
			\LeftLabel{\textsc{(FALSE)}}
			\UnaryInfC{$\Gamma$ $\vdash$ false :Bool}
			\AxiomC{$\checkmark$}
			\LeftLabel{\textsc{(TRUE)}}
			\UnaryInfC{$\Gamma$ $\vdash$ true :Bool}
			\AxiomC{$\checkmark$}
			\LeftLabel{\textsc{(FALSE)}}
			\UnaryInfC{$\Gamma$ $\vdash$ false :Bool}
			\LeftLabel{\textsc{(IF-THEN-ELSE)}}
			\TrinaryInfC{$\Gamma$ $\vdash$ if false then true else false :Bool}
			\LeftLabel{\textsc{(APP)}}
			\BinaryInfC{$\Gamma$ $\vdash$ f (if false then true else 							false):Bool}
		\end{prooftree}}
		} 
	
	
    \vspace*{1 cm}
	\item f:Bool$\rightarrow$ Bool $\vdash\:\fn$ x:Bool.f(if x then false else x) :Bool$\rightarrow$Bool 
		Sia $\Gamma$ = f:Bool$\rightarrow$Bool \\
		\scalebox{.75}{
		\parbox{1cm}{
		\begin{prooftree}  
			\AxiomC{f:Bool $\rightarrow$ Bool $\in$ $\Gamma$, x:Bool}
			\LeftLabel{\textsc{(VAR)}}
			\UnaryInfC{$\Gamma$, x:Bool $\vdash$ f: Bool $\rightarrow$ Bool} 
			\LeftLabel{\textsc{(FUN)}}
			\UnaryInfC{$\Gamma$ $\vdash\:\fn$ x:Bool.f: ? $\rightarrow$ Bool 
			$\rightarrow$ Bool} 
			\AxiomC{x:Bool $\in$ $\Gamma$}
			\LeftLabel{\textsc{(VAR)}}
			\UnaryInfC{$\Gamma$ $\vdash$ x :?}
			\AxiomC{$\checkmark$}
			\LeftLabel{\textsc{(TRUE)}}
			\UnaryInfC{$\Gamma$ $\vdash$ true :Bool}
			\AxiomC{$\checkmark$}
			\LeftLabel{\textsc{(FALSE)}}
			\UnaryInfC{$\Gamma$ $\vdash$ false :Bool}
			\LeftLabel{\textsc{(IF-THEN-ELSE)}}
			\TrinaryInfC{$\Gamma$ $\vdash$ (if x then false else x):? 
			$\rightarrow$ Bool}
			\LeftLabel{\textsc{(APP)}}
			\BinaryInfC{$\Gamma\:\vdash\:\fn$ x:Bool.f(if x then false 
			else x)
			 :Bool$\rightarrow$Bool}
		\end{prooftree}}
		} \\
		Quindi ?=Bool.	
\end{enumerate}
 
    \vspace*{1 cm}

\subsection*{Esercizio 2.1}
Trovare un contesto $\Gamma$ tale che $\Gamma\:\vdash$' f x y : Bool sia derivabile.

\subsubsection*{Svolgimento}
 	\begin{prooftree} 
		\AxiomC{f:$T_2\:\rightarrow\:T_1\:\rightarrow$ Bool $\in\:\Gamma$}
		\LeftLabel{\textsc{(VAR)}}
		\UnaryInfC{$\Gamma\:\vdash$ f :$T_2\:\rightarrow\:T_1\:\rightarrow$
		 Bool}
		\AxiomC{x:$T_2\:\in$ x:$T_2$}
		\LeftLabel{\textsc{(VAR)}}
		\UnaryInfC{$\Gamma\:\vdash$ x :$T_2$}
		\LeftLabel{\textsc{(APP)}}
		\BinaryInfC{$\Gamma\:\vdash$ f x :$T_1\:\rightarrow$ Bool}
		\AxiomC{y :$T_1$ $\in$ $\Gamma$}
		\LeftLabel{\textsc{(VAR)}}
		\UnaryInfC{$\Gamma\:\vdash$ y :$T_1$}
		\LeftLabel{\textsc{(APP)}}
		\BinaryInfC{$\Gamma\:\vdash$ f x y :Bool}
	\end{prooftree} 
	
	Esiste un contesto per questo programma, in quanto il programma si puo' ottenere applicando le regole e costruendo l'albero di derivazione. \\
	Fanno parte del contesto le seguenti regole di tipo:
\begin{itemize}
	\item f:$T_2\:\rightarrow\:T_1\:\rightarrow$ Bool
	\item x :$T_2$
	\item y :$T_1$
\end{itemize}

\vspace*{2 cm}
Caso f (x y):\\
 	\begin{prooftree} 
		\AxiomC{f :$T_1\:\rightarrow$ Bool $\:\in\:\Gamma$}
		\LeftLabel{\textsc{(VAR)}}
		\UnaryInfC{$\Gamma\:\vdash$ f :$T_1\:\rightarrow$ Bool}
		\AxiomC{x:$T_2\:\rightarrow\:T_1\:\in\:\Gamma$}
		\LeftLabel{\textsc{(VAR)}}
		\UnaryInfC{$\Gamma\:\vdash$ x :$T_2$}
		\AxiomC{y:$T_2\:\in\:\Gamma$}
		\LeftLabel{\textsc{(VAR)}}
		\UnaryInfC{$\Gamma\:\vdash$ y :$T_2$}
		\LeftLabel{\textsc{(APP)}}
		\BinaryInfC{$\Gamma\:\vdash$ x y :$T_1$}
		\LeftLabel{\textsc{(APP)}}
		\BinaryInfC{$\Gamma\:\vdash$ f (x y) :Bool}
	\end{prooftree} 
	
	Esiste un contesto anche per questo programma. \\
	Fanno parte del contesto le seguenti regole di tipo:
\begin{itemize}
	\item f:$T_1\:\rightarrow\:$ Bool
	\item x :$T_2\:\rightarrow\:T_1$
	\item y :$T_2$
\end{itemize}


\subsection*{Esercizio 2.2}
Il giudizio $\Gamma\:\vdash$' x x : T e' derivabile? Se si', trovare una derivazione per qualche $\Gamma$, T, altrimenti provare che non e' derivabile.

\subsubsection*{Svolgimento}
 	\begin{prooftree} 
		\AxiomC{$\Gamma\:\vdash$ x :$T_1\:\rightarrow\:$T}
		\AxiomC{$\Gamma\:\vdash$ x :$T_1$}
		\LeftLabel{\textsc{(APP)}}
		\BinaryInfC{$\Gamma\:\vdash$ x x :T}
	\end{prooftree} 

No, non e' derivabile. Sarebbe necessario avere un sistema di tipi che ammetta il tipo ricorsivo. \\



\section{I Tipi Semplici (note 45)}
 
\subsubsection*{Esercizio 3.2}
Provare che ogni sottotermine di un termine ben tipato e' ben tipato.
\subparagraph{Svolgimento}
Per la risoluzione di questo esericizio e' necessario utilizzare il metodo dell'induzione sui passi di derivazione, ovvero sull'altezza dell'albero di derivazione.

\begin{description}

 \item[Caso Base $h=1$] A questo livello il termine e il sottotermine coincidono. Quindi, il fatto che ogni sottotermine risulta essere ben tipato e' un assioma.

 \item[Caso Induttivo $h$ ] Questo caso e' composto dall'applicazione della regola e dai vari sottotermini che la riguardano. Quindi premesso che questa applicazione sia valida per $h = 0$ e che l'ipotesi sia vera anche per h allora per induzione vale anche su $h + 1$ e suoi sottotermini contenuti al passo $h$.
 
 \end{description}
 
\subsubsection*{Esercizio 3.12} 
Dimostrare subject reduction per induzione sulla derivazione di $\Gamma\:\vdash$ M : T.

\subparagraph*{Svolgimento}

Si vuole quindi dimostrare che:

\begin{center}
	Se $\Gamma \vdash{} M : T$ e $M \to{} M'$, allora $\Gamma \vdash{} M' : T$.
\end{center}
%La dimostrazione per induzione va fatta sul numero di passi che $M$ portano a $M'$.
%Dobbiamo ragionare su quella che e' la lunghezza dei cammini della derivazione:

%Caso Base l = 0:\\
%Otteniamo che $M$ = $M'$ e la tesi e' trivialmente vera.

%Caso Induttivo l+1:\\
%In questo passo ho che $M \rightarrow^{*}_l M_1 \rightarrow M'$, e per ipotesi induttiva ho che $\Gamma \vdash M_1 : T$.\\
%A questo punto ho che $M_1 \rightarrow M'$ e quindi posso applicare subject-reduction per ottenere che $\Gamma \vdash M' : T$.

%Ovviamente grazie ai teoremi di Sostituzione, Preservazione e Progressione so per certo che qualsiasi sia il contesto %il risultato non cambia

\begin{description}


 \item[TRUE] $\Gamma \vdash{} M : T$ \'e $\Gamma \vdash{} \true{} :
  \Bool$.
  Il teorema \'e vacuamente vero per questa derivazione siccome
  $\not \exists{M'}. M \to{} M'$.
  
 \item[False]
  $\Gamma \vdash{} M : T$ \'e $\Gamma \vdash{} \false{} : \Bool$.
  Analogo a True.

\item[Nat]
  $\Gamma \vdash{} M : T$ \'e $\Gamma \vdash{} n : \Nat$. Analogo a True.

\item[Var] $\Gamma \vdash{} M : T$ \'e $\Gamma \vdash{} x : T$.
  Analogo a True.

\item[Fun] $\Gamma \vdash{} M : T$ \'e
  $\Gamma \vdash{} fn x\:T_1 C : T_1 \to{} T_2$. Analogo a True.
  
  \item[Sum]
  $\Gamma \vdash{} M : T$ \'e $\Gamma \vdash{} A + B : \Nat$.
 
  Per quanto assunto dalla regola di tipo \emph{Sum}, sappiamo per certo che
  anche $A$ e $B$ hanno tipo $\Nat$ sotto contesto $\Gamma$.

  Supponiamo che $A + B \to{} M'$ (sappiamo che $M'$ \'e unico poich\'e l'esecuzione
   \'e  deterministica).Tale riduzione pu\'o essere:

  \begin{itemize}
    \item \emph{Sum}: $M' \equiv{} n$. Per la regola di tipo
      \emph{Nat}, si ha che $T \equiv{} \Nat$ e che
      $\Gamma{} \vdash{} M' : \Nat$;
    \item \emph{Sum-Left}: $M' \equiv{} A' + B$. 
    Per gli assunti della regola $\emph{Sum}$ ho che $\Gamma \vdash A : \Nat$ e per $\emph{Sum-Left}$ $A \to A'$.
    Posso quindi applicare l'ipotesi induttiva\footnote{perch \'e  la derivazione $\Gamma \vdash A :\Nat$ richiede meno passaggi}
    per ottenere il giudizio $ \Gamma \vdash A' : \Nat$.
    Sono quindi soddisfatte le premesse della regola $\emph{Sum}$ per $M'$ e quindi  $\Gamma{} \vdash{} M' : \Nat \equiv T$;

    \item \emph{Sum-Right}: $M' \equiv{} v + B'$. Il ragionamento  \'e  analogo al caso $\emph{Sum-Left}$.
  \end{itemize}

\item[Minus]
  $\Gamma \vdash{} M : T$ \'e $\Gamma \vdash{} A - B : \Nat$. Analogo a
  \emph{Sum}.

\item[IfThenElse] $\Gamma \vdash{} M : T$ \'e 
  $\Gamma \vdash{} \IF{M_1}{M_2}{M_3}$.
  
  Dal giudizio di tipo abbiamo che:

  \begin{itemize}
  	\item $\Gamma \vdash M_1 : \Bool$;
    \item $\Gamma \vdash{} M_2 : T$;
    \item $\Gamma \vdash{} M_3 : T$.
  \end{itemize}

  Supponiamo che $\IF{M_1}{M_2}{M_3} \to{} M'$ (sappiamo che $M'$ \'e unico poich\'e l'esecuzione
  \'e deterministica). Tale riduzione pu\'o essere:

  \begin{itemize}
    \item \myrule{If}: $M' \equiv{} \Gamma \vdash{} \mbox{if } M'_1
      \mbox{ then } M_2 \mbox{ else } M_3$. Dato che il giudizio $\Gamma \vdash M_1 : \Bool$ richiede una derivazione in meno e che $M_1 \to M_1'$, possiamo applicare l'ipotesi induttiva per ottenere il giudizio $\Gamma \vdash M_1' : \Bool$. Si ha quindi che la regola di tipo \myrule{IfThenElse} continua a valere per il termine $M'$ e dato che i termini $M_2$ e $M_3$ non sono cambiati, anche $M'$ ha tipo $T$;
    \item \myrule{If-True}: $M' \equiv{} M_2$. $M_2$ ha tipo $T$ e quindi anche $M'$ ha tipo $T$;
    \item \myrule{If-False}: $M' \equiv{} M_3$.  Analogo a \myrule{If-True}
  \end{itemize}
  
  
\item[App] $\Gamma \vdash{} M : T$ \'e  $\Gamma \vdash{} A \: B : T_2$.
  Dal giudizio di tipo abbiamo che:

  \begin{itemize}
    \item $\Gamma \vdash{} A : T_1 \to{} T_2$;
    \item $\Gamma \vdash{} B : T_1$.
  \end{itemize}

  Supponiamo che $A \: B \to{} M'$ (sappiamo che $M'$ \'e  unico poich\'e  l'esecuzione
  \'e  deterministica). Tale riduzione pu\'o essere:

  \begin{itemize}
    \item \myrule{Beta}: $M' \equiv{} \Gamma \vdash{}
      Z[x \coloneqq{} B]$\footnote{dato $A \equiv{} FN{x}{T_1}Z$}.
      Per la regola di tipo $\myrule{Fun}$ ho che $Z : T_2$ e quindi, per
       il Substitution Lemma, $\Gamma \vdash{} Z[x \coloneqq{} B] : T_2$. Per
      questo motivo si ha che $\Gamma \vdash{} M' : T_2$ con $T_2 \equiv T$;
    \item \myrule{App-1}: $M' \equiv{} A'\: B$. Dato che il giudizio $\Gamma \vdash A : T_1 \to T_2$ richiede una derivazione con meno passaggi e che $A \to A'$ posso applicare l'ipotesi induttiva per ottenere il giudizio $A' : T_1 \to T_2$. Il termine $M'$ soddisfa le premesse della regola di tipo \myrule{App} e quindi $M'$ ha tipo $T_2 \equiv T$.

    \item \myrule{App-2}: $M' \equiv{} v\: B'$. Analogo ad \myrule{App-1}.
  \end{itemize}

 
\end{description}




\subsubsection*{Esercizio 3.13}

Vale l'opposto di subject reduction, i.e. se $\Gamma\:\vdash$ $M'\: :  $T e $M\:\rightarrow\:$M', allora $\Gamma\:\vdash$ $M$ : $T$ (detto subject expansion)? Dimostrarlo oppure dare un controesempio.
\subparagraph*{Svolgimento}
La risposta e' no:

E sufficiente pensare a $M \rightarrow M_1$ con $\Gamma\vdash$ $M_1$ : $\sigma$ e che 
$\Gamma\nvdash$ $M$: $\sigma$\\
Quanto ripostato sopra pu\`o essere semplificato tramite questo esempio:\\
Supponimao di avere $M_1$ $\equiv$ $A$ : $T$ e $M$ $\equiv$ if true then A else B, ovvero il caso in cui $M$ $\rightarrow$ $M_1$ per la regola (If-True).\\
In questo caso la derivazione di $\Gamma$ $\vdash$ $M$ : $T$ si riesce ad ottenere usando la regola $IfThenElse$, tuttavia questo impone che nel contesto ci siano questi giudizi:
\begin{itemize}
\item $\Gamma$ $\vdash$ $true$:$Bool$ che \`e un assioma
\item $\Gamma$ $\vdash$ $A$:$T$ soddisfatto per l'ipotesi 
\item  $\Gamma$ $\vdash$ $B$:$T$ questa informazione non si riesce a repire ne' dall'ipotesi, ne' dal contesto e pertanto non c'\`e garanzia sull'uguaglianza di tipi di $A$ e $B$\\
Pertanto possiamo creare un termine $N$ come $if$ $true$ $then$ $4$ $else$ $false$, da cui si riesce ad attenere $M'$ $\equiv$ $A$ $\equiv$ $4$. Questo termine, $M'$ \`e ben tipato, lo stesso non si puo' dire per M. 


\end{itemize}







\subsubsection*{Esercizio 3.14}
Se al posto della regola $(APP)\:$ si definisce la regola seguente:

\begin{prooftree} 
	\AxiomC{$\Gamma\:\vdash$ $M\: :  $T$\:\rightarrow\:$T}
	\AxiomC{$\Gamma\:\vdash$ $N\ :$T}
	
	\LeftLabel{\textsc{(APP')}}
	\BinaryInfC{$\Gamma\:\vdash$ $M$ $N$ : $T$  }
\end{prooftree} 

sarebbe ancora vero il teorema di safety?
\subparagraph*{Svolgimento}
Si, lo si puo' dimostrare considerando che APP' non e' altro che una specificazione di APP: se poniamo X uguale all'insieme dei tipi di APP, Y e' un sottoinsieme di X corrispondente ai possibili tipi di APP'.

Di conseguenza il teorema di safety vale anche per APP'

\subsubsection*{Esercizio 3.15}
Se al posto delle regole $(APP)\:$ e $(FUN)\:$ si definissero le seguenti regole:
\begin{prooftree} 
	\AxiomC{$\Gamma\:\vdash$ $M\: :  $T$\:\rightarrow\:$T}
	\AxiomC{$\Gamma\:\vdash$ $N\ :$T}
	
	\LeftLabel{\textsc{(APP')}}
	\BinaryInfC{$\Gamma\:\vdash$ $M$ $N$ : $T$  }
\end{prooftree} 

e


\begin{prooftree} 
	\AxiomC{$\Gamma\:$, x : $T1\vdash$ $M\: : $T}
	\AxiomC{}
	\LeftLabel{\textsc{(FUN')}}
	\BinaryInfC{$\Gamma\:\vdash$ $fn$ x : $T1.M:T\rightarrow\:$T  } 
\end{prooftree} 

sarebbe ancora vero il teorema di safety?

\subparagraph*{Svolgimento}

La risposta e' no:\\
In questo caso la $FUN_1$ non fa alcun controllo sul tipo dell'argomento e quindi esiste la possibilita' di arrivare ad un passo della derivazione che produce un termine STUCK, invalidando il teorema stesso.\\
Supponiamo di avere il termine:\\
$M$ $=$ $(fn$ $x$:$Bool.if$ $x$ $then$ $1$ $else$ $0)$ $(1)$ 

e creiamo il suo albero di derivazione:




\begin{prooftree} 
	\AxiomC{$ \checkmark $}
	\UnaryInfC{$x$:$Bool$ $\in$  $\Gamma$ }
	\RightLabel{\textsc{($Var$)}}
	\UnaryInfC{$\Gamma$ $\vdash$ $x$:$Bool$ }
	\AxiomC{$ \checkmark $}
	\LeftLabel{\textsc{($Nat$)}}
	\UnaryInfC{$\Gamma$ $\vdash$ $1$:$Nat$}
	\AxiomC{$ \checkmark $}
	\LeftLabel{\textsc{($Nat$)}}
	\UnaryInfC{$\Gamma$ $\vdash$ $0$:$Nat$}
	\LeftLabel{\textsc{($IfThenElse$)}}
	\TrinaryInfC{$x$:$Bool$ $\vdash$ $if$ $x$ $then$ $1$ $else$ $0$ :$Nat$}
	\LeftLabel{\textsc{($FUN_1$)}}
	\UnaryInfC{$\empty \vdash fn$ $x$:$Bool.if$ $x$ $then$ $1$ $else$ $0$ :$Nat \rightarrow$ :$Nat$}
	\AxiomC{$ \checkmark $}
	\LeftLabel{\textsc{($Nat$)}}
	\UnaryInfC{$\empty \vdash 1$:$Nat$}
	\LeftLabel{\textsc{($APP_1$)}}
	\BinaryInfC{$\empty \vdash (fn$ $x$:$Bool.if$ $x$ $then$ $1$ $else$ $0)$ $(1)$  } 
\end{prooftree}


In questo caso possiamo osservare che il termine $M$ risulta essere ben tipato secondo le nuove regole di tipo.\\
Tuttavia, \`e evidente che l'invocazione della funzione in questione porta $M$ ad un termine stuck, questo perch\`e il sottotermine $N$ $=$ $if$ $x$ $then$ $1$ $else$ $0$ si aspetta un $Bool$, infatti le regole della semantica operazionale che gestiscono l'$if$, permettono di avanzare nella computazione.\\
Nel caso in cui la guardia non sia un valore, la computazione procede con la valutazione della guardia, altrimenti procede il ramo $TRUE$ o con il ramo $FALSE$.\\
In questo caso, poich\`e il valore passato alla funzione \`e un intero, e quindi gi\`a un valore le uniche regole applicabili sarebbero $(IF-TRUE)$ e $(IF-FALSE)$ ma poich\`e la guardia \'e 1 non sono applicabili e quindi si otterrebbe un termine STUCK.


\subsubsection*{Esercizio 3.16}
Se si aggiungessero al sistema di tipi i seguenti due assiomi

\begin{prooftree} 
	\AxiomC{}
	\AxiomC{}
	\LeftLabel{\textsc{(TRUE')}}
	\BinaryInfC{$\Gamma\:\vdash$ true : $Nat$  } 
\end{prooftree}

e

\begin{prooftree} 
	\AxiomC{}
	\AxiomC{}
	\LeftLabel{\textsc{(FALSE')}}
	\BinaryInfC{$\Gamma\:\vdash$ false : $Nat$  } 
\end{prooftree}

sarebbe ancora vero il teorema di safety?
\subparagraph*{Svolgimento}
La risposta e' no:\\
Data la definizione di Teorema di Safety, e' evidente che viene meno la stessa ipotesi ovvero che M sia un valore o un che non evolva in un termine STUCK.

Con queste regole di tipo, infatti, sarebbe permessa anche l'operazione SUM che genererebbe un termine STUCK, invalidando il teorema stesso.\
Supponiamo di avere il termine: $M$ $=$ $true+4$


\begin{prooftree} 
	\AxiomC{$ \checkmark $}
	\LeftLabel{\textsc{($Nat$)}}
	\UnaryInfC{$\empty \vdash 4$:$Nat$}
	\AxiomC{$ \checkmark $}
	\LeftLabel{\textsc{($Nat$)}}
	\UnaryInfC{$\empty \vdash true$:$Nat$}
	\LeftLabel{\textsc{($SUM$)}}
	\BinaryInfC{$\empty \vdash$ $true$ $+$ $4$  } 
\end{prooftree}

Anche in questo caso, come nel precendete, notiamo che le regole di tipo permettono la derivazione dell'albero.\\
Tuttavia, se applicassimo le regole della semantica operazionale non potremo avanzare perch\`e entrambi gli addendi sono gi\'a dei valori e quindi non posso applicare $SUM-LEFT$ o $SUM-RIGHT$ ma non posso neppure applicare $SUM$ perch\'e genero un termine STUCK in quanto $true$ non pu\'o essere sommato a $4$.

\subsubsection*{Esercizio 3.17}

Dimostrare il seguente fatto: se $\Gamma\:\vdash$ $M$ : $T$ \`e derivabile, allora  \textit{fv($M$)}  $\subseteq$ \textit{Dom($\Gamma$)}
\subparagraph*{Svolgimento}

Si procede con la dimostrazione per induzione:

\begin{itemize}


\item Caso Base 
		\begin{itemize}[label=$\star$]

		\item \textbf{True}  $\Gamma \vdash{} M : T$ \'e $\Gamma \vdash{} \true{} : \Bool$.
  		Il teorema \'e vacuamente vero per questa derivazione siccome $\not \exists{M'}. M \to{} M'$.\\
  		Il tutto si pu\`o riassumere con $fv(M)$ $=$ $fv(true)$ $=$ $\emptyset$ $\subseteq$ $\emph{Dom(}\Gamma\emph{)}$

  
 		\item \textbf{False} $\Gamma \vdash{} M : T$ \'e $\Gamma \vdash{} \false{} : \Bool$. Analogo a True.

		\item \textbf{Nat} $\Gamma \vdash{} M : T$ \'e $\Gamma \vdash{} n : \Nat$. Analogo a True.

		\item \textbf{Var}  Se $\Gamma$ $\vdash$ $x$ : $T$ , allora per la regola $(Var)$ possiamo asserire che $x$ : $T$ $\in$ $\Gamma$ e di conseguenza abbiamo che  $fv(M)$ $=$ $fv(x)$ $=$ $\{x\}$ $\subseteq$ $\emph{Dom(}\Gamma\emph{)}$

  
		\end{itemize}
		In questo passo il fatto e' un assioma, questo perche' non esistono variabili libere quindi $fv(M)$ e' $\emptyset$ e il vuoto e un sottoinsieme del $Dom(\Gamma)$

\item Caso Induttivo
		
		\begin{itemize}[label=$\star$]

		\item \textbf{SUM}  in questo caso $M$ $\equiv$ $A$ $+$ $B$ e quindi per definizione le variabili libere di $M$ sono $f$ $v(A$ $+$ $B)$ $=$ $f$ $v(A)$ $\cup$ $f$ $v(B)$.\\ Per ipotesi induttiva sappiamo che $fv(A)$ e $fv(B)$ $\subseteq$ $\emph{Dom(}\Gamma\emph{)}$.\\Poich\`e, come anticipato, $fv(M)$ $\equiv$ $fv(A)$ $\cup$ $fv(B)$ possiamo dedurre che $fv(M)$ $\subseteq$ $\emph{Dom(}\Gamma\emph{)}$.
  		
  
 		\item \textbf{MINUS} in questo caso $M$ $\equiv$ $A$ $-$ $B$ e la sua dimostrazione risulta essere analoga a quella di SUM

		\item \textbf{IfThenElse} in questo caso $M$ $\equiv$ $if$ $C$ $then$ $A$ $else$ $B$ e quindi per definizione, le variabili libere di $M$ sono $fv(M)$ $=$ $fv(A)$ $\cup$ $fv(B)$ $\cup$ $fv(C)$.\\ Per ipotesi induttiva sappiamo che $fv(A)$, $fv(B)$ e $fv(C)$ $\subseteq$ $\emph{Dom(}\Gamma\emph{)}$.\\Poich\`e, come anticipato, $fv(M)$ $\equiv$ $fv(A)$ $\cup$ $fv(B)$ $\cup$ $fv(C)$ possiamo dedurre che $fv(M)$ $\subseteq$ $\emph{Dom(}\Gamma\emph{)}$.

		\item \textbf{FUN}  in questo caso $M$ $\equiv$ $fn$ $x.N$ e quindi per definizione le variabili libere di $M$ sono 
		$fv(M)$ $=$ $fv(N)\textbackslash \{x\}$ ($fv(M)$ $\subseteq$ $fv(N)$).\\ Per ipotesi induttiva sappiamo che $fv(N)$ $\subseteq$ $\emph{Dom(}\Gamma\emph{)}$.\\Poich\`e, come anticipato, $fv(M)$ $\equiv$ $fv(N)\textbackslash \{x\}$ possiamo dedurre che $fv(M)$ $\subseteq$ $\emph{Dom(}\Gamma\emph{)}$. 
		
		\item \textbf{APP} in questo caso $M$ $\equiv$ $A$ $B$ e quindi per definizione, le variabili libere di $M$ sono $fv(M)$ $=$ $fv(A)$ $\cup$ $fv(B)$.\\ Per ipotesi induttiva sappiamo che $fv(A)$ e $fv(B)$ $\subseteq$ $\emph{Dom(}\Gamma\emph{)}$.\\Poich\`e, come anticipato, $fv(M)$ $\equiv$ $fv(A)$ $\cup$ $fv(B)$ possiamo dedurre che $fv(M)$ $\subseteq$ $\emph{Dom(}\Gamma\emph{)}$. 
  
		\end{itemize}

\end{itemize}

\subsubsection*{Esercizio 3.18}
Ricostruire il tipo dei seguenti termini:
\begin{itemize}
\item $fn$ $x$:$T1.fn$ $y$:$T2.if$ $y$ then $x$ else true
\item $fn$ $x:Nat:\rightarrow\:$Bool.x
\item $fn$ $f:T.fn$ $x:T'.f$( if true  then  $x$  else  $f$ $x$)
\item $fn$ $f:T1.fn$ $g:T2.$ if ($f$ ($g$ true)) then $f$ ($fn$ $x:$T3.true) else $f$($fn$ $x:T4.x$)
\end{itemize}

\subparagraph*{Svolgimento}
\begin{enumerate}[label=\alph*), leftmargin=*]


\item $fn$ $x$:$T1.fn$ $y$:$T2.if$ $y$ then $x$ else true  \\
		\scalebox{.75}{
		\parbox{1cm}{
		\begin{prooftree}
			\AxiomC{y:Bool $\in$ $\Gamma$}
			\LeftLabel{\textsc{(VAR)}} 
			\LeftLabel{\textsc{(VAR)}}
			\UnaryInfC{$\Gamma$,$x$:$T1$, $\vdash$ $y$ : $Bool$}
			\LeftLabel{\textsc{(VAR)}}
			\AxiomC{x:B$x$:$Nat:\rightarrow Bool$ $\in$ $\Gamma$}
			\LeftLabel{\textsc{(VAR)}}
			\UnaryInfC{$\Gamma$,$x$:$T1$, $\vdash$ $x$ : $Bool$}
			\LeftLabel{\textsc{(VAR)}}
			\AxiomC{$\checkmark$}
			\LeftLabel{\textsc{(TRUE)}}
			\UnaryInfC{$\Gamma$,$x$:$T1$, $\vdash$ $true$ : $Bool$}
			\LeftLabel{\textsc{(IF-THEN-ELSE)}}
			\TrinaryInfC{$x$:$T1$,$y$:$T2$ $\vdash$ $if$ $y$ then $x$ else true}
			\LeftLabel{\textsc{(FUN)}}
			\UnaryInfC{$x$:$T1$ $\vdash$  $fn$ $y$:$T2.if$ $y$ then $x$ else true}
			\LeftLabel{\textsc{(FUN)}}
			\UnaryInfC{$\emptyset\:\vdash$  $fn$ $x$:$T1.fn$ $y$:$T2.if$ $y$ then $x$ else true}
		\end{prooftree}}
		}
		
		Il risultato finale e' $Bool \rightarrow (Bool \rightarrow Bool)$
		
\item $fn$ $x:Nat:\rightarrow\:$Bool.x  \\
		\scalebox{.75}{
		\parbox{1cm}{
		\begin{prooftree}
		
			\AxiomC{$x$:$Nat:\rightarrow Bool$ $\in$ $\Gamma$}
			\LeftLabel{\textsc{(VAR)}}
			\UnaryInfC{$x$:$Nat:\rightarrow Bool$ $\vdash$  $x$:$T$}
			\LeftLabel{\textsc{(FUN)}}
			\UnaryInfC{$\emptyset\:\vdash$  $fn$ $x:Nat:\rightarrow\:$Bool.x}
		\end{prooftree}}
		}
		
		Il risultato finale e' $(Nat \rightarrow Bool) \rightarrow (Nat \rightarrow Bool)$
		
		
\item $fn$ $f:T.fn$ $x:T'.f$(if true  then  $x$  else  $f$ $x$)  \\
		\scalebox{.75}{
		\parbox{1cm}{
		\begin{prooftree}
			\AxiomC{$f$:$T$,$x$:$T_1$ $\vdash$ $f$:$T_3 \rightarrow T_2$}
			
			%%else
			
			\AxiomC{$f$:$T$,$x$:$T_1$ $\vdash$ $f:T_4 \rightarrow T$}
			\AxiomC{$T_1 = T_4$}
			\LeftLabel{\textsc{(VAR)}}
			\UnaryInfC{$f$:$T$,$x$:$T_1$ $\vdash$ $x:T_4$}
			\LeftLabel{\textsc{(APP)}}
			\BinaryInfC{$f$:$T$,$x$:$T_1$ $\vdash$ $f x$ : $T_3$}
			
			%%			
					
			%%then
			
			\AxiomC{$T_1 = T_3$}
			\LeftLabel{\textsc{ }}
			\UnaryInfC{$f$:$T$,$x$:$T_1$ $\vdash$ $x$ : $T_3$}
						
			%%			
			%% if
			\AxiomC{$\checkmark$}
			\LeftLabel{\textsc{(TRUE)}}
			\UnaryInfC{$f$:$T$,$x$:$T_1$ $\vdash$ $true$ : $Bool$}
			%%
			\LeftLabel{\textsc{(IF-THEN-ELSE)}}
			\TrinaryInfC{$f$:$T$,$x$:$T_1$ $\vdash$ $ if true$ then $x$ else $f$ $x$ : $T_3$}
			
			\LeftLabel{\textsc{(APP)}}
			\BinaryInfC{$f$:$T$,$x$:$T_1$ $\vdash$ $f$ $(if true$ then $x$ else $f$ $x$)}
			\LeftLabel{\textsc{(FUN)}}
			\UnaryInfC{$f$:$T$ $\vdash$  $fn$ $x$:$T_1.f$ $(if true$ then $x$ else $f$ $x$)}
			\LeftLabel{\textsc{(FUN)}}
			\UnaryInfC{$\emptyset\:\vdash$  $fn$ $f:T.fn$ $x:T_1.f$(if true  then  $x$  else  $f$ $x$)}
		\end{prooftree}}
		}
		
		Il risultato finale e' dato dal fatto che $T = T_1 = T_" = T_3 = T_4$ quindi $(T \rightarrow T) \rightarrow (T \rightarrow T \rightarrow T)$
		
	
\item $fn$ $f:T1.fn$ $g:T2.$if ($f$ ($g$ true)) then $f$ ($fn$ $x:$T3.true) else $f$($fn$ $x:T4.x$)  \\
		\scalebox{.35}{
		\parbox{1cm}{
		\begin{prooftree}
			
			%%else
			\AxiomC{$x$:$T_4$ $\in$ $\Gamma$}
			\LeftLabel{\textsc{(VAR)}}
			\UnaryInfC{$f$:$T_1$,$g$:$T_2$,$x$:$T_4$ $\vdash$ x : $T_4$}
			\LeftLabel{\textsc{(FUN)}}
			\UnaryInfC{$f$:$T_1$,$g$:$T_2$ $\vdash$ $fn$ $x$:$T_4$.x : $T_4 \rightarrow T_4$}
			\AxiomC{$f$:$T_1$,$g$:$T_2$ $\vdash$ $f$ $T_5 \rightarrow T_1 $}
			\LeftLabel{\textsc{(APP)}}
			\BinaryInfC{$f$:$T_1$,$g$:$T_2$ $\vdash$ $f$($fn$ $x:T_4.x$)}
			
			%%			
					
			%%then
			\AxiomC{$\checkmark$}
			\LeftLabel{\textsc{(TRUE)}}
			\UnaryInfC{$f$:$T_1$,$g$:$T_2$,$x$:$T_3$ $\vdash$ true : $Bool$}
			\LeftLabel{\textsc{(FUN)}}
			\UnaryInfC{$f$:$T_1$,$g$:$T_2$ $\vdash$ $fn$ $x$:$T_3$.true : $T_3 \rightarrow Bool$}
			\AxiomC{$f$:$T_1$,$g$:$T_2$ $\vdash$ $f$ $T_5 \rightarrow T_1 $}
			\LeftLabel{\textsc{(APP)}}
			\BinaryInfC{$f$:$T_1$,$g$:$T_2$ $\vdash$ $f$ ($fn$ $x$:$T_3$.true)}
						
			%%			
			%% if
			
			\AxiomC{$f$:$T_1$,$g$:$T_2$ $\vdash$ $g$:$ Bool \rightarrow T_2 $}
			\AxiomC{$\checkmark$}
			\LeftLabel{\textsc{(TRUE)}}
			\UnaryInfC{$f$:$T_1$,$g$:$T_2$ $\vdash$ $true$:$ Bool $}
			\LeftLabel{\textsc{(APP)}}
			\BinaryInfC{$f$:$T_1$,$g$:$T_2$ $\vdash$ $g true$:$ T_2$}
			\AxiomC{$f$:$T_1$,$g$:$T_2$ $\vdash$ $f$:$ T_2 \rightarrow Bool $}
			\LeftLabel{\textsc{(APP)}}
			\BinaryInfC{$f$:$T_1$,$g$:$T_2$ $\vdash$ $f$ ($g$ true)) : $Bool$}
			%%
						
			\LeftLabel{\textsc{(IF-THEN-ELSE)}}
			\TrinaryInfC{$f$:$T_1$,$g$:$T_2$ $\vdash$ $if$ ($f$ ($g$ true)) then $f$ ($fn$ $x$:$T_3$.true) else $f$($fn$ $x:T_4.x$)}
			\LeftLabel{\textsc{(FUN)}}
			\UnaryInfC{$f$:$T_1$ $\vdash$ $fn$ $g:T_2.$if ($f$ ($g$ true)) then $f$ ($fn$ $x$:$T_3$.true) else $f$($fn$ $x:T_4.x$)}
			\LeftLabel{\textsc{(FUN)}}
			\UnaryInfC{$\emptyset\:\vdash$  $fn$ $f:T_1.fn$ $g:T_2.$if ($f$ ($g$ true)) then $f$ ($fn$ $x$:$T_3$.true) else $f$($fn$ $x:T_4.x$)}
		\end{prooftree}}
		}	
		
		Il risultato finale e' dato dal fatto che:
		\begin{itemize}
		
	
		\item $T_4 = Bool$
		\item $T_1 = T_2 \rightarrow Bool$
		\item $T_5 = T_3 \rightarrow Bool$
		\item $T_4 = Bool$
		\item $T_3 = T_4 = Bool$
		\item $T_5 = Bool \rightarrow Bool$
		\item $T_2 = (Bool \rightarrow Bool)$
		\item $T_1 = (Bool\rightarrow Bool) \rightarrow Bool$
		
		\end{itemize}
		e quindi il tipo del termine e' quindi $(Bool\rightarrow Bool) \rightarrow Bool$
		

\end{enumerate} 
\section{Estensioni del linguaggio (note 6)}
\subsection*{Esercizio 4.1}
Discutere quali altri regole/strategie di riduzione sono possibili per i termini coppia. 

\subsubsection*{Svolgimento}
Si potrebbero modificare le regole PAIR 1 e PAIR 2 e mantenere le altre:  

\begin{prooftree}
	\AxiomC{}
	\LeftLabel{\textsc{(PAIR-NOT-EVAL 1)}}
	\UnaryInfC{(M$_1$, M$_2$).\_1 $\rightarrow$ M$_1$}
\end{prooftree}

\begin{prooftree}
	\AxiomC{}
	\LeftLabel{\textsc{(PAIR-NOT-EVAL 2)}}
	\UnaryInfC{(M$_1$, M$_2$).\_2 $\rightarrow$ M$_2$}
\end{prooftree}

In questo modo non serve valutare i termini M$_1$ e M$_2$ prima di fare la proiezione. A seguito della proiezione, solo il termine estratto viene valutato. Le regole PROJECT 1 e PROJECT 2 vanno mantenute perch\`e il termine M potrebbe per esempio essere il risultato di una funzione ed \`e necessario che ci siano delle regole di derivazione che portino ad uno stato in cui sono applicabili le due aggiunte. EVAL PAIR 1 ed EVAL PAIR 2, invece, vanno mantenute perch\`e la coppia, ammesso che contenga valori finali, \`e un buon termine finale anch'essa. Di conseguenza, non \`e necessario che venga estratto uno dei due valori perch\`e l'esecuzione si possa definire corretta e nel caso in cui ci si trovi come termine finale una coppia con valori non finali, \`e necessaria una regola di valutazione che porti avanti l'esecuzione.

\subsection*{Esercizio 4.2}
Scrivere la valutazione dei termini (4 $-$ 1, if true then false else false).\_1 e (fn x : Nat $\ast$ Nat.x.\_2) (4 $-$ 2, 3 $+$ 1).

\subsubsection*{Svolgimento}

\textbf{Termine 1: (4 $-$ 1, if true then false else false).\_1} 
\begin{prooftree}
	\AxiomC{}
	\LeftLabel{\textsc{(MINuS)}}
	\UnaryInfC{4 $-$ 1 $\rightarrow$ 3}
	\LeftLabel{\textsc{(EVAL PAIR 1)}}
	\UnaryInfC{(4 $-$ 1, if true then false else false) $\rightarrow$ (3, if true then false else false)}
	\LeftLabel{\textsc{(PROJECT 1)}}
	\UnaryInfC{(4 $-$ 1, if true then false else false).\_1 $\rightarrow$ (3, if true then false else false).\_1}
\end{prooftree}

\begin{prooftree}
	\AxiomC{}
	\LeftLabel{\textsc{(IF-TRuE)}}
	\UnaryInfC{if true then false else false $\rightarrow$ false}
	\LeftLabel{\textsc{(EVAL PAIR 2)}}
	\UnaryInfC{(3, if true then false else false) $\rightarrow$ (3, false)}
	\LeftLabel{\textsc{(PROJECT 2)}}
	\UnaryInfC{(3, if true then false else false).\_1 $\rightarrow$ (3, false).\_1}
\end{prooftree}

\begin{prooftree}
	\AxiomC{} 
	\LeftLabel{\textsc{(PAIR 1)}}
	\UnaryInfC{(3, false).\_1 $\rightarrow$ 3}
\end{prooftree}


\textbf{Termine 2: (fn x : Nat $\ast$ Nat.x.\_2) (4 $-$ 2, 3 $+$ 1)} 
\begin{prooftree}
	\AxiomC{}
	\LeftLabel{\textsc{(MINuS)}}
	\UnaryInfC{4 $-$ 2 $\rightarrow$ 2}
	\LeftLabel{\textsc{(EVAL PAIR 1)}}
	\UnaryInfC{(4 $-$ 2, 3 $+$ 1) $\rightarrow$ (2, 3 $+$ 1)}
	\LeftLabel{\textsc{(APP2)}}
	\UnaryInfC{(fn x : Nat $\ast$ Nat.x.$\_$2) (4 $-$ 2, 3 $+$ 1) $\rightarrow$ (fn x : Nat $\ast$ Nat.x.$\_$2) (2, 3 $+$ 1)}
\end{prooftree}

\begin{prooftree}
	\AxiomC{}
	\LeftLabel{\textsc{(SuM)}}
	\UnaryInfC{3 $+$ 1 $\rightarrow$ 4}
	\LeftLabel{\textsc{(EVAL PAIR 2)}}
	\UnaryInfC{(2, 3 $+$ 1) $\rightarrow$ (2, 4)}
	\LeftLabel{\textsc{(APP2)}}
	\UnaryInfC{(fn x : Nat $\ast$ Nat.x.$\_$2) (2, 3 $+$ 1) $\rightarrow$ (fn x : Nat $\ast$ Nat.x.$\_$2) (2, 4)}
\end{prooftree}
	 
\begin{prooftree}
	\AxiomC{} 
	\LeftLabel{\textsc{(BETA)}}
	\UnaryInfC{(fn x : Nat $\ast$ Nat.x.$\_$2) (2, 4) $\rightarrow$ (2, 4).$\_$2}
\end{prooftree}	 
	 
\begin{prooftree}
	\AxiomC{} 
	\LeftLabel{\textsc{(PAIR 2)}}
	\UnaryInfC{(2, 4).$\_$2 $\rightarrow$ 4}
\end{prooftree}	 
	 
	 
\subsection*{Esercizio 4.3}
Dimostrare il teorema di safety per il linguaggio contenente interi, booleani, funzioni e records.

\subsubsection*{Svolgimento}
Per dimostrare che il teorema di Safety continua a valere anche per l'estensione del linguaggio con i records \`e necessario andare ad estendere i teoremi di Progressione e di Subject-Reduction.\\
Per rendere pi\`u semplice la dimostrazione, inoltre, \'e necessario anche aggiornare il Lemma di Inversione.

\subparagraph{Teorema di Progressione}

\begin{centering}
	Se $M$ \`e un termine chiuso e ben tipato ($\emptyset \vdash  M : T$), allora o $M$ \`e un valore oppure esiste $M'$ tale che $M \to M'$.
\end{centering}

\noindent Per poter procedere con la dimostrazione del teorema di Progressione \`e necessario aggiungere alla dimostrazione due nuovi casi induttivi, mentre i casi per le altre regole di tipo restano invariati e continuano a valere.

\begin{itemize}
	\item \myrule{Type-Record} Se $M = \{ l_i = M_i \:^{i = 1 \ldots n} \}$ e $\emptyset \vdash M : T$, sappiamo che per la regola \myrule{Type-Record} valgono i giudizi di tipo $\emptyset \vdash M_i : T_i \: \forall \: i = 1\ldots n$ e che gli alberi di derivazione relativi ai vari giudizi sono di altezza inferiore all'albero del giudizio principale.\\ \`E quindi possibile applicare l'ipotesi induttiva sui sotto-termini $M_i$, i quali o sono un valore $v_i$ di tipo $T_i$ oppure possono avanzare in un termine $M_i'$:
	\begin{itemize}
		\item Se sono tutti dei valori si ha $M = \{ l_i = v_i \:^{i = 1 \ldots n} \}$, ovvero $M$ \`e un valore record di tipo $\{ l_i : T_i \:^{i = 1 \ldots n} \}$ e quindi il teorema di Progressione continua a valere banalmente.
		\item Se c'\`e almeno un $M_i$ che non \`e un valore, \`e possibile identificare il termine $M_j$ di indice minimo che non \`e un valore e per il quale continua a valere il giudizio di tipo $\emptyset \vdash M_j : T_j$. Vale quindi l'ipotesi induttiva, ovvero $\exists M_j'$ tale che $M_j \to M_j'$. Posso quindi applicare la regola \myrule{Eval-Record} per far avanzare il termine $M$ al termine $M'$, dove al posto di $M_j$ compare $M_j'$. Anche in questo caso il teorema di Progressione continua a valere.
	\end{itemize}
	\item \myrule{Type-Select} Se $M = N.l_j$ e $\emptyset \vdash M : T$, sappiamo che per la regola \myrule{Type-Select} vale il giudizio di tipo $\emptyset \vdash N : \{ l_i : T_i \:^{i = 1 \ldots n}\}$ (con un albero di derivazione pi\`u piccolo, ovvero il passo precendete) e che $j \in \{1 \ldots n\}$.
	Si ha quindi che $N$ pu\`o essere:
	\begin{itemize}
		\item un valore di tipo record, ovvero $N = \{ l_i = v_i \:^{i = 1 \ldots n} \}$ e quindi $M = \{ l_i = v_i \:^{i = 1 \ldots n} \}.l_j$ con $j \in 1 \ldots n$. In questo caso il termine $M$ pu\`o avanzare nel termine $M' = v_j$ perch\`e sono soddisfatte le premesse della regola \myrule{Select} e quindi il teorema di Progressione continua a valere.
		
		\item un record non ancora completamente valutato, ovvero $N = \{ l_i = v_i \:^{i = 1 \ldots x-1}, l_x = M_x, l_k = M_k \:^{k = x+1\ldots n}\}, \exists M_x': M_x \to M_x'$ e quindi per \myrule{Eval-Record} $N \to N'$ e pertanto anche $M \to M' = N'.l_j$.
		
		\item $N = \IF{M_1}{M_2}{M_3}$ oppure $N = A \: B$. In entrambi i casi vale il giudizio $\emptyset \vdash N : \{ l_i : T_i \:^{i = 1 \ldots n}\}$, ovvero $N$ \`e un termine chiuso, ben tipato e non \`e un valore. 
		Quindi per ipotesi induttiva ho che $\exists N'$ tale che $N \to N'$ e quindi per \myrule{Eval-Select} $\exists M' = N'.l_j$ tale che $M \to M'$.
		Pertanto il teorema di Progressione continua a valere.
	\end{itemize}
\end{itemize}


\subparagraph{Lemma di Inversione}

\begin{itemize}
	\item \myrule{Type-Record}: Se $\Gamma \vdash M = \{ l_i = M_i \:^{i = 1 \ldots n} \} : \{ l_i : T_i \:^{i = 1 \ldots n} \}$ \`e derivabile, allora $\forall i \in 1 \ldots n. \Gamma \vdash M_i : T_i$ \`e derivabile.
	\item \myrule{Type-Select}: Se $\Gamma \vdash M.l_j : T_j$ \`e derivabile, allora $\forall i \in 1\ldots n.\exists T_i. \Gamma \vdash M :  \{ l_i : T_i \:^{i = 1 \ldots n} \}$ \`e derivabile e $j \in 1 \ldots n$.
\end{itemize}

\noindent La dimostrazione di questi casi \`e banale perch\`e derivano dall'applicazione delle regole di tipo.


\subparagraph{Teorema di Subject-Reduction (Preservazione)}

\begin{centering}
	Se $\Gamma \vdash M : T$ e $M \to M'$, allora $\Gamma \vdash M':T$.
\end{centering}

\noindent Anche per questo teorema \`e necessario aggiungere un nuovo caso base per l'assioma \myrule{Select} e due nuovi casi induttivi per \myrule{Eval-Select} e \myrule{Eval-Record}.

\vspace{10px}

\myrule{Select}: $M \equiv N.l_j$ con $N = \{ l_i = v_i \:^{i = 1 \ldots n} \} : \{ l_i : T_i \:^{i = 1 \ldots n} \}$.
Per la regola di tipo \myrule{Type-Select} $M$ ha tipo $T \equiv T_j$ per $j \in 1\ldots n$.
Oltre a ci\`o $M' = v_j$ per $j \in 1\ldots n$ e, per la regola \myrule{Select}, $v_j$ \`e il valore associato all'etichetta $l_j$ di $N$. Quindi, per il Lemma di Inversione esteso, dato che $\Gamma \vdash N : \{ l_i : T_i \:^{i = 1 \ldots n} \}$ \`e derivabile, sono anche derivabili i giudizi $\Gamma \vdash M_i :T_i \: \forall i \in 1 \ldots n$.
In particolare, \`e derivabile il giudizio $\Gamma \vdash v_j : T_j$ e quindi $M' : T_j$, con $T_j \equiv T$.

\vspace{10px}

\myrule{Eval-Select}: $M \equiv N.l$ ed \`e derivabile il giudizio $\Gamma \vdash M \equiv N.l : T $. Inoltre, $ N \to N'$ e $M' \equiv N'.l$ con $l$ appartenente all'insieme delle etichette del record. 
Per il Lemma di Inversione esteso, ho che $\Gamma \vdash N : \{ l_i : T_i \:^{i = 1 \ldots n} \}$ \`e derivabile con un albero di derivazione di altezza inferiore.
Posso quindi applicare l'ipotesi induttiva per dire che anche $\Gamma \vdash N' : \{ l_i : T_i \:^{i = 1 \ldots n} \}$ \`e derivabile, per poi concludere applicando la regola \myrule{Type-Select} a $M' \equiv N'.l$, ottenendo che $\Gamma \vdash N'.l : T$ \`e derivabile e quindi anche $\Gamma \vdash M':T$ \`e derivabile.

\vspace{10px}

\myrule{Eval-Record}: Essendo $M \equiv \{ l_i = v_i \:^{i = 1 \ldots j-1}, l_j = M_j, l_k = M_k \:^{k = j+1\ldots n}\}$, \`e derivabile il giudizio $\Gamma \vdash M : T \equiv \{ l_i : T_i \:^{i = 1 \ldots n} \}$ e $M_j \to M'_j$.
Per il primo caso del Lemma di Inversione esteso, ho che tutti i giudizi $\Gamma \vdash M_i : T_i$ sono derivabili ed in particolare \`e derivabile $\Gamma \vdash M_j : T_j$.
Per ipotesi induttiva sull'altezza dell'albero $M_j \to M'_j$, ottengo che \`e derivabile anche il giudizio $\Gamma \vdash M'_j : T_j$.
Infine, dato che $M' \equiv \{ l_i = v_i \:^{i = 1 \ldots j-1}, l_j = M'_j, l_k = M_k \:^{k = j+1\ldots n}\}$, posso applicare la regola \myrule{Type-Record} per dire che  $\Gamma \vdash M' : \{ l_i : T_i \:^{i = 1 \ldots n} \} \equiv T$ \`e derivabile e quindi anche $\Gamma \vdash M' : T$ \`e derivabile.

\subparagraph{Teorema di Safety}

\begin{centering}
	Se $\emptyset \vdash M :T$ e $M \to^* M'$ con $M'$ tale che $M' \not\to$, allora $M'$ \`e un valore.
\end{centering}

\noindent La dimostrazione \`e una diretta conseguenza dei due teoremi precedenti: per il teorema di Preservazione anche $\emptyset \vdash M': T$ \`e derivabile e, per il teorema di Progressione $M'$ deve essere un valore, perch\'e altrimenti esisterebbe un $M''$ al quale pu\`o ridursi.


	
\section{Estensioni del linguaggio (note 7)}
\subsection*{Esercizio 5.1}
Si ridimostri il teorema di safety per il linguaggio contenente interi, booleani, funzioni, records e tipi varianti. 
\textbf{{\color{red} DA FARE}}
\section{Eccezioni (note 8)}
\subsection*{Esercizio 6.1}
Si dia la semantica operazionale dei termini indicati sopra, osservando come il sollevamento delle eccezioni comporti un salto non locale del flusso di controllo.

\textbf{{\color{red} DA FARE}}


\subsection*{Esercizio 6.2}
Si definisca la semantica operazionale per il liguaggio esteso con i costrutti visti in precedenza: unit, records e varianti.


\textbf{{\color{red} DA FARE}}


\subsection*{Esercizio 6.3}
Si ridimostri il teorema di safety per il linguaggio contenente interi, booleani, funzioni, records, tipi varianti ed eccezioni.

\textbf{{\color{red} DA FARE? Non lo so, non l'ho scritto negli appunti,credo di no}}
\section{Subtyping (note 9)}
\subsection*{Esercizio 7.1}
Scrivere le derivazioni dei giudizi:
\begin{itemize}
	\item \{l:\{a:Nat, b:Nat\}, l':\{m:Nat\}\} <: \{l:\{a:Nat\}, l':\{\}\}
	\item \{l:\{a:Nat, b:Nat\}, l':\{m:Nat\}\} <: \{l:\{a:Nat\}, l':\{m:Nat\}\}	
	\item \{l:\{a:Nat, b:Nat\}, l':\{m:Nat\}\} <: \{l:\{a:Nat\}\}
\end{itemize} 

\subsubsection*{Svolgimento}
 
\textbf{A): \{l:\{a:Nat, b:Nat\}, l':\{m:Nat\}\} <: \{l:\{a:Nat\}, l':\{\}\}}
	
\begin{prooftree} 
	\AxiomC{(SUB-WIDTH)}
	\LeftLabel{\textsc{}}
	\UnaryInfC{$\emptyset\:\vdash$ l:\{a:Nat, b:Nat\}<: l:\{a:Nat\}}
	\AxiomC{(SUB-WIDTH)}
	\LeftLabel{\textsc{}}
	\UnaryInfC{$\emptyset\:\vdash$ l':\{m:Nat\}<: l':\{\}}
	\LeftLabel{\textsc{(SUB-DEPTH)}}
	\BinaryInfC{$\emptyset\:\vdash$ \{l:\{a:Nat, b:Nat\}, l':\{m:Nat\}\} <: \{l:\{a:Nat\}, l':\{\}\}}
\end{prooftree}
 
\textbf{B): \{l:\{a:Nat, b:Nat\}, l':\{m:Nat\}\} <: \{l:\{a:Nat\}, l':\{m:Nat\}\}}
	
\begin{prooftree} 
	\AxiomC{(SUB-WIDTH)}
	\LeftLabel{\textsc{}}
	\UnaryInfC{$\emptyset\:\vdash$ l:\{a:Nat, b:Nat\}<: l:\{a:Nat\}}
	\AxiomC{(REFLEX)}
	\LeftLabel{\textsc{}}
	\UnaryInfC{$\emptyset\:\vdash$ l':\{m:Nat\}<: l':\{m:Nat\}}
	\LeftLabel{\textsc{(SUB-DEPTH)}}
	\BinaryInfC{$\emptyset\:\vdash$ \{l:\{a:Nat, b:Nat\}, l':\{m:Nat\}\} <: \{l:\{a:Nat\}, l':\{m:Nat\}\}}
\end{prooftree}

\textbf{C): \{l:\{a:Nat, b:Nat\}, l':\{m:Nat\}\} <: \{l:\{a:Nat\}\}}
	
\scalebox{.85}{
\parbox{1cm}{
\begin{prooftree} 	
	\AxiomC{(SUB-WIDTH)}
	\LeftLabel{\textsc{}}
	\UnaryInfC{$\emptyset\:\vdash$ l:\{a:Nat, b:Nat\} <: l:\{a:Nat\}}
	\AxiomC{(SUB-WIDTH)}
	\LeftLabel{\textsc{}}
	\UnaryInfC{$\emptyset\:\vdash$ l':\{m:Nat\}<: l':\{\}}
	\LeftLabel{\textsc{(SUB-DEPTH)}}
	\BinaryInfC{$\emptyset\:\vdash$ \{l:\{a:Nat, b:Nat\}, l':\{m:Nat\}\} <: \{l:\{a:Nat\}, l':\{\}\}}
	\AxiomC{(SUB-WIDTH)}
	\LeftLabel{\textsc{}}
	\UnaryInfC{$\emptyset\:\vdash$ \{l:\{a:Nat\}, l':\{\}\} <: \{l:\{a:Nat\}\}}
	\LeftLabel{\textsc{(TRANS)}}
	\BinaryInfC{$\emptyset\:\vdash$ \{l:\{a:Nat, b:Nat\}, l':\{m:Nat\}\} <: \{l:\{a:Nat\}\}}
\end{prooftree}}}

\subsection*{Esercizio 7.2}
Si scriva la derivazione di \{a:Nat, b:Bool, c:Nat\} <: \{b:Bool\} 

\subsubsection*{Svolgimento}
\scalebox{.85}{
\parbox{1cm}{
 \begin{prooftree} 
	\AxiomC{\{a:Nat, b:Bool, c:Nat\} e' permutazione di \{b:Bool, a:Nat, c:Nat\}}
	\LeftLabel{\textsc{PERMUTE}}
	\UnaryInfC{$\emptyset\:\vdash$ \{a:Nat, b:Bool, c:Nat\} <: \{b:Bool, a:Nat, c:Nat\}}
	\AxiomC{(SUB-WIDTH)}
	\LeftLabel{\textsc{}}
	\UnaryInfC{$\emptyset\:\vdash$ \{b:Bool, a:Nat, c:Nat\} <: \{b:Bool\}}
	\LeftLabel{\textsc{(TRANS)}}
	\BinaryInfC{$\emptyset\:\vdash$ \{a:Nat, b:Bool, c:Nat\} <: \{b:Bool\}}
\end{prooftree} }}

\subsection*{Esercizio 7.3}

Dare la derivazione del giudizio $\emptyset\:\vdash$ (fn r:\{l:Nat\}.r.l + 2) \{l= 0, l'= 1\}:Nat. Esiste una sola derivazione di questo giudizio? 

\subsubsection*{Svolgimento}
\scalebox{.75}{
\parbox{1cm}{
 \begin{prooftree} 
	\AxiomC{r.l:Nat $\in\:\Gamma$}	
	\LeftLabel{\textsc{(VAR)}}
	\UnaryInfC{$\emptyset$, r:\{l:Nat\} $\vdash$ r.l:Nat}
	\AxiomC{(NAT)}	
	\LeftLabel{\textsc{}}
	\UnaryInfC{$\emptyset$, r:\{l:Nat\} $\vdash$ 2:Nat}
	\LeftLabel{\textsc{(SUM)}}
	\BinaryInfC{$\emptyset$, r:\{l:Nat\} $\vdash$ (r.l + 2):Nat}
	\LeftLabel{\textsc{(FUN)}}
	\UnaryInfC{$\emptyset\:\vdash$ (fn r:\{l:Nat\}.r.l + 2):\{l:Nat\} $\rightarrow$ Nat}
	\AxiomC{(NAT)}
	\LeftLabel{\textsc{}}
	\UnaryInfC{$\emptyset\:\vdash$ 0:Nat}
	\AxiomC{(NAT)}
	\LeftLabel{\textsc{}}
	\UnaryInfC{$\emptyset\:\vdash$ 1:Nat}
	\LeftLabel{\textsc{(TYPE-RECORD)}}
	\BinaryInfC{$\emptyset\:\vdash$ \{l=0, l'=1\}:\{l:Nat, l':Nat\}}
	\AxiomC{(SUBWIDTH)}
	\LeftLabel{\textsc{}}
	\UnaryInfC{\{l:Nat, l':Nat\}<:\{l:Nat\}}
	\LeftLabel{\textsc{(SUBSUMPTION)}}
	\BinaryInfC{$\emptyset\:\vdash$ \{l=0, l'=1\}:\{l:Nat\}}
	\LeftLabel{\textsc{(APP)}}
	\BinaryInfC{$\emptyset\:\vdash$ (fn r:\{l:Nat\}.r.l + 2) \{l= 0, l'= 1\}:Nat}
\end{prooftree} }}

\subsection*{Esercizio 7.4}
Quale potrebbbe essere la relazione di sottotipo dei variant types? 
 
\subsubsection*{Svolgimento} 
 \begin{prooftree} 
	\AxiomC{$\Gamma$ $\vdash$ M : $T_j$}
	\AxiomC{j $\in$ 1..n+k}	
	\LeftLabel{\textsc{(SUBTYPE VARIANT)}}
	\BinaryInfC{$\Gamma$ $\vdash$ <$l_j$ = M> : <$l_i$ : $T_i$\textsuperscript{i $\in$ 1..n}>} 
\end{prooftree} 

In questo modo, modificando TYPE VARIANT, si ottiene il subtyping in larghezza (WIDTH). Per ottenere il subtyping in profondita' (DEPTH) basta usare SUBSUMPTION.

\subsection*{Esercizio 7.5}
Quali proprieta' del sistema di tipi si perderebbero se avessimo definito la relazione di
subtyping con una regola di troppo? E se l'avessimo definita con una regola in meno? E se l'avessimo definita con una regola in meno? Se avessimo definito le regole in modo tale che \{l : Nat\} <: \{l : Nat, l' : Nat\}, quale proprieta' del sistema non sarebbe piu' vera? Identificarla e darne un controesempio.
\textbf{{\color{red} DA FARE}}

\subsection*{Esercizio 7.12}
Ridimostrare il teorema di progressione, preservazione, il substitution lemma e il teorema
di safety per il linguaggio con i record e il subtyping. 
\textbf{{\color{red} DA FARE}}

\subsection*{Esercizio 7.16}
Trovare due termini M e N tali che M $\rightarrow$ N, $\Gamma\vdash$ M : T, $\Gamma\vdash$ N : S con S<::T e T<::S, cioe' esibire un caso in cui il tipo di un termine decresce durante la computazione. 
\textbf{{\color{red} DA FARE}}
\section{Featherweight Java (note 11)} 
\subsection*{Esercizio 7.1}\
\\ 
Si noti che una class table pu\`o contenere definizioni di classi mutuamente ricorsive.
\linebreak Scrivere un esempio:


\vspace{0,5cm}
\noindent
\subparagraph{Svolgimento}:
Definiamo le due seguenti Classi

\begin{lstlisting}
Class A extends B {A(){super();} }

Class B extends A {B(){super();} }

\end{lstlisting}

\noindent
A e B sono mutamente ricorsive. La relativa Class Table non \`e per\`o ben fatta, in quanto contiene una relazione di sottotipo indotta con cicli.

\vspace{1cm}
\subsection*{Esercizio 7.2}\
\\ 
Dato il seguente codice:
\begin{lstlisting}
class A extends Object { A(){ super(); } }

class B extends Object { B(){ super(); } }

class Pair extends Object { 
	Object fst; Object snd;
	
	Pair(Object fst, Object snd){
		super(); this.fst=fst; this.snd=snd; 
	} 
	
	Pair setfst(Object newfst){ return new Pair(newfst, this.snd); }
}

\end{lstlisting}

\vspace{0,5cm}
Descrivere la semantica operazionale dei seguenti termini:

\begin{itemize}
\item new Pair(new A(), new B()).snd
\item (Pair) new Pair(new A(), new B())
\item new Pair(new A(),new B()).setfst(new B())
\item (Pair) (new Pair(new Pair(new A(), new B()), new A()).fst)).snd
\item (B) ((A)new C())
\end{itemize}

\vspace{0,5cm}
\subparagraph{Svolgimento}:

Per motivi di spazio i seguenti termini saranno abbreviati con delle lettere
\begin{itemize}
\item[$\ast$] K $\equiv$ new Pair(new A(),new B())
\item[$\ast$] H $\equiv$ Pair setfst(Object newfst) \{ return new Pair(newfst, this.snd) 
\item[$\ast$] J $\equiv$ CT(Pair)= class Pair extends Object {Object fst, Object snd; Pair(Object fst, Object snd),Pair setfst(Object newfst)}
\end{itemize}

\vspace{0,5cm}

\begin{itemize}
\item \fbox{new Pair(new A(), new B()).snd}

\begin{prooftree}
	\AxiomC{J}
	\AxiomC{fields(Object) = $\emptyset$}	
	\BinaryInfC{fields(Pair) = (fst : Object, snd : Object)}
	\LeftLabel{(PROJNEW)}	
	\AxiomC{snd $\in$ \{fst , snd\} }
	\BinaryInfC{new Pair(new A(), new B()).snd $\rightarrow$ new B()}
	\end{prooftree}
	
\vspace{1,5cm}

\item \fbox{(Pair) new Pair(new A(), new B())}


\begin{prooftree}
	\AxiomC{}	
	\RightLabel{(REFLEX)}	
	\UnaryInfC{Pair <: Pair}
	\LeftLabel{(CASTNEW)}
	\UnaryInfC{(Pair) new Pair(new A(), new B()) $\rightarrow$ new Pair(new A(), new B())}
	\end{prooftree}
\vspace{1,5cm}

\item \fbox{new Pair(new A(),new B()).setfst(new B())}



\begin{prooftree}
	\AxiomC{J}
	\AxiomC{H}	
	\BinaryInfC{mbody(setfst, Pair)= (newfst, new Pair(newfst, this.snd))}
	\LeftLabel{(INVKNEW)}	
	\AxiomC{|fst|=|newB()|}
	\BinaryInfC{K.setfst(new B()) $\rightarrow$ new Pair(newfst,this.snd)\{fst=new B(),this=K\}}
\end{prooftree}

\vspace{1,5cm}

	
\item \fbox{((Pair) (new Pair(new Pair(new A(), new B()), new A()).fst)).snd}

\vspace{0,5cm}
Passo 1

\begin{prooftree}
	\AxiomC{J}
	\AxiomC{field(Object) = $\emptyset$}
	\BinaryInfC{fields (Pair) = \{Object fst,Object snd\} }
	\AxiomC{$fst \in \tilde{f}$}
	\LeftLabel{(PROJNEW)}	
	\BinaryInfC{(new Pair(K, new A()).fst) $\rightarrow$ K }
	\LeftLabel{(CAST)}	
	\UnaryInfC{((Pair) (new Pair(K, new A()).fst)) $\rightarrow$ (Pair) K  }
	\LeftLabel{(FIELD)}
	\UnaryInfC{((Pair) (new Pair(K, new A()).fst)).snd $\rightarrow$ ((Pair) K).snd}
\end{prooftree}

\vspace{0,5cm}
Passo 2

\begin{prooftree}
	\AxiomC{}
	\LeftLabel{(REFLEX)}	
	\UnaryInfC{Pair <: Pair}
	\RightLabel{(CASTNEW)}	
	\UnaryInfC{((Pair) K $\rightarrow$ K }
	\LeftLabel{(FIELD)}
	\UnaryInfC{((Pair) K).snd $\rightarrow$ K.snd}
\end{prooftree}

\vspace{0,5cm}
Passo 3

\begin{prooftree}
	\AxiomC{J}
	\AxiomC{field(Object) = $\emptyset$}
	\BinaryInfC{fields (Pair) = \{Object fst,Object snd\} }
	\AxiomC{snd $\in$ $\widetilde{f}$}	
	\LeftLabel{((PROJNEW))}
	\BinaryInfC{(new Pair(new A(), new B()).snd $\rightarrow$ new B()}
\end{prooftree}

\vspace{0,5cm}

\item \fbox{(B) ((A)new C())}
\vspace{0,5cm}

Possiamo fare due ipotesi di subtyping: 
\vspace{0,5cm}	
	\begin{itemize}
	\item[-] C <: A <: B
			
Passo 1
\vspace{0,3cm}
	\begin{prooftree}
		\AxiomC{CT(C)=class C extends A\{...\}}	
		\UnaryInfC{C <: A}
		\RightLabel{(CASTNEW)}	
		\UnaryInfC{(A)new C() $\rightarrow$ new C() }
		\LeftLabel{(CAST)}
		\UnaryInfC{(B) ((A)new C()) $\rightarrow$ (B) new C() }
	\end{prooftree}				
\vspace{1cm}

Passo 2
\vspace{0,5cm}
	\begin{prooftree}
		
		\AxiomC{CT(A)=class A extends B\{...\}}
		\UnaryInfC{A <: B}
		\AxiomC{CT(C)=class C extends A\{...\}}
		\UnaryInfC{C <: A}	
		\BinaryInfC{C <: B}
		\LeftLabel{(CASTNEW)}
		\UnaryInfC{(B) new C() $\rightarrow$ new C() }
	\end{prooftree}	
\vspace{1,5cm}

	\item[-] C <: B <: A
	
	Passo 1
\vspace{0,5cm}
	\begin{prooftree}
		\AxiomC{CT(B)=class B extends A\{...\}}
		\UnaryInfC{B <: A}
		\AxiomC{CT(C)=class C extends B\{...\}}
		\UnaryInfC{C <: B}	
		\BinaryInfC{C <: A}
		\RightLabel{(CASTNEW)}	
		\UnaryInfC{(A)new C() $\rightarrow$ new C() }
		\LeftLabel{(CAST)}
		\UnaryInfC{(B) ((A)new C()) $\rightarrow$ (B) new C() }
	\end{prooftree}				
\vspace{1cm}

Passo 2
\vspace{0,3cm}
	\begin{prooftree}
		\AxiomC{CT(C)=class C extends B\{...\}}	
		\UnaryInfC{C <: B}
		\LeftLabel{(CASTNEW)}
		\UnaryInfC{(B) new C() $\rightarrow$ new C() }
	\end{prooftree}	
		\end{itemize}
\end{itemize}

\vspace{1cm}
\subsection*{Esercizio 7.3}\
\\
\
Scrivere un programma con override di un metodo e descriverne la valutazione, evidenziando il binding dinamico
per la chiamata del metodo riscritto.

\vspace{0,5cm}
\subparagraph{Svolgimento}:

\begin{lstlisting}
class A extends Object{ 

	Nat fst; Nat snd;
 
	A (Nat fst, Nat snd){ 
		super(); this.fst=fst; this.snd=snd; 
	}

	Nat print() { return fst; } 
}

class B extends A{  
	B (Nat fst, Nat snd){ 

	super(); this.fst=fst; this.snd=snd; 
	}
} 

Nat print() { return snd; }
 

\end{lstlisting}    

\vspace{0,6cm}

consideriamo ora il seguente termine:

\vspace{0,3cm}
\hspace{0,3cm}\textit{((A) new B(10,4)).print()}

\vspace{0,3cm}
Per motivi di spazio il seguente termine sar\`a abbreviato con una lettera
\begin{itemize}
\item[$\ast$] J $\equiv$ CT(B)= class B extends A {Object fst, Object snd; B(Nat fst, Nat snd),Nat print()}
\end{itemize}

\vspace{1cm}
Passo 1
\vspace{0,3cm}
	\begin{prooftree}
		\AxiomC{J}	
		\UnaryInfC{B <: A}
		\RightLabel{(CASTNEW)}
		\UnaryInfC{(A) new B(10,4) $\rightarrow$ new B(10,4)}
		\LeftLabel{(INVKRECV)}
		\UnaryInfC{((A) new B(10,4)).print() $\rightarrow$ new B(10,4).print()}
	\end{prooftree}	
		


\vspace{0,5cm}
Passo 2
\vspace{0,3cm}
	\begin{prooftree}
		\AxiomC{J}
		\AxiomC{Nat print() \{ return snd \} $\in$ $\widetilde{M}$}	
		\BinaryInfC{mbody($\emptyset$,B)=(snd, return snd)}
		\AxiomC{$\emptyset$ = $\emptyset$}
		\LeftLabel{(INVKNEW)}
		\BinaryInfC{ new B(10,4).print() $\rightarrow$ 4}
	\end{prooftree}	

\vspace{1cm}
\subsection*{Esercizio 7.4}\
\\
\
Perch\`e c'\`e una regola di tipo sia per l'Up-cast che per il Downcast, mentre c'\`e la sola regola di valutazione per
Up-cast, nella semantica operazionale?

\vspace{0,5cm}
\subparagraph{Svolgimento}:

\vspace{0,3cm}
L'operazione di Up-cast, applicata su termini non stuck, non produrr\`a mai termini stuck in quanto effettua una conversione da un tipo pi\`u informativo(sottoclasse) ad un tipo meno informativo(superclasse).
Ci\`o non \`e sempre valido con l'operazione di Down-cast, la quale effettua invece una conversione da un tipo meno informativo(superclasse) ad uno che richiede pi\`u informazione (sottoclasse), con conseguente mancanza di valori da assegnare ai campi dati della sottoclasse stessa.
    
La regola di typing di Down-cast, \`e inserita per permettere la valutazioni di termini, che il compilatore non considerebbe ben tipati anche se a run-time non evolvono in un termine stuck.

\vspace{0,5cm}
Consideriamo ad esempio il seguente termine: 

	- (B)((C) new A())
	
	con la seguente relazione di sottotipo (A<:B<:C)

\vspace{0,5cm}	
Valutazione di tipo:

\vspace{0,3cm}
\begin{prooftree}
		\AxiomC{$|\emptyset| = |\emptyset|$ $fields(A)=\emptyset$}
		\LeftLabel{(NEW)}
		\UnaryInfC{$\emptyset \vdash $ new A(): A }
		\AxiomC{A <: C}
		\LeftLabel{(UP-CAST)}
		\BinaryInfC{$\emptyset \vdash $ (C)new A(): C }		
		\AxiomC{B <: C}	
		\AxiomC{C $\neq$ B }
		\RightLabel{(DOWN-CAST)}
		\TrinaryInfC{$\emptyset \vdash $ (B) ((C)new A()): B }
	\end{prooftree}
	
\vspace{1cm}	
Applicazione regole di semantica operazionale:

\vspace{0,3cm}
Passo 1
\begin{prooftree}
		\AxiomC{CT(A)=class A extends B\{...\}}
		\UnaryInfC{A <: B}
		\AxiomC{CT(B)=class B extends C\{...\}}
		\UnaryInfC{B <: C}	
		\BinaryInfC{A <: C }
		\RightLabel{(CASTNEW)}	
		\UnaryInfC{(C)new A() $\rightarrow$ new A() }
		\LeftLabel{(CAST)}
		\UnaryInfC{(B) ((C)new A()) $\rightarrow$ (B) (new A()) }
	\end{prooftree}	

\vspace{0,3cm}
Passo 2

	\begin{prooftree}
		\AxiomC{CT(A)=class A extends B\{...\}}	
		\UnaryInfC{A <: B}
		\LeftLabel{(CASTNEW)}
		\UnaryInfC{(B) new A() $\rightarrow$ new A() }
	\end{prooftree}	

\vspace{0,5cm}
Risulta essere ben tipato grazie anche all'utilizzo della regola di tipo DOWN-CAST, ed a Run-time effettivamente non evolve in un termine stuck.




\vspace{1cm}			 
\subsection*{Esercizio 7.5}\
\\
\
Ha senso aggiungere a FJ la regola di sub-typing per i tipi freccia A$\rightarrow$B?


\vspace{0,5cm}
\noindent
\subparagraph{Svolgimento}:

\vspace{0,3cm}
In FJ le funzioni non appartengono all'insieme Termini t della sintassi del linguaggio.

I tipi freccia li troviamo solamente all'interno di metodi, i quali hanno gi\`a una regola di typing (INVK) che, essendo il sistema di  regole sintax-direct (o algoritmico), integra direttamente la regola di sub-typing.

%\vspace{1cm}			 
%\subsection*{\fbox{Esercizio 7.12}}\
%\\
%\
%Aggiungere a FJ il termine ClassCastException: come cambia la semantica operazionale? Le regole di tipo? Il teorema di Safety e i teoremi di Preservazione e Progressione? Aggiungere
%in seguito anche la possibilit\`a di gestire le eccezioni.

%\vspace{0,5cm}
%\noindent
%\textbf{Svolgimento}:
 
\section{Imperative Featherweight Java (note 12)}
\vspace{1cm}

\subsection*{\fbox{Esercizio 8.1}}
Scrivere una regola per eliminare i riferimenti non piu' utilizzati.


Svolgimento:


I riferimenti agli oggetti che non vengono piu' utilizzati all'interno di un termine e, possono essere eliminati.

consideriamo la seguente regola:

\begin{prooftree}
	\AxiomC{o $\notin$ fref($e$)}
	\AxiomC{o $\notin$ Codom($\sigma$)}	
	\BinaryInfC{$\langle\sigma\:[o\mapsto(C,f\widetilde{:}v)]\:,\:e\rangle$ $\rightarrow$ $\langle\sigma\:,\:e\rangle$}
	\end{prooftree}

\vspace{0,5cm}

\begin{itemize}
\item fref($e$) = Insieme dei riferimenti a tutti gli oggetti utilizzati nel termine $e$
\item Codom($\sigma$) = Insieme degli oggetti che sono riferiti da altri oggetti presenti nella memoria.
\end{itemize}

\vspace{1,5cm}



\subsection*{\fbox{Esercizio 8.2}}
Descrivere il comportamento del seguente programma:
 
\begin{lstlisting}
class D extends Object {

	Object f; 
	
	D(Object f) {super(); this.f=f;}
 
	Object m() {return this;}
	
}


class C extends D {

	C(Object f) {super();}
 
	Object m() {return this;}

}

Object z=new Object();

C x=new C(z);

C y=new D(x);

x.m(); x=y; x.m()

y.f=new Object();

z=null; x.f=z; y.f.m();

\end{lstlisting}

Risoluzione:

Definiamo la configurazione iniziale sostituendo allo store $\sigma$ l'insieme vuoto($\emptyset$) ed all'espressione $e$ l'insieme delle istruzioni del programma. Otteniamo dunque la seguente configurazione iniziale:


$\langle\emptyset\:,\:Object\:z=new\:Object();\:C\:x=new\:C(z);\:.....\rangle$

\begin{itemize}

\vspace{0,5cm}
\item Creazione oggetto $o_1$ da memorizzare nella variabile z

$\rightarrow$ $\langle\emptyset\:,\:Object\:z=o_1;\:C\:x=new\:C(z);\:.....\rangle$

\vspace{0,5cm}

\begin{prooftree}
	\AxiomC{$o_1 \notin Dom(\emptyset)$}
	\AxiomC{field(Object)= $\emptyset$}
	\AxiomC{$|\emptyset| = |\emptyset|$ }
	\RightLabel{(NEW)}
	\TrinaryInfC{$\langle\emptyset,\:new Object()\rangle$ $\rightarrow$ $\langle\emptyset\:[o_1\mapsto (Object)]\:,\:o_1\rangle$}
	\LeftLabel{(CONG)}
	\UnaryInfC{$\langle\emptyset\:,E[t]\rangle$ $\rightarrow$ $\langle\emptyset\:[o_1\mapsto (Object)]\:,\:E[o_1]\rangle$}
	\end{prooftree}
\vspace{1cm}

		\begin{itemize}
		\item[-] E[] $\equiv$ Object z = []
		\item[-] $t$ $\equiv$ new Object() 
		\end{itemize}

\vspace{0,5cm}
\item Associazione dell'oggetto $o_1$ a z

$\rightarrow$ $\langle\emptyset\:\:[o_1\:\mapsto\:(Object)]\:\:,\:C\:x=new\:C(z);\:.....\rangle$

\begin{prooftree}
	\AxiomC{$z\:\notin Dom(\sigma)$ }
	\RightLabel{(NEW)}
	\UnaryInfC{$\langle\sigma\:,\:Object\:z=\:o_1\:;\:e\rangle$ $\rightarrow$ $\langle\sigma\:[z\mapsto o_1]\:,\:e\rangle$}
	\LeftLabel{(CONG)}
	\UnaryInfC{$\langle\sigma\:,E[t]\rangle$ $\rightarrow$ $\langle\sigma\:[z\:\mapsto\:o_1]\:,\:E[o_1]\rangle$}
	\end{prooftree}
\vspace{1cm}

		\begin{itemize}
		\item[-] E[] $\equiv$ []
		\item[-] $t$ $\equiv$ Object z=$o_1$ 
		\item[-] $\sigma\:\equiv\:\emptyset\:\:[o_1\:\mapsto\:(Object)]$
		\end{itemize}


\vspace{0,5cm}
\item Dereferenzazione della varibile z per creare l'oggetto $o_2$

$\rightarrow$ $\langle\emptyset\:\:[o_1\:\mapsto\:Object]\:[z\:\mapsto:o_1]\:,\:C\:x=new\:C(o_1);\:.....\rangle$

\vspace{0,5cm}
\item Creazione dell'oggetto $o_2$ da associare nella variabile x

$\rightarrow$ $\langle\emptyset\:\:[o_1\:\mapsto\:Object]\:[z\:\mapsto\:o_1]\:[o_2\:\mapsto\:(C,f:o_1)]\:,\:C\:x=new\:C(z);\:.....\rangle$

\vspace{0,5cm}
\item Associazione dell'oggetto $o_2$ ad x

$\rightarrow$ $\langle\emptyset\:\:[o_1\:\mapsto\:Object]\:[z\:\mapsto\:o_1]\:[o_2\:\mapsto\:(C,f:o_1)]\:[x\:\mapsto\:o_2]\:,\:C\:y=new\:D(x);\:.....\rangle$

\vspace{0,5cm}
\item Dereferenzazione della variabile x per creare l'oggetto $o_2$

$\rightarrow$ $\langle\emptyset\:\:[o_1\:\mapsto\:Object]\:[z\:\mapsto\:o_1]\:[o_2\:\mapsto\:(C,f:o_1)]\:[x\:\mapsto\:o_2]\:,\:C\:y=new\:D(o_2);\:.....\rangle$

\vspace{0,5cm}
\item Creazione dell'oggetto $o_3$ di tipo D da associare nella variabile y di tipo C. La classe C e' un sottotipo della classe D, pero' non aggiunge nessun nuovo campo dati e nessun nuovo metodo, effettua solamente l'override del metodo m().

$\rightarrow$ $\langle\emptyset\:\:[o_1\:\mapsto\:Object]\:[z\:\mapsto\:o_1]\:[o_2\:\mapsto\:(C,f:o_1)]\:[x\:\mapsto\:o_2]\:[o_3\:\mapsto\:(D,f:o_2)]\:,\:C\:y=o_3;\:.....\rangle$

\vspace{0,5cm}
\item Associazione dell'oggetto $o_3$ ad y

$\rightarrow$ $\langle\emptyset\:\:[o_1\:\mapsto\:Object]\:[z\:\mapsto\:o_1]\:[o_2\:\mapsto\:(C,f:o_1)]\:[x\:\mapsto\:o_2]\:[o_3\:\mapsto\:(D,f:o_2)]\:[y\:\mapsto\:o_3]\:,\:x.m();.....\rangle$

\vspace{0,5cm}
\item Dereferenzazione di x per effettuare l'invocazione del metodo m().

$\rightarrow$ $\langle\emptyset\:\:[o_1\:\mapsto\:Object]\:[z\:\mapsto\:o_1]\:[o_2\:\mapsto\:(C,f:o_1)]\:[x\:\mapsto\:o_2]\:[o_3\:\mapsto\:(D,f:o_2)]\:[y\:\mapsto\:o_3]\:,\:o_2.m();.....\rangle$

\vspace{0,5cm}
\item Invocazione del metodo m() della classe C perche' la funzione mbody effettua la scelta di tale metodo in base alla class-table dell'oggetto di invocazione, il metodo restituisce l'oggetto stesso quindi $o_2$.

Lo store rimane inalterato e si ha solo il passaggio all'istruzione successiva.

$\rightarrow$ $\langle\emptyset\:\:[o_1\:\mapsto\:Object]\:[z\:\mapsto\:o_1]\:[o_2\:\mapsto\:(C,f:o_1)]\:[x\:\mapsto\:o_2]\:[o_3\:\mapsto\:(D,f:o_2)]\:[y\:\mapsto\:o_3]\:,\:x=y;.....\rangle$

\vspace{0,5cm}
\item Dereferenzazione della variabile y.

$\rightarrow$ $\langle\emptyset\:\:[o_1\:\mapsto\:Object]\:[z\:\mapsto\:o_1]\:[o_2\:\mapsto\:(C,f:o_1)]\:[x\:\mapsto\:o_2]\:[o_3\:\mapsto\:(D,f:o_2)]\:[y\:\mapsto\:o_3]\:,\:x=o_3;.....\rangle$

\vspace{0,5cm}
\item Associazione dell'oggetto $o_3$ di tipo D alla variabile x. Si effetua una modifica allo store in quanto la variabile x gia' esisteva ed era associata all'oggetto $o_2$.

$\rightarrow$ $\langle\emptyset\:\:[o_1\:\mapsto\:Object]\:[z\:\mapsto\:o_1]\:[o_2\:\mapsto\:(C,f:o_1)]\:\underline{[x\:\mapsto\:o_3]}\:[o_3\:\mapsto\:(D,f:o_2)]\:[y\:\mapsto\:o_3]\:,\:x.m();.....\rangle$

\vspace{0,5cm}
\item Dereferenzazione di x per effettuare l'invocazione del metodo m().

$\rightarrow$ $\langle\emptyset\:\:[o_1\:\mapsto\:Object]\:[z\:\mapsto\:o_1]\:[o_2\:\mapsto\:(C,f:o_1)]\:[x\:\mapsto\:o_3]\:[o_3\:\mapsto\:(D,f:o_2)]\:[y\:\mapsto\:o_3]\:,\:o_3.m();.....\rangle$

\vspace{0,5cm}
\item Invocazione del metodo m() della classe D perche' la funzione mbody effettua la scelta di tale metodo in base alla class-table dell'oggetto di invocazione, il metodo restituisce l'oggetto stesso quindi $o_3$.

Lo store rimane inalterato e si ha solo il passaggio all'istruzione successiva.

$\rightarrow$ $\langle\emptyset\:\:[o_1\:\mapsto\:Object]\:[z\:\mapsto\:o_1]\:[o_2\:\mapsto\:(C,f:o_1)]\:[x\:\mapsto\:o_3]\:[o_3\:\mapsto\:(D,f:o_2)]\:[y\:\mapsto\:o_3]\:,\:y.f=new\:Object();.....\rangle$

\vspace{0,5cm}
\item Dereferenzazione di y

$\rightarrow$ $\langle\emptyset\:\:[o_1\:\mapsto\:Object]\:[z\:\mapsto\:o_1]\:[o_2\:\mapsto\:(C,f:o_1)]\:[x\:\mapsto\:o_3]\:[o_3\:\mapsto\:(D,f:o_2)]\:[y\:\mapsto\:o_3]\:,\:o_3.f=new\:Object();.....\rangle$

\vspace{0,5cm}
\item Creazione dell'oggetto $o_4$ di tipo Object da associare alla struttura dati f dell'oggetto $o_3$, associato alla variabile y, che referenziava l'oggetto $o_2$.

$\rightarrow$ $\langle\emptyset\:\:[o_1\:\mapsto\:Object]\:[z\:\mapsto\:o_1]\:[o_2\:\mapsto\:(C,f:o_1)]\:[x\:\mapsto\:o_3]\:\underline{[o_3\:\mapsto\:(D,f:o_4)]}\:[y\:\mapsto\:o_3]\:[o_4\:\mapsto\:Object]\:,\:o_3.f=o_4;.....\rangle$

\vspace{0,5cm}
\item Associazione alla struttura dati dell'oggetto $o_3$ associato alla variabile y.

Lo store rimane inalterato e si ha solo il passaggio all'istruzione successiva.

$\rightarrow$ $\langle\emptyset\:\:[o_1\:\mapsto\:Object]\:[z\:\mapsto\:o_1]\:[o_2\:\mapsto\:(C,f:o_1)]\:[x\:\mapsto\:o_3]\:[o_3\:\mapsto\:(D,f:o_4)]\:[y\:\mapsto\:o_3]\:[o_4\:\mapsto\:Object]\:,\:z=null;.....\rangle$

\vspace{0,5cm}
\item Associazione di \textbf{null} alla variabile z. 

$\rightarrow$ $\langle\emptyset\:\:[o_1\:\mapsto\:Object]\:\underline{[z\mapsto\:null]}\:[o_2\:\mapsto\:(C,f:o_1)]\:[x\:\mapsto\:o_3]\:[o_3\:\mapsto\:(D,f:o_4)]\:[y\:\mapsto\:o_3]\:[o_4\:\mapsto\:Object]\:,\:x.f=z;.....\rangle$

\vspace{0,5cm}
\item Dereferenzazione di z.

$\rightarrow$ $\langle\emptyset\:\:[o_1\:\mapsto\:Object]\:\underline{[z\mapsto\:null]}\:[o_2\:\mapsto\:(C,f:o_1)]\:[x\:\mapsto\:o_3]\:[o_3\:\mapsto\:(D,f:o_4)]\:[y\:\mapsto\:o_3]\:[o_4\:\mapsto\:Object]\:,\:x.f=null;.....\rangle$

\vspace{0,5cm}
\item Dereferenzazione di x.

$\rightarrow$ $\langle\emptyset\:\:[o_1\:\mapsto\:Object]\:\underline{[z\mapsto\:null]}\:[o_2\:\mapsto\:(C,f:o_1)]\:[x\:\mapsto\:o_3]\:[o_3\:\mapsto\:(D,f:o_4)]\:[y\:\mapsto\:o_3]\:[o_4\:\mapsto\:Object]\:,\:o_3.f=null;.....\rangle$

\vspace{0,5cm}
\item Assegnazione di $null$ ad $o_3.f$. Considerando che la variabile y si riferisce all'oggetto $o_3$, abbiamo che anche y.f = $null$

$\rightarrow$ $\langle\emptyset\:\:[o_1\:\mapsto\:Object]\:[z\mapsto\:null]\:[o_2\:\mapsto\:(C,f:o_1)]\:[x\:\mapsto\:o_3]\:[o_3\:\mapsto\:(D,f:null)]\:[y\:\mapsto\:o_3]\:[o_4\:\mapsto\:Object]\:,\:y.f.m();\rangle$

\vspace{0,5cm}
\item Dereferenzazione di y.

$\rightarrow$ $\langle\emptyset\:\:[o_1\:\mapsto\:Object]\:[z\mapsto\:null]\:[o_2\:\mapsto\:(C,f:o_1)]\:[x\:\mapsto\:o_3]\:[o_3\:\mapsto\:(D,f:null)]\:[y\:\mapsto\:o_3]\:[o_4\:\mapsto\:Object]\:,\:o_3.f.m();\rangle$

\vspace{0,5cm}
\item Il programma termina restituendo NPE, in quanto si tenta di invocare il metodo m() su $null$.

\end{itemize}


\vspace{1,5cm}
\subsection*{\fbox{Esercizio 8.3}}
\
\\
\
\vspace{0,6cm}
Dare una derivazione di tipo per il termine C x=$o_1$; $o_1$.f=y; x=$o_2$.
\vspace{0,6cm}

\begin{prooftree}
	\AxiomC{$o_1\::\:F\:\in\:\Gamma$}
	\RightLabel{(OID)}
	\UnaryInfC{$\Gamma\:\vdash\:o_1:F$}
	\AxiomC{(A)}
	\RightLabel{(CLASS)}
	\UnaryInfC{$F\:<:\:C $}
	\AxiomC{(B)}
	\AxiomC{(C)}
	\UnaryInfC{$\Gamma\:,\:x=o_2\::T$}
	\RightLabel{(SEQ)}
	\BinaryInfC{$\Gamma\:,\:x:C\:\vdash\:o_1.f=y,\:x=o_2\::\:T$}
	\LeftLabel{(DICHIAR)}
	\TrinaryInfC{$C\:x=o_1;\:o_1.f=y;\:x=o_2\::\:T$}
	\end{prooftree}
	
\vspace{2,5cm}

\begin{itemize}

\item[(A)]
\
\
($CT(F)\:=\:class\:F\:extends\:C\:\{..\}$)
\vspace{1cm}

\item[(B)]
\
\\
\
\vspace{0,3cm}
\begin{bprooftree}
	\AxiomC{$o_1\::\:F\:\in\:\Gamma$}
	\RightLabel{(Oid)}
	\UnaryInfC{$\Gamma\:,\:x:C\:\vdash\:o_1\::\:F$}
	\AxiomC{$f\:\in\:field(N)$}
	\RightLabel{(Fld)}
	\BinaryInfC{$\Gamma\:,\:x:C\:\vdash\:o_1.f\::\:N$}
	\AxiomC{$CT(N)..$}
	\RightLabel{(CL)}
	\UnaryInfC{$N\:<:\:M $}
	\AxiomC{$y\::\:M\:\in\:\Gamma$}
	\RightLabel{(Var)}
	\UnaryInfC{$\Gamma\:,\:x:C\:\vdash\:y\::\:M$}
	\LeftLabel{(Fld Ass)}
	\TrinaryInfC{$\Gamma\:,\:x:C\:\vdash\:o_1.f=y\::\:M$}
	\end{bprooftree}

\vspace{1cm}

\item[(C)]
\
\\
\\
\\
\
\vspace{0,8cm}
\begin{bprooftree}
	\AxiomC{$o_2\::\:T\:\in\:\Gamma$}
	\RightLabel{(OID)}
	\UnaryInfC{$\Gamma\:,\:x:C\:\vdash\:o_2\::\:T$}
	\AxiomC{$CT(T)= class T\:extends\:C\{..\}$}
	\RightLabel{(CL)}
	\UnaryInfC{$T\:<:\:C $}
	\AxiomC{$x\::\:C\:\in\:\Gamma$}
	\RightLabel{(Var)}
	\UnaryInfC{$\Gamma\:\vdash\:x\::\:C$}
	\LeftLabel{(Ass)}
	\TrinaryInfC{$\Gamma\:,\:x:C\:\vdash\:x=o_2\::\:T$}
	\end{bprooftree}

\end{itemize}







\newpage
\bibliographystyle{plain}
\bibliography{biblist}

\end{document} 
